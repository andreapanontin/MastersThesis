\documentclass[../Main]{subfiles}
\begin{document}
\section{Divided powers, exponentials and crystals}
\begin{defn}[Ideal with divided powers]\label{defn:dividedPowers}
	Let $A$ be a ring, and $I \triangleleft A$ an ideal of $A$.
	We say that $I$ is equipped with {\em divided powers} iff
	we are given a family of mappings $\left\{ \gamma_n \right\}_{n \geq 1}$,
	where $\gamma_n\colon I \to A$ for all $n \in \mathbb{N}$,
	satisfying, for all $\lambda \in A$ and $x,y \in I$, the following conditions:
	\begin{enumerate}
		\item $\gamma_0(x) = 1, \gamma_1(x) = x$ and 
			$\gamma_n(x) \in I$ for all $n \geq 2$;
		\item $\gamma_n(\lambda x) = \lambda^n \gamma_n(x)$;
		\item $\gamma_n(x) \cdot \gamma_m(x) =
			\frac{\left( m + n \right)!}{m! n!} \gamma_{m+n}(x)$;
		\item $\gamma_n(x+y) = \sum_{i=0}^{n} \gamma_{n-i}(x) \gamma_i(y)$;
		\item $\gamma_m(\gamma_n(x)) = 
			\frac{\left( mn \right)!}{\left( n! \right)^m m!} \gamma_{mn}(x)$.
	\end{enumerate}
	Given such a system, we say that $(I,\gamma)$
	is an {\em ideal with divided powers},
	where we denoted $\gamma \coloneqq \left\{ \gamma_n \right\}_{n \in \mathbb{N}}$.
	Moreover we might sometimes use the notation
	$x^{(n)} \coloneqq \gamma_n(x)$.
	Finally, to stress the ring we are working in, we might write
	$\left(A, I, \gamma\right)$ to denote $I \triangleleft A$
	an ideal with divided powers given by $\gamma$, and 
	we will refer to it as a {\em ring with divided powers}.
\end{defn}


\begin{defn}[Nilpotent divided powers]\label{defn:NilpotentDividedPowers}
	Given $\left(A, I, \gamma\right)$ as before, we say that the divided
	powers are nilpotent iff there is $N \in \mathbb{N}$ such that,
	for all $i_1 + \cdots + i_k \geq N$,
	the ideal generated by elements of the form
	\begin{equation*}
		\gamma_{i_1}(x_1) \cdot \cdots \cdot \gamma_{i_n}(x_k)
	\end{equation*}
	is zero.
\end{defn}


\begin{rem}[]
	We can make the following easy remarks.
\begin{enumerate}
\item Axiom $2$ of \cref{defn:dividedPowers}
	implies that $\gamma_n(0) = 0$ for all $n \in \mathbb{N}$.

\item Axioms $1$ and $3$ tell us that
	$n! \gamma_n(x) = x^n$.

\item Reasoning by induction one can show that
	\begin{equation*}
		\frac{\left( mn \right)!}{\left( n! \right)^m m!} =
		\prod_{k=1}^{m-1} \frac{\left( kn + n - 1 \right)!}{(kn)! (n-1)!}
	,\end{equation*}
	which shows that it is an integer.
	In fact it can be interpreted as the number of partition of a set with $mn$
	elements into $m$ subsets of $n$ elements each.

\item In \cref{defn:NilpotentDividedPowers},
	if we take $k = N$ and $i_1 = \ldots i_N = 1$,
	then, thanks to axiom $1$ of \cref{defn:dividedPowers}, the ideal $I$
	is nilpotent.
	In particular $I^N = (0)$.
\end{enumerate}
\end{rem}


\begin{ex}\leavevmode\vspace{-.2\baselineskip}
\begin{enumerate}
	\item Given any ring $A$, $(0)$ is an ideal with divided powers,
		with $\gamma_n(0) = 0$ for all $n \in \mathbb{N}$.

	\item If $A$ is a $\mathbb{Q}$-algebra, every ideal has a unique
		divided powers structure, given by $\gamma_n(x) = x^n/n!$.

	\item If $V$ is a discrete valuation ring of unequal characteristic $p$
		and uniformizer $\pi$, we can write $p = u \pi^e$,
		where $u$ an invertible element and $e$ the absolute ramification
		index of $V$.
		Then $\left( \pi \right)$ has a divided powers structure iff $e \leq p-1$.
		In such case $\gamma$ is unique, determined by
		$\gamma_n(x) \coloneqq x^n/n!$.
		In fact it is known that, denoted by $\nu_p$ the valuation
		of $\mathbb{Z}_{p}$ normalized to have value group $\mathbb{Z}$,
		and if $n = a_0 + \cdots + a_kp^k$ is the $p$-adic expansion
		of $n \in \mathbb{N}$, then
		\begin{equation*}
			\nu_p(n!) = 
			\frac{n - s_p(n)}{p-1}
		,\end{equation*}
		where $s_p(n) = a_0 + \cdots a_k$.
		Then, in order for $\gamma_n$ to all have values in $(\pi)$
		we need that $\nu(\gamma_n(x)) > 0$ for all $x \in (\pi)$.
		By axiom $2$ of \cref{defn:dividedPowers}
		it is enough to check it for $\pi$.
		Then, assuming $\nu$ is normalized to have values in $\mathbb{Z}$,
		we have
		\begin{equation*}
			\nu(\gamma_n(\pi)) = \nu(\pi^n/n!) =
			n - e \cdot \nu_p(n!) =
			n \cdot \frac{p - 1 - e}{p - 1} + e \cdot \frac{s_p(n)}{p - 1}
		.\end{equation*}
		It is clear that $\nu(\gamma_n(\pi)) > 0$ for all $n \in \mathbb{N}$
		iff $p - 1 - e \geq 0$ iff $e \leq p - 1$.
\end{enumerate}
\end{ex} 


\begin{defn}[Morphism of divided powers]
	A morphism of divided powers, denoted by
	\begin{equation*}
	\begin{tikzcd}[row sep = 0ex
		,/tikz/column 1/.append style={anchor=base east}
		,/tikz/column 2/.append style={anchor=base west}]
		u\colon \left(A, I, \gamma\right) 
		\arrow[r, "", rightarrow] &
		\left(B, J, \delta\right)
	,\end{tikzcd}
	\end{equation*} 
	is a ring homomorphism $u\colon A \to B$ such that
	$u(I) \subset J$ and $u(\gamma_n(x)) = \delta_n(u(x))$
	for all $x \in I$.
\end{defn}

\noindent
Let's now introduce the analogous of the symmetric algebra 
in the divided powers context:
\begin{thm}[{\cite[\S3, theorem 3.9]{Berthelot}}]
	Let $M$ be an $A$-module.
	Then there exists a divided powers algebra 
	$\left(\Gamma_A(M), \Gamma_A^+(M), \gamma\right)$ and an $A$-linear
	map $\varphi\colon M \to \Gamma_A^+(M)$ with the following
	universal property:
	given any other divided powers algebra
	$\left(B, J, \delta\right)$ and any $A$-linear
	map $\psi\colon M \to J$,
	then there is a unique divided powers morphism
	$\overline{\psi}\colon \Gamma_A(M) \to B$
	such that $\overline{\psi} \circ \varphi = \psi$.
	Moreover the divided powers algebra $\Gamma(M)$ satisfies:
\begin{enumerate}
	\item $\Gamma^+(M)$ is a graded algebra, with
		$\Gamma_A^0(M) = A$, $\Gamma_A^1(M) = M$
		and $\Gamma_A^+(M) = \oplus_{i \geq 1} \Gamma_A^i(M)$.

	\item The functor $M \mapsto \Gamma_A(M)$ is compatible with
		base change $A \to A'$, i.e. given any $A$algebra $A'$
		then
		\begin{equation*}
			\Gamma_{A'}(M \otimes_A A') \simeq A' \otimes_A \Gamma_A(M)
		.\end{equation*}

	\item Given a direct system of $A$-modules $\left\{ M_j \right\}_{j \in J}$ 
		and denoted $M = \varinjlim_j M_j$, then $\Gamma_A(M) = \varinjlim_{j} \Gamma_A(M_j)$.

	\item For any pair of $A$-algebras $M, N$, we have $\Gamma_A(M \oplus N)
		\simeq \Gamma_A(M) \otimes \Gamma_A(N)$.

	\item Denote $x^{(1)} \coloneqq \varphi(x)$ and $x^{(n)} \coloneqq \gamma_n(\varphi(x)) \in
		\Gamma^n(M)$, following notation of \cref{defn:dividedPowers}.
		Then the $A$-module $\Gamma^n(M)$ is generated by
		\begin{equation*}
		\left\{ x_1^{(q_1)} \cdots x_k^{(q_k)} \ \middle|\ 
		q_1 + \cdots + q_k = n \right\}
		.\end{equation*}
		Moreover, if $\{ x_i \}_{i \in I}$ is a basis
		for $M$, then $\{ x^{(n)}_i \}_{I \in I}$
		is a basis for $\Gamma^n(M)$,
		for all $n \geq 1$.
\end{enumerate}
	To make notation cleaner, when the base $A$ is clear, we
	will write $\Gamma(M)$ for $\Gamma_A(M)$.
\end{thm}


\noindent
The point of introducing all of these definitions is to allow one to define the following
inverse maps.
\begin{defn}[]
	Let $\left(A, I, \gamma\right)$ be a nilpotent divided powers ring.
	Then we can define two maps
	\begin{equation*}
	\begin{tikzcd}[row sep = 0ex
		,/tikz/column 1/.append style={anchor=base east}
		,/tikz/column 2/.append style={anchor=base west}]
		\exp\colon I \arrow[r, "", rightarrow] &
		\left( 1 + I \right)^*\\
		\log\colon \left( 1 + I \right)^* \arrow[r, "", rightarrow] &
		I
	,\end{tikzcd}
	\end{equation*} 
	given by 
	$\exp (x) \coloneqq \sum_{n\geq 0} \gamma_n(x)$
	and $\log (1+x) \coloneqq \sum_{n\geq 1} 
	(-1)^{n-1} \left( n-1 \right)! \gamma_n(x)$.
	Then, as outlined in \cite[Chapter III, \S1.6]{Messing},
	one checks that these maps are inverses to each other
	by reducing to the universal case $\widehat{\Gamma_{\mathbb{Z}}(\mathbb{Z})}$.
\end{defn}


\begin{rem}[]
	The above constructions can all be globalized to the case of
	$\mathcal{O}_S$ algebras and modules, where $S$ denotes a scheme.
	In particular one needs to replace $A$ by the above mentioned scheme $S$,
	$I$ by a quasi-coherent sheaf of ideals $\mathscr{I}$ of $\mathcal{O}_S$ and
	$M$ by a quasi-coherent $\mathcal{O}_S$-module.
	Then a system of divided powers on $\mathscr{I}$ is the data
	of, for all $U \subset S$ open, a system of divided powers of
	$\Gamma(U,\mathscr{I})$ in which divided power morphisms and restriction
	maps commute.
	Then we will call one such triple $\left(S, \mathscr{I}, \gamma\right)$
	a {\em scheme with divided powers}.

	Moreover one generalizes morphism of divided powers in the following way:
	let $\left(S, \mathscr{I}, \gamma\right)$ and $\left(S', \mathscr{I}', \gamma'\right)$
	be schemes with divided powers.
	A {\em morphism of divided powers}
	\begin{equation*}
	f\colon \left(S, \mathscr{I}, \gamma\right) \to 
	\left(S', \mathscr{I}', \gamma'\right)
	\end{equation*}
	is a morphism of schemes
	$f\colon S \to S'$ such that
	$f^{-1}(\mathscr{I}') \mathcal{O}_S \subset \mathscr{I}$
	and, for all $U' \subset S'$ open,
	\begin{equation*}
	\begin{tikzcd}[row sep = 0ex
		,/tikz/column 1/.append style={anchor=base east}
		,/tikz/column 2/.append style={anchor=base west}]
		f^\#(U')\colon 
		\left(\mathcal{O}_{S'}(U'), \mathscr{I}'(U'), \gamma'\right)
		\arrow[r, "", rightarrow] &
		\left(\mathcal{O}_{ S }(f^{-1}U'), \mathscr{I}(f^{-1}U'), \gamma\right)
	\end{tikzcd}
	\end{equation*} 
	is a morphism of divided powers rings.

	Moreover the divided powers algebra $\Gamma(M)$ is defined
	as the sheaf associated to the presheaf 
	$U \mapsto \Gamma_{\mathcal{O}_S(U)}(M(U))$.
	Finally the divided powers on an ideal $\mathscr{I} \subset \mathcal{O}_S$
	are said to be {\em nilpotent} iff,
	locally on $S$, they satisfy conditions in \cref{defn:NilpotentDividedPowers}.
\end{rem}


\subsection{}
The notation of this section will follow that of \cite[Capther III]{Messing}.
This means that it might not be consistent with our previous exposition.


\begin{defn}[Quasi-coherent (co)algebra]
	Let $S$ be a scheme.
\begin{enumerate}
\item We say that $U$ is a $\mathcal{O}_{ S }$ {\em algebra}
	iff it is an $\mathcal{O}_{ S }$-module
	which is also endowed with an $\mathcal{O}_{ S }$-algebra structure.
\item We say that $U$ is an $\mathcal{O}_{ S }$ {\em co-algebra}
	iff it is an $\mathcal{O}_{ S }$-algebra, endowed with
	morphisms of $\mathcal{O}_S$-algebras
	$\Delta\colon U \to U \otimes_{\mathcal{O}_{ S }} U$
	and $\eta\colon U \to \mathcal{O}_{ S }$
	satisfying the properties of Hopf algebra morphisms,
	as defined in \cref{defn:HopfAlgebra}.
\end{enumerate}
	If, moreover, $U$ is quasi-coherent as $\mathcal{O}_{ S }$-module,
	we say that it is {\em quasi-coherent} $\mathcal{O}_{ S }$ (co)-algebra.
\end{defn}


\begin{defn}[Cospec]
	Let $S$ be a scheme and $U$ an $\mathcal{O}_{ S }$ co-algebra.
	We define the functor 
	\begin{equation*}
	\begin{tikzcd}[row sep = 0ex
		,/tikz/column 1/.append style={anchor=base east}
		,/tikz/column 2/.append style={anchor=base west}]
		\mathrm{Cospec}(U)\colon 
		\mathsf{Sch}_{ S }\arrow[r, "", rightarrow] &
		\mathsf{Sets} \\
		S' \arrow[r, "", mapsto] & 
		\left\{ y \in \Gamma(S', U_{S'}) \ \middle|\ 
		\eta(y) = 1, \Delta(y) = y \otimes y\right\}
	.\end{tikzcd}
	\end{equation*} 
\end{defn}


\begin{rem}[{\cite[Chapter 3, \S2.1]{Messing}}]
	The functor $\mathrm{Cospec}(U)$ is a sheaf for the fpqc topology
	for all $\mathcal{O}_{ S }$ co-algebra $U$.
	As a consequence we obtain a covariant functor $U \mapsto \mathrm{Cospec}(U)$
	from the category of $\mathcal{O}_{ S }$ co-algebras to the category
	of fpqc sheaves on $S$.
	Moreover this functor is compatible with inverse images.
\end{rem}


\noindent
Let's now investigate the relation between $\mathrm{Cospec}$ and $\mathrm{Spec}$.
We need some preliminary definitions.


\begin{defn}[Internal hom of $\mathcal{O}_{ X }$-modules]\label{defn:iHomOXMod}
	Let $X$ be a scheme, and $\mathcal{F}, \mathcal{G}$ be two $\mathcal{O}_{ X }$-modules.
	We define the internal hom, $\mathcal{O}_{ X }$-module, as
	\begin{equation*}
	\begin{tikzcd}[row sep = 0ex
		,/tikz/column 1/.append style={anchor=base east}
		,/tikz/column 2/.append style={anchor=base west}]
		\mathcal{H}\mathrm{om}_{ \mathcal{O}_{ X } } \left( \mathcal{F}, \mathcal{G} \right)\colon 
		\mathsf{Op}(X)^{op} \arrow[r, "", rightarrow] &
		\mathcal{O}_{ X }\text{-}\mathsf{Mod} \\
		U \arrow[r, "", mapsto] & 
		\mathrm{Hom}_{ \left.\mathcal{O}_{ X }\right|_{U}  } \left( \left.\mathcal{F}\right|_{U}, 
				\left.\mathcal{G}\right|_{U} \right)
	,\end{tikzcd}
	\end{equation*} 
	where we denoted by $U \subset X$ an open subset and by
	$\mathcal{O}_{ X }\text{-}\mathsf{Mod}$ the category of $\mathcal{O}_{ X }$-modules.
	This is indeed a sheaf of abelian groups, see 
	\cite[\href{https://stacks.math.columbia.edu/tag/00AK}{Section 00AK}]{SP} for a reference.
	Moreover the $\mathcal{O}_{ X }$-module structure is given as follows.
	Fixed any $U \subset X, \varphi \in \mathrm{Hom}_{ \left.\mathcal{O}_{ X }\right|_{U}  } 
	\left( \left.\mathcal{F}\right|_{U} , \left.\mathcal{G}\right|_{U} \right)$ and
	$f \in \mathcal{O}_{ X }(U)$, we can define $f \varphi \in
	\mathrm{Hom}_{ \left.\mathcal{O}_{ X }\right|_{U}  } \left( \left.\mathcal{F}\right|_{U} , 
	\left.\mathcal{G}\right|_{U} \right)$ either by precomposing $\varphi$ by multiplication
	by $f$ on $\left.\mathcal{F}\right|_{U}$ or by postcomposing $\varphi$
	by multiplication by $f$ on $\left.\mathcal{G}\right|_{U}$.
\end{defn}


\begin{rem}[]
	In the following, in order to keep a consistent notation with
	\cite{Messing} we will use the following notation.
	Let $S$ be a scheme and $U$ an $\mathcal{O}_{ S }$-module
	and $S' \in \mathsf{Sch}_{ S }$, with structure morphism
	$f\colon S' \to S$.
	We denote by 
	\begin{equation*}
	U_{S'} \coloneqq f^*U = 
	f^{-1}U \otimes_{f^{-1}\mathcal{O}_{ S }} \mathcal{O}_{ S' }
	.\end{equation*}
\end{rem}


\begin{rem}[]\label{rem:CospecSections}
	We start out giving the following useful identification:
	let $U$ be a quasi coherent $\mathcal{O}_{ S }$ co-algebra.
	Then we have a one to one correspondence
	\begin{equation*}
	\Gamma(S', \mathrm{Cospec}(U)) \simeq
	\mathrm{Hom}_{ \mathcal{O}_{ S' }\text{-}\mathsf{co}\text{-}\mathsf{alg}} 
	\left( \mathcal{O}_{ S' }[X], U_{S'} \right)
	,\end{equation*}
	where $(\mathcal{O}_{ S }[X])(U) \coloneqq (\mathcal{O}_{ S }(U))[X]$ 
	is given the co-algebra structure of the additive group scheme, as seen in
	\cref{ex:AdditiveGroupScheme} of \cref{ex:AffineGroupSchemesExamples},
	and the right hand side denotes the morphisms of $\mathcal{O}_{ S' }$
	co-algebras which, analogously to morphisms of Hopf algebras, 
	are required to preserve the algebra structure and the morphism $\Delta$.
	More explicitly this identification associates
	$\varphi\colon \mathcal{O}_{ S' }[X] \to U_{S'}$ to
	$\varphi(U)(X) \in \Gamma(S', U_{S'})$.
	Finally it is clear that the above identification is functorial in $S'$.
\end{rem}


\noindent
In order to investigate more the notion of $\mathrm{Cospec}$ we need to 
be able to construct schemes starting from quasi-coherent $\mathcal{O}_{ S }$-algebras
or $\mathcal{O}_{ S }$-modules.


\begin{defn}[Relative spectrum and Vector bundles]
	Let fix a scheme $S$.
\begin{enumerate}
	\item Assume that $\mathscr{A}$ is a quasi-coherent $\mathcal{O}_{ S }$-algebra.
		We define the {\em relative spectrum} of $\mathscr{A}$, denoted by
		$\underline{\mathrm{Spec}}_S(\mathscr{A})$, as the gluing of
		$\mathrm{Spec}(\Gamma(U, \mathscr{A}))$, where $U$ ranges
		over all affine open subsets of $S$.

	\item  Let $E$ be a quasi-coherent sheaf of $\mathcal{O}_{ S }$-modules.
		Denote by $\mathrm{Sym}(E)$ the symmetric algebra associated to 
		$E$ (which, thanks to \cref{SheafqcSymExt} of
		\cref{rem:SheafSymExtProperties} is quasi-coherent).
		We define the {\em vector bundle} associated to $E$ as
		\begin{equation*}
			\mathbf{V}(E) \coloneqq
			\underline{\mathrm{Spec}}_S (\mathrm{Sym}(E))
		.\end{equation*}
\end{enumerate}
\end{defn}


\begin{rem}[]\leavevmode\vspace{-.2\baselineskip}
\begin{enumerate}
	\item Notice that the constructions outlined above can actually be carried out,
		as can be checked at 
		\cite[\href{https://stacks.math.columbia.edu/tag/01LL}{Section 01LL}]{SP}
		and
		\cite[\href{https://stacks.math.columbia.edu/tag/01M1}{Section 01M1}]{SP}.

	\item In the case of relative spectrum, as proved in 
		\cite[\href{https://stacks.math.columbia.edu/tag/01LP}{Lemma 01LP}]{SP},
		$\underline{\mathrm{Spec}}_S(\mathscr{A})$, where $\mathscr{A}$ is
		an $\mathcal{O}_{ S }$-algebra, is canonically an $S$-scheme.
		In fact there is a morphism of schemes
		\begin{equation*}
		\begin{tikzcd}[row sep = 0ex
			,/tikz/column 1/.append style={anchor=base east}
			,/tikz/column 2/.append style={anchor=base west}]
			\pi\colon \underline{\mathrm{Spec}}_S(\mathscr{A})
			\arrow[r, "", rightarrow] &
			S
		,\end{tikzcd}
		\end{equation*} 
		where, for all $U \subset S$ affine, $\pi^{-1}(U) \simeq \mathrm{Spec}(\mathscr{A}(U))$.

	\item Using notations as above, we see that $\mathbf{V}(E)$ is endowed with
		some extra structure: it inherits the grading of $\mathrm{Sym}(E)$
		thanks to
		\begin{equation*}
			\pi_* \mathcal{O}_{ \mathbf{V}(E) } =
			\bigoplus_{n \geq 0} \mathrm{Sym}^n(E)
		.\end{equation*}
		Then $\pi_* \mathcal{O}_{ \mathbf{V}(E) }$ is a graded
		$\mathcal{O}_{ S }$-algebra and $E$ is just the degree $1$ part
		of this.
\end{enumerate}
\end{rem}


\begin{rem}[]
	Moreover, for a finite locally free $\mathcal{O}_{ S }$-algebra $A$,
	we can see that $\mathrm{Cospec}(A^{\vee})$, as an fppf sheaf, is representable 
	by $\underline{\mathrm{Spec}}_S(A)$ and
	the category of finite locally free $S$-schemes
	is equivalent to the category of finite locally-free (co-commutative)
	$\mathcal{O}_{ S }$-algebras.

	In particular, given $A$ as above, we introduce its dual,
	finite locally free $\mathcal{O}_{ S }$-module 
	$A^{\vee} \coloneqq \mathcal{H}\mathrm{om}_{\mathcal{O}_{ S }} \left( A, \mathcal{O}_{ S } \right)$,
	defined as in \cref{defn:iHomOXMod}. 
	Then $A^{\vee}$ can be endowed with the structure of co-algebra, which is
	given following exactly the same ideas as in \cref{rem:dualHopfStructure}.

	Then, invoking \cref{rem:CospecSections}, we can construct 
	our desired isomorphism: let $S' \in \mathsf{Sch}_{ S }$
	\begin{align*}
		\Gamma(S', \mathrm{Cospec}(A^{\vee})) &\simeq
		\mathrm{Hom}_{ \mathcal{O}_{ S' }\text{-}\mathsf{co}\text{-}\mathsf{alg}} 
		\left( \mathcal{O}_{ S' }[X], A^\vee_{S'} \right) \simeq
		\mathrm{Hom}_{ \mathcal{O}_{ S' }}
		\left( A_{S'}, \mathcal{O}_{ S' } \right) \\
		&\simeq
		\Gamma(S', \underline{\mathrm{Spec}}_S(A))
	,\end{align*}
	where the last isomorphism holds by 
	\cite[\href{https://stacks.math.columbia.edu/tag/01LV}{Lemma 01LV}]{SP}.
	In fact we can conclude thanks to functoriality of the above isomorphisms.
\end{rem}


\begin{rem}[{\cite[Chapter III, \S2.1.3]{Messing}}]
	The above construction can be generalized to
	filtered direct limits.
	Let, in fact, $U = \varinjlim_i U_i$ be a filtered direct limit
	of co-algebras. One obtains an isomorphism
	\begin{equation*}
	\begin{tikzcd}[row sep = 0ex
		,/tikz/column 1/.append style={anchor=base east}
		,/tikz/column 2/.append style={anchor=base west}]
		\varinjlim_i \mathrm{Cospec}(U_i) \arrow[r, "\sim", rightarrow] &
		\mathrm{Cospec}(U)
	.\end{tikzcd}
	\end{equation*} 
	Let's now consider filtered direct limits of 
	finite locally free $S$-schemes $A_i$, 
	and denote $U_i \coloneqq A_i^\vee$ as above.
	Then $\varinjlim_i U_i$ is a limit 
	of finite locally-free $\mathcal{O}_{ S }$ co-algebras.
	In particular, given a {\em Barsotti-Tate group} or a {\em formal Lie variety}
	over $S$, one can write it as $\mathrm{Cospec}(U)$ for an appropriate 
	$\mathcal{O}_{ S }$ co-algebra $U$.
\end{rem}


\begin{defn}[fppf sheaf associated to an $\mathcal{O}_{ S }$-module]
	Let $S$ be a scheme and $M$ a quasi-coherent sheaf of $\mathcal{O}_{ S }$-modules.
	We define $\underline{M}$, the {\em sheaf of $\underline{\mathcal{O}_{ S }}$-modules},
	as the fppf sheaf whose sections on $f\colon T \to S$ fppf are given by
	$\Gamma(T, f^*M)$.
\end{defn}


\begin{rem}[]
	Let's notice that in the above definition, by $\underline{\mathcal{O}_{ S }}$
	we mean the $\underline{\mathcal{O}_{ S }}$-module associated to the 
	quasi-coherent sheaf $\mathcal{O}_{ S }$ itself.
	Moreover, for a morphism of schemes $f\colon T \to S$, by $f^* M$ we mean 
	the $\mathcal{O}_{ T }$-module $f^{-1} M \otimes_{f^{-1}\mathcal{O}_{ S }} \mathcal{O}_{ T }$,
	where $f^{-1} M$ is the usual inverse image of sheaves and
	$f^{-1}\mathcal{O}_{ S } \to \mathcal{O}_{ T }$ is defined starting from $f^\#$.
\end{rem}


\begin{defn}[]
	Let $U$ be an augmented co-algebra on a scheme $S$.
\begin{enumerate}
\item We say that a section $x$ of $U$ is {\em primitive} iff
	$\Delta(x) = x \otimes 1 + 1 \otimes x$;

\item we denote by $\underline{\mathrm{Lie}}(U)$ the sheaf of $\underline{\mathcal{O}_{ S }}$-modules
	whose sections over $S'$, for $S' \to S$ fppf, are given by the primitive
	elements of $\Gamma(S', U_{S'})$ and operations are induced by those on $\underline{U}$.
\end{enumerate}
\end{defn}


\begin{rem}[{\cite[Chapter III, section 2, example 2.2.2]{Messing}}]
	Let $U$ be a finite and locally free $\mathcal{O}_{ S }$-module,
	where $S$ is a scheme.
	Then we have the following isomorphism $(U^\vee)^\vee \simeq U$,
	which implies $X = \mathrm{Cospec}(U) \simeq 
	\underline{\mathrm{Spec}}_S(U^\vee)$.
	Let's, moreover, define $\omega_X$ with respect to the section $e\colon S \to X$
	associated to the counit $\eta\colon U \to \mathcal{O}_{ S }$ in
	\begin{equation*}
		\mathrm{Hom}_{ \mathcal{O}_{ S' }}
		\left( A_{S'}, \mathcal{O}_{ S' } \right) 
		\simeq
		\Gamma(S', \underline{\mathrm{Spec}}_S(A))
	.\end{equation*}
	It can be viewed as the dual of the tangent space at the origin
	of our group $X$.
	Analogously the requirements that sections of $\underline{\mathrm{Lie}}(U)$
	be primitive, can be seen as a formal version of Leibnitz rule,
	so that these sections can be paired with left invariant derivations, i.e.
	elements of the tangent space at the origin.
	All in all, the above can be expanded to obtain an isomorphism
	\begin{equation*}
		\underline{\mathrm{Lie}}(X) \coloneqq
		\underline{\mathrm{Lie}}(U) \simeq
		\underline{\mathcal{H}\mathrm{om}}_{ \mathcal{O}_{ S } } 
		\left(\omega_X , \mathcal{O}_{ S } \right)
	.\end{equation*}
	tk: checkit
\end{rem}
\end{document}

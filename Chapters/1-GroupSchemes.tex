\documentclass[../Main]{subfiles}
\begin{document}
\chapter{Preliminaries}
\section{Affine group schemes}
In the following we will use these notations for important categories:
$\mathsf{Sch}_{S}$ for the category of schemes over $S$,
$\mathsf{Gp}$ for that of groups,
$\mathsf{Ab}$ for that of abelian groups
and $\mathsf{Sets}$ for that of sets.
Moreover, by {\em ring} or {\em algebra}, we will mean one which is commutative and with unity.
Finally we will often denote an object $X \in \mathrm{Ob} \left(\mathsf{C}\right)$ 
of a category $\mathsf{C}$ simply by $X \in \mathsf{C}$.
\begin{defn}[$S$-Group scheme]
	Let $F\colon \mathsf{Sch}_S^{op} \to \mathsf{Gp}$ be a functor.
	Assume that $F$ is representable by $G \in \mathsf{Sch}_{ S }$, i.e. such that, 
	functorially in $T \in \mathsf{Sch}_{ S }$,
	\begin{equation}
		\iota F(T) \simeq \mathrm{Hom}_{\mathsf{Sch}_S} \left( T, G \right)
	,\end{equation} 
	for $\iota\colon \mathsf{Gp} \to \mathsf{Sets}$ the forgetful functor.
	Then we call $G$ a {\em group scheme} over $S$
	or $S$-group scheme.
\end{defn}

\begin{rem}[$T$-points of an $S$-scheme]
	Let's recall the standard notation: let $T \in \mathsf{Sch}_{ S }$, one defines
	the $T$-points of $G \in \mathsf{Sch}_{ S }$ as
	\begin{equation}
		G(T) \coloneqq \mathrm{Hom}_{\mathsf{Sch}_{ S }} \left( T, G \right)
	.\end{equation} 
	If, in particular, $G$ is a {\em group scheme} then, by definition,
	$G(T)$ is endowed with group structure for every $T \in \mathsf{Sch}_{ S }$.
\end{rem}

\begin{defn}[Commutative $S$-group scheme]
	We say that a group scheme $G \in \mathsf{Sch}_{ S }$ is {\em commutative}
	iff $G(T)$ is an abelian group for all $T \in \mathsf{Sch}_{ S }$.
\end{defn}

\begin{rem}[]
	By definition, a group is a set $G \in \mathsf{Sets}$, endowed with an operation,
	an inverse map, and an identity element satisfying the usual properties.
	These can be rewritten in terms of commutative diagrams.
	At first one writes the above three (respectively) as the following maps:
	\begin{equation}
			\begin{tikzcd}[row sep = 0ex
         ,/tikz/column 1/.append style={anchor=base east}
         ,/tikz/column 2/.append style={anchor=base west}
         ,/tikz/column 3/.append style={anchor=base east}]
			m\colon G \cross G \arrow[r, "", rightarrow] &
			G & \text{(multiplication)} \\
			\mathrm{inv}\colon G \arrow[r, "", rightarrow] &
			G & \text{(inverse)} \\
			\varepsilon \colon S \arrow[r, "", rightarrow] &
			G & \text{(unit)} 
		\end{tikzcd}
	\end{equation} 
	for $\left\{ e \right\}$ the terminal object in $\mathsf{Gp}$.
	Let's write $\pi\colon G \to \left\{ e \right\}$
	as the unique arrow in the terminal object
	and $\Delta\colon G \to G \cross G$ the diagonal morphism.
	Then, associativity of product, inverse element and
	identity are characterized by: 
	\begin{align}\label{eq:GroupAxioms}
		m \circ (id_G \cross m) &= m \circ (m \cross id_G),\\
		m \circ (id_G \cross \mathrm{inv}) \circ \Delta &=
		m \circ (\mathrm{inv} \cross id_G) \circ \Delta = \varepsilon \circ \pi,\\
		m \circ (\varepsilon \cross id_G) &=
		m \circ (id_G \cross \varepsilon) = id_G
	.\end{align} 
\end{rem}

\begin{rem}[]\label{rem:StructuralMorphisms}
	Given a group scheme $G \in \mathsf{Sch}_{ S }$, Yoneda's lemma allows to translate
	the group structure of $G(T)$, for $T \in \mathsf{Sch}_{ S }$, into a group structure on $G$.
	In particular one obtains that there exist unique maps
	\begin{equation}
			\begin{tikzcd}[row sep = 0ex
         ,/tikz/column 1/.append style={anchor=base east}
         ,/tikz/column 2/.append style={anchor=base west}
         ,/tikz/column 3/.append style={anchor=base east}]
			m\colon G \cross_{ S } G \arrow[r, "", rightarrow] &
			G & \text{(multiplication)} \\
			\mathrm{inv}\colon G \arrow[r, "", rightarrow] &
			G & \text{(inverse)} \\
			\varepsilon \colon S \arrow[r, "", rightarrow] &
			G & \text{(unit)} 
		\end{tikzcd}
	\end{equation} 
	inducing the group structure on $G(T)$ via Yoneda embedding.
	Then, again by Yoneda's lemma, also the above maps have to
	satisfy the properties written in 
	\eqref{eq:GroupAxioms}.
	More explicitly:
	\begin{enumerate}
		\item Associativity of the product
			\begin{equation}
			\begin{tikzcd}[column sep=4.5em]
				G \cross_{ S } G \cross_{ S } G 
				\arrow[r, "id_G \cross_S m", rightarrow] 
				\arrow[d, "m \cross_S id_G"', rightarrow] &
				G \cross_{ S } G \arrow[d, "m", rightarrow] \\
				G \cross_{ S } G \arrow[r, "m"', rightarrow] &
				G.
			\end{tikzcd}
			\end{equation} 
		\item Inverse morphism 
			\begin{equation*}
			\begin{tikzcd}[column sep=2.7em]
				G \cross_{ S } G \arrow[rr, "id_G \cross_S \mathrm{inv}", rightarrow] & &
				G \cross_{ S } G \arrow[d, "m", rightarrow] \\
				G \arrow[u, "\Delta", rightarrow] 
				\arrow[r, "\pi", rightarrow] &
				S \arrow[r, "\varepsilon", rightarrow] & 
				G
			\end{tikzcd}
			\qquad
			\begin{tikzcd}[column sep=2.7em]
				G \cross_{ S } G \arrow[rr, "\mathrm{inv} \cross_S id_G", rightarrow] & &
				G \cross_{ S } G \arrow[d, "m", rightarrow] \\
				G \arrow[u, "\Delta", rightarrow] 
				\arrow[r, "\pi", rightarrow] &
				S \arrow[r, "\varepsilon", rightarrow] & 
				G
			\end{tikzcd}
			\end{equation*} 
		\item Identity element 
			\begin{equation}
			\begin{tikzcd}
				G \cross_{ S } G \arrow[r, "m", rightarrow] &
				G \\
				S \cross_{ S } G \arrow[r, "id_G", rightarrow] 
				\arrow[u, "\varepsilon \cross_S id_G", rightarrow] &
				G \arrow[u, "id_G"', rightarrow] 
			\end{tikzcd}
			\qquad
			\begin{tikzcd}
				G \cross_{ S } G \arrow[r, "m", rightarrow] &
				G \\
				G \cross_{ S } S \arrow[r, "id_G", rightarrow] 
				\arrow[u, "id_G \cross_S \varepsilon", rightarrow] &
				G. \arrow[u, "id_G"', rightarrow] 
			\end{tikzcd}
			\end{equation} 
	\end{enumerate}
\end{rem}

\begin{rem}[]
	One can generalize the definition of group object, from the category
	of $S$-schemes, to any category $\mathsf{C}$ admitting finite products
	(hence with fianl object given by the empty product) in the same manner as above.
	In fact, being $S$ the final object in $\mathsf{Sch}_{ S }$, one sees
	that fiber products over $S$ (seen in the category of schemes)
	are just products in $\mathsf{Sch}_{ S }$.
\end{rem}

\begin{defn}[Morphism of group schemes]
	Let $G, G'$ be group schemes, a {\em homomorphism of group schemes}
	\begin{equation}
		\begin{tikzcd}
			\alpha\colon G \arrow[r, "", rightarrow] &
			G'
		\end{tikzcd}
	\end{equation} 
	is a morphism $G \to G'$ in $\mathsf{Sch}_{ S }$ s.t., for all
	$T \in \mathsf{Sch}_{ S }$, the corresponding morphism at the level of $T$-points
	is a group homomorphism
	\begin{equation}
	\begin{tikzcd}[row sep = 0ex
		,/tikz/column 1/.append style={anchor=base east}
		,/tikz/column 2/.append style={anchor=base west}]
		\alpha(T)\colon G(T) \arrow[r, "", rightarrow] &
		G'(T) \\
		g \arrow[r, "", mapsto] & \alpha \circ g.
	\end{tikzcd}
	\end{equation} 
\end{defn}

\begin{rem}[Morphism of group schemes]
	Let $G, G'$ be group schemes, representing the functors $F,F'$.
	Then, by Yoneda's lemma, giving a homomorphism $\alpha\colon G \to G'$
	is equivalent to giving a morphism between the functors they represent.

	Again by Yoneda's lemma, one sees that a morphism $\alpha\colon G \to G'$
	in $\mathsf{Sch}_{ S }$ is a morphism of group schemes iff
	it preserves products, i.e. iff
	\begin{equation}
		\alpha \circ m = m' \circ (\alpha, \alpha)
	,\end{equation} 
	for $m, m'$ the product morphisms of $G$ and $G'$ respectively.
\end{rem}

\begin{defn}[Category of group schemes]
	Combining all of the definitions so far, one can define the category $\mathsf{Gp}_S$
	of {\em $S$-group schemes}, or more simply {\em $S$-groups},
	as the category whose objects are $S$-group schemes
	and morphisms are homomorphisms of $S$-group schemes.
\end{defn}

\begin{rem}[Kernels and cokernels]
	As with any category, one can define kernels and cokernels 
	in $\mathsf{Gp}_S$ via the usual universal properties.
	With reagrds to kernels, one can easily check that, given a morphism
	$\alpha\colon G \to G'$ of $S$-groups, then 
	\begin{equation}
	\begin{tikzcd}
		G \cross_{ G' } S \arrow[r, "i", rightarrow] &
		G,
	\end{tikzcd}
	\end{equation} 
	where $i$ is the projection on the first factor, is a kernel for $\alpha$.
	Hence kernels exist in $\mathsf{Gp}_S$.
	
	When it comes to cokernels, instead, one finds difficulties, much like
	with sheaves of abelian groups.
	In fact, given a morphism $\alpha\colon G \to G'$, one cannot always find an
	element $H \in \mathsf{Gp}_S$ representing the functor
	\begin{equation}
	\begin{tikzcd}
		T \arrow[r, "", mapsto] &
		\coker (\alpha_T) =
		G(T)/G'(T).
	\end{tikzcd}
	\end{equation}
\end{rem}

%tk: decide what to do of this remark
%In the following we'll tend to focus on {\em separated} groups schemes over $S$.
%Notice that in that case, since $\varepsilon$ is a closed immersion, it is also an affine morphism,
%hence as soon as $G$ is affine, also $S$ is.
\begin{rem}[]
	Then, if we assume that $G = \mathrm{Spec}(A)$ is affine, we can translate the above $S$-scheme
	morphisms, for $S = \mathrm{Spec}(R)$ affine, into $R$-algebra morphisms.
	Then the properties defined by diagrams (tk: references) will translate into properties
	for these new morphisms.
	Recall that the structural morphism $\pi\colon G \to S$ corresponds to a
	morphism $R \to A$ making $A$ into an $R$-algebra.
	Moreover the diagonal morphism $\Delta\colon G \to G \cross_{ S } G$ corresponds
	to the multiplication morphism of $R$-algebras:
	\begin{equation}
	\begin{tikzcd}[row sep = 0ex
		,/tikz/column 1/.append style={anchor=base east}
		,/tikz/column 2/.append style={anchor=base west}]
		\widetilde{\Delta}\colon A \otimes_R A \arrow[r, "", rightarrow] &
		A \\
		a \otimes_R b \arrow[r, "", mapsto] & a \cdot b
	.\end{tikzcd}
	\end{equation} 
	Then, thanks to the arrow-reversing equivalence of categories between
	affine $R$-schemes and commutative $R$-algebras, one translates the structural morphisms
	of schemes defined in \ref{rem:StructuralMorphisms} into the following $R$-algebras morphisms:
	\begin{equation}
			\begin{tikzcd}[row sep = 0ex
         ,/tikz/column 1/.append style={anchor=base east}
         ,/tikz/column 2/.append style={anchor=base west}
         ,/tikz/column 3/.append style={anchor=base east}]
			\widetilde{m}\colon A \arrow[r, "", rightarrow] &
			A \otimes_R A & \text{(comultiplication)} \\
			\widetilde{\mathrm{inv}}\colon A \arrow[r, "", rightarrow] &
			A & \text{(antipode)} \\
			\widetilde{\varepsilon} \colon A \arrow[r, "", rightarrow] &
			R & \text{(counit/augmentation)} 
		\end{tikzcd}
	\end{equation} 
	Using the contravariant equialence of categories, then,
	one can translate the properties satisfied by $m, \varepsilon$ and $\mathrm{inv}$
	into analogous for $\widetilde{m}, \widetilde{\varepsilon}$ and $\widetilde{\mathrm{inv}}$.
\end{rem}

\begin{defn}[Hopf algebra]
	An {\em Hopf algebra} is an $R$-algebra $A$,
	endowed with a comultiplication, a couint and an antipode map,
	respectively:
	\begin{equation}
			\begin{tikzcd}[row sep = 0ex
         ,/tikz/column 1/.append style={anchor=base east}
         ,/tikz/column 2/.append style={anchor=base west}
         ,/tikz/column 3/.append style={anchor=base east}]
			\widetilde{m}\colon A \arrow[r, "", rightarrow] &
			A \otimes_R A \\
			\widetilde{\mathrm{inv}}\colon A \arrow[r, "", rightarrow] &
			A \\
			\widetilde{\varepsilon} \colon A \arrow[r, "", rightarrow] &
			R,
		\end{tikzcd}
	\end{equation} 
	satisfying the conditions obtained by dualizing those
	of remark \ref{rem:StructuralMorphisms}, more explicitly:
	\begin{align}
		( \widetilde{m} \otimes_R id_A ) \circ \widetilde{m} &=
		( id_A \otimes_R \widetilde{m} ) \circ \widetilde{m},\\
		( id_A \otimes_R \widetilde{\varepsilon} ) \circ \widetilde{m} &=
		( \widetilde{\varepsilon} \otimes_R id_A ) \circ \widetilde{m} =
		id_A,\\
		\widetilde{\Delta} \circ ( id_A \otimes_R \widetilde{\mathrm{inv}} ) 
		\circ \widetilde{m} &=
		\widetilde{\Delta} \circ ( \widetilde{\mathrm{inv}} \otimes_R id_A ) 
		\circ \widetilde{m} =
		(R \to A) \circ \widetilde{\varepsilon}
	.\end{align} 
\end{defn}

\begin{rem}[]
	tk: add a remark/proposition on how to translate the group defining morphisms
	into Hopf algebra morphism and viceversa.
\end{rem}


\begin{ex}[Some interesting examples]
	In all of the following examples we assume, for simplicity, that $S = \mathrm{Spec}(R)$
	and $G = \mathrm{Spec}(A)$ are affine schemes.
	\begin{enumerate}
		\item The {\em additive group scheme}, denoted by $\mathbb{G}_a$, is defined by
			\begin{align}
				\mathbb{G}_a\colon \mathsf{Sch}_{ S } &\longrightarrow \mathsf{Gp} \\
				X &\longmapsto \Gamma \left( X , \mathcal{O}_{ X } \right) \nonumber
			,\end{align} 
			in which $\Gamma \left( X , \mathcal{O}_{ X } \right)$ is viewed
			as an additive group.
			Then $\mathbb{G}_a$ is represented by 
			\begin{equation}
				G = \mathrm{Spec}(R[x])
			.\end{equation} 
			In fact, from the characterization of morphisms into affine schemes,
			we obtain that
			\begin{equation}
			\mathrm{Hom}_{\mathsf{Sch}_S} \left( X, G \right) \simeq
			\mathrm{Hom}_{R \text{-}\mathsf{Alg}} 
			\left( R[x], \Gamma \left( X , \mathcal{O}_{ X } \right) \right) \simeq
			\Gamma \left( X , \mathcal{O}_{ X } \right)
			,\end{equation} 
			functorially in $X$.
			But then an $R$-algebra morphism 
			$R[x] \to \Gamma \left( X , \mathcal{O}_{ X } \right)$
			is determined by the image of $x$, hence
			the last isomorphism above.

			Now we can work to translate the morphisms $m, \varepsilon$ and $\mathrm{inv}$
			into Hopf algebra morphisms.
			Let's start by $m$:
			given $f,g \in \mathrm{Hom}_{\mathsf{Sch}_S} \left( X, G \right)$,
			they correspond to 
			\begin{equation}
			\varphi, \psi \in
			\mathrm{Hom}_{R \text{-}\mathsf{Alg}} 
			\left( R[x], \Gamma \left( X , \mathcal{O}_{ X } \right) \right)
			.\end{equation} 
			In particular each of the two can be identified to the element 
			$s \in \Gamma \left( X , \mathcal{O}_{ X } \right)$ to which
			it maps $x \in R[x]$.
			Then we define $f + g$ to be the morphism of schemes corresponding
			to $\varphi + \psi$, i.e. the morphism given by 
			\begin{equation}
				\left( \varphi + \psi \right)(x) \coloneqq
				\varphi (x) + \psi(x)
			.\end{equation} 
			Now, by definition $f + g \coloneqq m \circ (f,g)$.
			Then we want to define $\widetilde{m}$ in such a way that
			\begin{equation}
			(\varphi, \psi) \circ \widetilde{m} = \varphi + \psi
			.\end{equation} 
			In particular we need only specify $\widetilde{m}(x)$ to identify our morphism,
			moreover as soon as we identify a morphism satisfying the above condition,
			$\widetilde{m}$ has to coincide with it by the bijectivity of the correspondance
			between morphism of $R$-algebras and $S$-schemes.
			We can easily check that $\widetilde{m}(x) \coloneqq x \otimes_R 1 + 1 \otimes_R x$
			does the job, in fact:
			\begin{equation}
				(\varphi, \psi) \circ \widetilde{m} (x) =
				\varphi(x) \cdot 1 + 1 \cdot \psi(x)
			.\end{equation} 
			If we concentrate on $\mathrm{inv}$ now, we see that
			$\mathrm{inv}\, f$ is the morphism of schemes associated to
			$-\varphi$, where $\varphi$ is the morphism of $R$-algebras associated
			to $f$.
			Then, from the above we obtain that $\widetilde{\mathrm{inv}}$
			must satisfy $-\varphi = \varphi \circ \widetilde{\mathrm{inv}}$.
			And, by linearity, we see that $\widetilde{\mathrm{inv}}(x) \coloneqq -x$ does the job.

			Analogously for the zero element, we know that $m \circ (f, \varepsilon) = f$
			for all $f$.
			Then, $\widetilde{m}$ has to satisfy $(\varphi, \widetilde{\varepsilon}) \circ
			\widetilde{m} = \varphi$ for all $\varphi$.
			But then, on $x$, it acts by
			\begin{equation}
				(\varphi, \widetilde{\varepsilon}) \circ \widetilde{m} (x) =
				\varphi(x) + \widetilde{\varepsilon}(x)
			.\end{equation} 
			And this can happen for all $\varphi$ iff $\widetilde{\varepsilon}(x) = 0$.

			Then we have obtained the explicit description of the {\em Hopf algebra}
			morphisms:
			\begin{align}
				\widetilde{m}(x) &= x \otimes_R 1 + 1 \otimes_R x,\\
				\widetilde{\varepsilon}(x) &= 0, \nonumber\\
				\widetilde{\mathrm{inv}}(x) &= -x \nonumber
			.\end{align} 

		\item The {\em general linear group scheme}, denoted by $\mathbb{GL}_n$,
			is defined by
			\begin{align}
				\mathbb{GL}_n\colon \mathsf{Sch}_{ S } &\longrightarrow \mathsf{Gp} \\
				X &\longmapsto \mathbb{GL}(n)(X) \nonumber
			,\end{align} 
			where $\mathbb{GL}(n)(X)$ is the set of invertible $n \cross n$
			matrices with coefficients in $\Gamma \left( X , \mathcal{O}_{ X } \right)$,
			viewed as a multiplicative group.
			Let's recall that, given $\left( x_{i,j} \right)_{i,j = 1}^n, 
			\left( y_{i,j} \right)_{i,j = 1}^n \in \mathbb{GL}(n)(X)$,
			their product is given by $\left( c_{i,j} \right)_{i,j = 1}^n$,
			where
			\begin{equation}
			c_{i,j} = \sum_{l=1}^{n} x_{i,l} y_{l,j}
			.\end{equation} 
			Then $\mathbb{GL}_n$ is represented by the $S$-scheme
			$\mathrm{Spec}(A)$, where
			\begin{equation}
				A = \frac{R[x_{i,j},y]_{i,j = 1}^n}{(y \cdot D(x_{1,1}, \ldots, x_{n,n})-1)}
			\end{equation} 
			and $D(x_{1,1}, \ldots, x_{n,n})$ is the polynomial for the determinant
			of a $n \cross n$ matrix.
			Reasoning as before, any $f \in \mathrm{Hom}_{\mathsf{Sch}_S}
			\left( X, G \right)$ is just the data of an invertible $n \cross n$ matrix
			with coefficients in $\Gamma \left( X , \mathcal{O}_{ X } \right)$
			(invertibility is granted by adding the indeterminate $y$).
			Then, considered two such functions $f,g$, with associated functions
			$\varphi, \psi$ acting on hte generators of the $R$-algebra
			respectively as
			\begin{align}
				x_{i,j} &\longmapsto a_{i,j}\\
				x_{i,j} &\longmapsto b_{i,j}\nonumber
			.\end{align} 
			Their product is the function $\varphi \cdot \psi$
			corresponding to the product of the matrices, i.e. acting on generators as
			\begin{equation}
				x_{i,j} \longmapsto c_{i,j} \coloneqq 
				\sum_{l=1}^{n} a_{i,l} b_{l,j}
			.\end{equation} 
			As before we want $\varphi \cdot \psi = (\varphi, \psi) \circ \widetilde{m}$.
			And it is clear that the map
			\begin{equation}
				\widetilde{m}\colon x_{i,j} \longmapsto 
				\sum_{l=1}^{n} x_{i,l} \otimes_R x_{l,j}
			\end{equation} 
			satisfies our request.
	\end{enumerate}
\end{ex}

tk: you have to add the part on duality, on exact sequence, on etale groups and separated algebras,
on the canonical short exact sequence with the connected component of the identity.
\end{document}

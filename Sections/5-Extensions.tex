\documentclass[../Main]{subfiles}
\begin{document}
\section{The crystals associated to Barsotti-Tate groups}
Let $S$ be a scheme and $G$ a finite locally-free $S$-group.

\begin{prop}[{\cite[Chapter IV, proposition 1.3]{Messing}}]
	The functor acting by
	\begin{equation*}
	\begin{tikzcd}[row sep = 0ex
		,/tikz/column 1/.append style={anchor=base east}
		,/tikz/column 2/.append style={anchor=base west}]
		M \arrow[r, "", mapsto] & \mathrm{Hom}_{ \mathsf{Gr}_S } \left( G, \underline{M} \right)
	\end{tikzcd}
	\end{equation*} 
	is represented by $\underline{\omega}_{G^\vee}$, where $G^\vee$ is
	the Cartier dual of $G$.
\end{prop}


\begin{rem}[]
	It can be checked that the isomorphism
	\begin{equation*}
	\begin{tikzcd}[row sep = 0ex
		,/tikz/column 1/.append style={anchor=base east}
		,/tikz/column 2/.append style={anchor=base west}]
		\mathrm{Hom}_{\mathsf{Gr}_S} \left( G, \underline{M} \right)
		\arrow[r, "\sim", rightarrow] &
		\mathrm{Hom}_{\mathsf{Gr}_S} \left( \underline{\omega}_{G^\vee}, \underline{M} \right)
	\end{tikzcd}
	\end{equation*} 
	is functorial in $G$.
	Thus we have a commutative diagram whose lower horizontal arrow is
	induced by the Cartier dual of $u$
	\begin{equation*}
	\begin{tikzcd}
		G \arrow[r, "u", rightarrow] 
		\arrow[d, "\alpha_G"', rightarrow] &
		H \arrow[d, "\alpha_H", rightarrow] \\
		\underline{\omega}_{G^\vee} \arrow[r, "", rightarrow] &
		\underline{\omega}_{H^\vee}
	.\end{tikzcd}
	\end{equation*}
\end{rem}


\begin{defn}[Extension]
	Let $\mathsf{C}$ be a category and $A,B \in \mathsf{C}$.
	We define an {\em extension} $X$ of $A$ by $B$ to be a short exact sequence
	\begin{equation*}
	\begin{tikzcd}
		0 \arrow[r, "", rightarrow] &
		B \arrow[r, "", rightarrow] &
		X \arrow[r, "", rightarrow] &
		A \arrow[r, "", rightarrow] &
		0
	,\end{tikzcd}
	\end{equation*}
	where $X \in \mathsf{C}$.
	Moreover, given two extensions of $A$ by $B$, given by $X$ and $X'$,
	we say that a {\em morphism of extensions} of $A$ by $B$
	is a morphism $f\colon X \to X'$ that makes the following diagram commute
	\begin{equation*}
	\begin{tikzcd}
		0 \arrow[r, "", rightarrow] &
		B \arrow[r, "", rightarrow] 
		\arrow[d, "", equals] &
		X \arrow[r, "", rightarrow] 
		\arrow[d, "f", rightarrow] &
		A \arrow[r, "", rightarrow] 
		\arrow[d, "", equals] &
		0\\
		0 \arrow[r, "", rightarrow] &
		B \arrow[r, "", rightarrow] &
		X' \arrow[r, "", rightarrow] &
		A \arrow[r, "", rightarrow] &
		0
	.\end{tikzcd}
	\end{equation*}
\end{defn}


\begin{rem}[]
	Notice that, thanks to the five lemma, in a morphism
	of extensions of $A$ by $B$, $f$ is always an isomorphism.
\end{rem}


\begin{defn}[Pullback and pushout]\leavevmode\vspace{-\baselineskip}
\begin{enumerate}
	\item Given a morphism $A' \to A$ and an extension $X$
		of $A$ by $B$, we define $X' \coloneqq X \cross_A A'$,
		so that we have the following commutative diagram
		of short exact sequences
		\begin{equation*}
		\begin{tikzcd}
			0 \arrow[r, "", rightarrow] &
			B \arrow[r, "", rightarrow] 
			\arrow[d, "", rightarrow] &
			X' \arrow[r, "", rightarrow] 
			\arrow[d, "", rightarrow] &
			A' \arrow[r, "", rightarrow] 
			\arrow[d, "", rightarrow] &
			0\\
			0 \arrow[r, "", rightarrow] &
			B \arrow[r, "", rightarrow] &
			X \arrow[r, "", rightarrow] &
			A \arrow[r, "", rightarrow] &
			0
		.\end{tikzcd}
		\end{equation*}
		The extension $X'$ of $A'$ by $B$
		is called the {\em pullback} of $X$ via $A' \to A$.

	\item Given a morphism $B \to B'$ and an extension $X$
		of $A$ by $B$, we define $X' \coloneqq B' \coprod_A X$,
		so that we have the following commutative diagram
		of short exact sequences
		\begin{equation*}
		\begin{tikzcd}
			0 \arrow[r, "", rightarrow] &
			B \arrow[r, "", rightarrow] 
			\arrow[d, "", rightarrow] &
			X \arrow[r, "", rightarrow] 
			\arrow[d, "", rightarrow] &
			A \arrow[r, "", rightarrow] 
			\arrow[d, "", rightarrow] &
			0\\
			0 \arrow[r, "", rightarrow] &
			B' \arrow[r, "", rightarrow] &
			X' \arrow[r, "", rightarrow] &
			A \arrow[r, "", rightarrow] &
			0
		.\end{tikzcd}
		\end{equation*}
		The extension $X'$ of $A$ by $B'$
		is called the {\em pushout} of $X$ via $B \to B'$.
\end{enumerate}
\end{defn}


\noindent
In our case we are mainly interested in the case of extensions of $G$
a {\em Barsotti-Tate} group over $S$, a scheme on which $p^N$ is zero.
In particular we are interested in extensions of $G$ by vector groups,
i.e. by a group $\underline{M}$ for a quasi-coherent $\mathcal{O}_{ S }$-module $M$.


\begin{defn}[Universal extension]
	We say that an extension $E(G)$ of $G$ by the vector group $\underline{V}(G)$ is
	{\em universal} iff, given any extension
	\begin{equation*}
	\begin{tikzcd}
		(\zeta) &
		0 \arrow[r, "", rightarrow] &
		\underline{M} \arrow[r, "", rightarrow] &
		(*) \arrow[r, "", rightarrow] &
		G \arrow[r, "", rightarrow] &
		0
	\end{tikzcd}
	\end{equation*}
	of $G$ by $\underline{M}$, there is a unique map $\varphi\colon \underline{V}(G) \to \underline{M}$
	such that the pushout of $E(G)$ by $\varphi$ is the given
	extension $(\zeta)$.
\end{defn}


\subsection{Crystals associated to Barsotti-Tate groups}


\begin{defn}[Lcally infinitesimally liftable Barsotti-Tate groups]
	Let $S_0$ be a scheme with $p$ locally nilpotent on it.
	We denote by $\mathsf{BT}'(S_0) \subset \mathsf{BT}(S_0)$ 
	the full subcategory whose objects satisfy the following property:
	$G_0$ is in $\mathsf{BT}'(S_0)$ iff
	there is an affine open cover $\left\{ U_i \right\}_{i \in I}$ of $S_0$, which depends on $G_0$, 
	on which, for all $i$ and all finite order thickening $U_i \hookrightarrow U$,
	there is a $G \in \mathsf{BT}(U)$ such that
	$\left.G\right|_{U_i} = \left.G_0\right|_{U_i}$.
	We call the Barsotti-Tate groups in $\mathsf{BT}'(S_0)$
	{\em locally infinitesimally liftable}.
\end{defn}


\begin{thm}[{\cite[Chapter IV, theorem 2.2]{Messing}}]\label{thm:UniqueLiftingCrystals}
	Let $S = \mathrm{Spec}(A)$ such that $p^N \cdot 1_S = 0$,
	$S-0 \coloneqq \mathbb{V}(I) \subset S$, where $I$ is
	an ideal of $A$ with nilpotent divided powers.
	Let $G, H \in \mathsf{BT}(S)$ and consider a homomorphism
	$u_0\colon G_0 \to H_0$ between their restrictions to $S_0$.
	By tk: add reference (thm 1.15 on Messing) $u_0$ defines
	a morphism of extensions $v_0 \coloneqq E(u_0)\colon E(G_0) \to E(H_0)$,
	making the diagram commute:
	\begin{equation*}
	\begin{tikzcd}
		0 \arrow[r, "", rightarrow] &
		\underline{V}(G_0) \arrow[d, "\underline{V}(u_0)", rightarrow] 
		\arrow[r, "", rightarrow] &
		E(G_0) \arrow[d, "v_0", rightarrow]
		\arrow[r, "", rightarrow] &
		G_0 \arrow[d, "u_0", rightarrow]
		\arrow[r, "", rightarrow] &
		0 \\
		0 \arrow[r, "", rightarrow] &
		\underline{V}(H_0) \arrow[r, "", rightarrow] &
		E(H_0) \arrow[r, "", rightarrow] &
		H_0 \arrow[r, "", rightarrow] &
		0
	.\end{tikzcd}
	\end{equation*}
	Then there is a unique morphism of $S$-groups (tk)
	$v\colon E(G) \to E(H)$, which is not necessarily a morphism
	of extensions, that satisfied the following properties:
\begin{enumerate}
	\item $v$ is a lifting of $v_0$;
	\item given $w\colon \underline{V}(G) \to \underline{V}(H)$ a lifting of $\underline{V}(u_0)$,
		denoted the inclusions by $i\colon \underline{V}(H) \to E(H)$ and by 
		$j\colon \underline{V}(G) \to E(G)$,
		such that $d\coloneqq (i \circ w - v \circ j)\colon \underline{V}(G) \to E(H)$
		induces the zero on $S_0$, then $d$ is an exponential.
		(tk: the definition of exponential for BT groups is given in III 2.4,
		think about whether its better to tex it or not.)
\end{enumerate}
\end{thm}


\begin{rem}[]
	tk: copy from rem 2.3. The morphism $v$ is independent of the choice of $w$
	in \cref{thm:UniqueLiftingCrystals}.
	In fact, chosen another lift $w'$ of $\underline{V}(u_0)$, tk.
\end{rem}


\begin{cor}[{\cite[Chapter IV, corollary 2.4.1]{Messing}}]
	Let $G, H, K \in \mathsf{BT}(S)$ as before and consider another
	homomorphism $u_0'\colon H_0 \to K_0$, where $K_0$ again denotes
	the restriction of $K$ to $S_0$.
	Denote by $E_S(u_0)$ the morphism $v$ whose existence is granted in
	\cref{thm:UniqueLiftingCrystals}.
	Then $E_S(u'_0 \circ u_0) = E_S(u'_0) \circ E_S(u_0)$.
\end{cor} 


\begin{cor}[{\cite[Chapter IV, corollary 2.4.2]{Messing}}]
	If, in the above notation, $G = H$ and $u_0 = \mathrm{id}_{ G_0 }$,
	then $E_S(u_0) = \mathrm{id}_{ G }$.
\end{cor} 


\begin{cor}[{\cite[Chapter IV, corollary 2.4.3]{Messing}}]
	Let $G,H, u_0$ as in \cref{thm:UniqueLiftingCrystals},
	with moreover $u_0$ an isomorphism.
	Then $E_S(u_0)$ is an isomorphism too.
\end{cor} 


\begin{cor}[{\cite[Chapter IV, corollary 2.4.4]{Messing}}]
	Suppose we are given a commutative diagram
	\begin{equation*}
	\begin{tikzcd}
		S_0 \arrow[r, "", hookrightarrow] &
		S \\
		S_0' \arrow[u, "", rightarrow] 
		\arrow[r, "", hookrightarrow] &
		S' \arrow[u, "", rightarrow] 
	,\end{tikzcd}
	\end{equation*}
	where $S_0 \hookrightarrow S$ and $S'_0 \hookrightarrow S'$ are
	nilpotent immersion as in the statement of \cref{thm:UniqueLiftingCrystals}.
	Let $S_0' = \mathbb{V}(J)$ and $S_0 = \mathbb{V}(I)$
	and assume that $S' \to S$ is a divided powers morphism.
	Consider $G,H \in \mathsf{BT}(S)$ and $u_0\colon G_0 \to H_0$ as before.
	Then the construction of $E_S(u_0)$ is compatible with the base
	change $S' \to S$.
	More explicitly we have
	\begin{equation*}
		E_{S'}(u_{0_{S'_0}}) =
		\left( E_S(u_0) \right)_{S'} = v_{S'}
	.\end{equation*}
\end{cor} 

tk: maybe in section 0 add what you mean by $\mathbb{V}(I)$ for an affine scheme.
\end{document}

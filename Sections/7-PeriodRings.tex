\section{Period Rings}
%to define:
%\begin{enumerate}
%\item Representations ($p/\ell$-adic reps),
%\item regular algebras
%\item admissible representations
%\item functors going from representations to modules
%\end{enumerate}
%
We introduce, for this section, the following notation:
$K$ will denote a complete discrete valuation ring, with residue field $k$
of characteristic $p$.
We will denote by $W \coloneqq W(k)$ the ring of Witt vectors with coefficients in $k$
and by $K_0 \coloneqq W[1/p]$ its field of fractions.
We will denote by $\sigma$ the absolute Frobenius acting
on $k, W$ and $K_0$.
We fix $e \coloneqq [ K : K_0 ]$ the absolute ramification index of $K$,
we denote by $\overline{K}$ a fixed separable closure of $K$,
and by $\mathbb{C}_K \coloneqq \widehat{\overline{K}}$ its completion.
Finally, given a field $F$, we denote by $O_F$ its ring of integers.
tk: this remark maybe to be said before in the introduction.



\subsection{Period rings}
\begin{defn}[]
	Let $A$ be an $\mathbb{F}_p$-algebra.
	We can associate it the perfect $\mathbb{F}_p$ algebra
	\begin{equation*}
		R(A) \coloneqq \varprojlim_{x \mapsto x^p} A =
		\left\{ \left( x_0, x_1, \ldots \right) \in \prod_{n \in \mathbb{N}} A
		\ \middle|\ x_{n+1}^p = x_n \text{ for all } n \in \mathbb{N} \right\}
	\end{equation*}
	endowed with the product ring structure.
\end{defn}


\begin{rem}[]\leavevmode\vspace{-.2\baselineskip}
\begin{enumerate}
\item The above $\mathbb{F}_p$-algebra is perfect
	since the $p$-th power map is clearly surjective by definition.
	Moreover it is injective since any element $\left( x_n \right)_{n \in \mathbb{N}}$
	satisfying $\left( x_n \right)^p = 0$ has, for all $n \geq 1$,
	$x_{n-1} = x_n^p = 0$.

\item We have a canonical morphism
	\begin{equation*}
	\begin{tikzcd}[row sep = 0ex
		,/tikz/column 1/.append style={anchor=base east}
		,/tikz/column 2/.append style={anchor=base west}]
		R(A) \arrow[r, "", rightarrow] &
		A \\
		\left( x_n \right)_{n \in \mathbb{N}} \arrow[r, "", mapsto] & 
		x_0
	.\end{tikzcd}
	\end{equation*} 
	Moreover it is clear that any morphism from a perfect $\mathbb{F}_p$-algebra
	to $A$ factors through the above projection.

\item If $A$ is already perfect, then it is clear that $R(A) \simeq A$.
	In particular the inverse map of the above projection is given by
	\begin{equation*}
	\begin{tikzcd}[row sep = 0ex
		,/tikz/column 1/.append style={anchor=base east}
		,/tikz/column 2/.append style={anchor=base west}]
		a \arrow[r, "", mapsto] &
		\left( a^{1/p^n} \right)_{n \in \mathbb{N}}
	.\end{tikzcd}
	\end{equation*} 

\item It can be shown that, given $F$ a field of characteristic $p$, then $R(F)$ is the largest
	perfect subfield of $F$.
	Denote now by $\overline{k}$ a fixed separable closure of $k$.
	Then, since $\overline{k}$ is already perfect, $R(\overline{k}) = \overline{k}$.
\end{enumerate}
\end{rem}


\begin{ntt}[]
	We introduce the following ring
	\begin{equation*}
		R \coloneqq R(O_{\mathbb{C}_K}/pO_{\mathbb{C}_K}) =
		R(O_{\overline{K}}/pO_{\overline{K}})
	.\end{equation*}
	It is a perfect $\mathbb{F}_p$-algebra and also canonically an
	algebra over $\overline{k}$, since $O_{\mathbb{C}_K}$ is.
\end{ntt}


\begin{prop}[{\cite[Proposition 4.3.1]{Brinon}}]
	Let $O$ be a $p$-adically separated and complete ring, and $\mathfrak{a} \triangleleft O$
	an ideal of $O$ containing $p$ and such that $\mathfrak{a}^N \subset pO$ for some $N \in \mathbb{N}$
	(i.e. the $\mathfrak{a}$-adic and $p$-adic topologies coincide).
	Then we can define a map
	\begin{equation*}
	\begin{tikzcd}[row sep = 0ex
		,/tikz/column 1/.append style={anchor=base east}
		,/tikz/column 2/.append style={anchor=base west}]
		R(O/\mathfrak{a}) \arrow[r, "", rightarrow] &
		\varprojlim_{x \mapsto x^p} O \\
		\left( x_n \right)_{n \in \mathbb{N}} \arrow[r, "", mapsto] & 
		\left( x^{(n)} \right)_{n \in \mathbb{N}}
	,\end{tikzcd}
	\end{equation*} 
	where we define $x^{(n)} \coloneqq \lim_{m \to \infty} \widehat{x_{n+m}}^{p^m}$,
	in which $\widehat{x_m}$ is any lift of $x_m$ to $O$.
	This map does not depend on the choice of lift, it is bijective
	and its inverse is given by
	\begin{equation*}
	\begin{tikzcd}[row sep = 0ex
		,/tikz/column 1/.append style={anchor=base east}
		,/tikz/column 2/.append style={anchor=base west}]
		\varprojlim_{x \mapsto x^p} O \arrow[r, "", rightarrow] &
		R(O/\mathfrak{a}) \\
		\left( x^{(n)} \right)_{n \in \mathbb{N}} \arrow[r, "", mapsto] & 
		\left( x^{(n)} \mathrm{mod}\, \mathfrak{a} \right)_{n \in \mathbb{N}}
	.\end{tikzcd}
	\end{equation*} 
	Moreover $R(O/pO) \simeq R(O/\mathfrak{a})$ and this common ring
	is a domain as soon as $O$ is a domain.
\end{prop}


\begin{rem}[]
	If we endow $\varprojlim_{x \mapsto x^p} O$ with the ring structure
	given, for any $x \coloneqq \left( x^{(n)} \right)$ and 
	$y \coloneqq \left( y^{(n)} \right)$, 
	by
	\begin{equation*}
		\left( xy \right)^{(n)} \coloneqq x^{(n)} y^{(n)}
		\qquad \text{ and } \qquad
		\left( x + y \right)^{(n)} \coloneqq
		\lim_{m \to \infty} \big( x^{(n+m)} +
		y^{(n+m)}\big)^{p^m}
	,\end{equation*}
	then the above bijection is an isomorphism of rings.
\end{rem}


\begin{ntt}[]
	In view of the above isomorphism we might see an element $x \in R$
	either as an element 
	$\left( x_n \right)_{n \in \mathbb{N}} \in \varprojlim_{x \mapsto x^p} O_{\mathbb{C}_K}/ (p)$
	or as an element
	$\left( x^{(n)} \right)_{n \in \mathbb{N}} \in \varprojlim_{x \mapsto x^p} O_{\mathbb{C}_K}$.
	We will use the above notation accordingly.
\end{ntt}


\begin{lem}[{\cite[Lemma 4.3.3]{Brinon}}]
	Denote by $\left| \ \cdot \ \right|_p$ the absolute value on $\mathbb{C}_K$
	normalized by $\left| p \right|_p = 1/p$.
	The map
	\begin{equation*}
	\begin{tikzcd}[row sep = 0ex
		,/tikz/column 1/.append style={anchor=base east}
		,/tikz/column 2/.append style={anchor=base west}]
		\left| \ \cdot \ \right|_R\colon R \arrow[r, "", rightarrow] &
		p^{\mathbb{Q}} \cup \left\{ 0 \right\} \\
		\left( x^{(n)} \right)_{n \in \mathbb{N}} \arrow[r, "", mapsto] & 
		\left| x^{(0)} \right|_p
	\end{tikzcd}
	\end{equation*} 
	is a $G_K$-equivariant absolute value on $R$ that makes
	$R$ the valuation ring for the unique valuation
	$\nu_R$ on $\mathrm{Frac}\, R$ extending
	$l \log_p \left| \ \cdot \ \right|_R$ on $R$
	and having value group $\mathbb{Q}$.
	Moreover $R$ is $\nu_R$-adically separated and complete
	and the subfield $\overline{k}$ of $R$ maps
	isomorphically onto the residue field of $R$.
\end{lem} 


\begin{ntt}[]
	Let's denote by $\zeta_{p^n}$ the primitive $p^n$-th roots of unity
	in $O_{\mathbb{C}_K}$.
	We denote by $\varepsilon$ the special element of $R$ given by
	\begin{equation*}
		\varepsilon \coloneqq \left( \varepsilon^{(n)} \right)_{n \in \mathbb{N}} =
		\left( 1, \zeta_p, \zeta_{p^2}, \ldots \right)
	.\end{equation*}
	We can see that any two such $\varepsilon$
	are $\mathbb{Z}_{p}^\cross$-powers of each other.
	Moreover we can compute the valuation
	\begin{equation*}
		\nu_R \left( \varepsilon - 1 \right) = \frac{ p }{ p - 1 }
	.\end{equation*}
\end{ntt}



\subsection{Galois representations}
\begin{defn}[$p$-adic representation]
	A {\em $p$-adic representation} of a profinite group $\Gamma$ is a
	representation $\rho\colon \Gamma \to \mathrm{Aut}_{\mathbb{Q}_p}(V)$
	of $\Gamma$ on a finite-dimensional $\mathbb{Q}_p$ vector space
	$V$, where $\rho$ is continuous.
	Here the topology on $\mathrm{Aut}_{\mathbb{Q}_p}(V)$ is
	that of $\mathrm{GL}_n(\mathbb{Q}_p)$, which is well defined,
	independently of the chosen basis.

	A {\em morphism of $p$-adic representations} $V_1, V_2$ of $\Gamma$
	is a $\Gamma$-equivariant linear map $f\colon V_1 \to V_2$, 
	i.e. a linear map that commutes with the action of $\Gamma$.
	More explicitly, denoting $\rho_i(\gamma)$ simply by $\gamma$ for
	an element $\gamma \in \Gamma$, a $\Gamma$-equivariant
	map satisfies, for all $v \in V_1$ and all $\gamma \in \Gamma$,
	$\gamma(f(v)) = f(\gamma(v))$.

	We denote the category of $p$-adic representations of $\Gamma$,
	whose objects and morphism have just been described,
	by $\mathsf{Rep}_{\mathbb{Q}_p}(\Gamma)$.
\end{defn}


\begin{rem}[]
	In general the above definition is used for representations of Galois groups.
	In particular one fixes $K$ a finite extension of $\mathbb{Q}_p$, $\overline{K}$
	a separable closure of $K$ and $G_K \coloneqq \mathrm{Gal}\left( \overline{K} / K \right)$,
	which is profinite.
	Then one studies $p$-adic representations of $\Gamma = G_K$.
\end{rem}


\begin{rem}[]
	tk: maybe one can add the example of Tate module of an elliptic curve here.
	Define a Tate module before though.
	See page 131 of my notes.
\end{rem}


\begin{defn}[$G_K$-regular algebra]
	Let $B$ be a commutative and integral $\mathbb{Q}_p$-algebra
	equipped with an action of $G_K$.
	Denote by $C \coloneqq \mathrm{Frac}(B)$ and by $E \coloneqq E^{G_K}$.
	We say that $B$ is {\em $G_K$-regular} iff
\begin{enumerate}
	\item $B^{G_K} = C^{G_K}$ and
	\item if $b \in B$ generates a vector space $\mathbb{Q}_p \cdot b$
		stable under the action of $G_K$, then $b$ is invertible in $B$.
\end{enumerate}
\end{defn}


\begin{rem}[]
	Notice that for any $G_K$-regular algebra $E$ is a field.
\end{rem}


\begin{defn}[]
	For a $G_K$-regular algebra $B$, one can define a
	functor
	\begin{equation*}
	\begin{tikzcd}[row sep = 0ex
		,/tikz/column 1/.append style={anchor=base east}
		,/tikz/column 2/.append style={anchor=base west}]
		\mathbb{D}_B\colon \mathsf{Rep}_{\mathbb{Q}_p}(G_K) \arrow[r, "", rightarrow] &
		\mathsf{VS}(E) \\
		V \arrow[r, "", mapsto] & 
		\mathbb{D}_B(V) \coloneqq \left( V \otimes_{\mathbb{Q}_p} B \right)^{G_K}
	,\end{tikzcd}
	\end{equation*} 
	where we denoted by $\mathsf{VS}(E)$ the category of $E$-vector spaces.
	Moreover we can define a natural $B$-linear, $G_K$-equivariant map
	\begin{equation*}
	\begin{tikzcd}[row sep = 0ex
		,/tikz/column 1/.append style={anchor=base east}
		,/tikz/column 2/.append style={anchor=base west}]
		\alpha_B(V)\colon 
		\mathbb{D}_B(V) \otimes_E B\arrow[r, "", rightarrow] &
		V \otimes_{\mathbb{Q}_p} B \\
		\left( v \otimes b_1 \right) \otimes b_2 \arrow[r, "", mapsto] & 
		v \otimes b_1b_2
	.\end{tikzcd}
	\end{equation*}
\end{defn}


\begin{prop}[{\cite[Theorem 5.2.1]{Brinon}}]
	Fix $V \in \mathsf{Rep}_{\mathbb{Q}_p}(G_K)$.
	Then the map $\alpha_B(V)$ is always injective and
	$\dim_E \mathbb{D}_B(V) \leq \dim_E V$.
	Moreover there is equality of dimensions iff $\alpha_B(V)$ is an isomorphism.
\end{prop}


\begin{defn}[Admissible representations]
	We say that $V \in \mathsf{Rep}_{\mathbb{Q}_p}(G_K)$ is a
	{\em $B$-admissible representation} iff 
	$\dim_E \mathbb{D})B(V) = \dim_E V$.
	We denote by $\mathsf{Rep}_{\mathbb{Q}_p}^B(G_K) \subset \mathsf{Rep}_{\mathbb{Q}_p}(G_K)$
	the full subcategory of $B$-admissible representations.
\end{defn}


\begin{prop}[{\cite[Theorem 5.2.1]{Brinon}}]
	The functor
	\begin{equation*}
	\begin{tikzcd}[row sep = 0ex
		,/tikz/column 1/.append style={anchor=base east}
		,/tikz/column 2/.append style={anchor=base west}]
		\mathbb{D}_B\colon \mathsf{Rep}_{\mathbb{Q}_p}^B(G_K) \arrow[r, "", rightarrow] &
		\mathsf{VS}_F(E) \\
		V \arrow[r, "", mapsto] & 
		\mathbb{D}_B(V) \coloneqq \left( V \otimes_{\mathbb{Q}_p} B \right)^{G_K}
	,\end{tikzcd}
	\end{equation*} 
	where we denoted by $\mathsf{VS}_F(E)$ the category of finite dimensional $E$-vector spaces,
	is exact and faithfuls.
	Moreover any subrepresentation and quotient of a $B$-admissible
	representation is $B$-admissible.
\end{prop}

\section{Comparison morphisms}
In this section we will finally put all the pieces together to construct
the comparison morphisms we hinted at in the introduction.
To reach this goal we will still need to construct the ring $B_{\mathrm{cris}}$
and give an introduction to the construction of period rings, carried out by Fontaine.

As of notation, we will fix the following.
We will denote, for this section, by $K$ a complete discrete valuation field, 
with perfect residue field $k$ of characteristic $p$ and uniformizer \(\pi\),
by $W \coloneqq W(k)$ the ring of Witt vectors with coefficients in $k$
and by $K_0 \coloneqq W[1/p]$ its field of fractions.
We will fix $e \coloneqq [ K : K_0 ]$ the absolute ramification index of $K$,
we denote by $\overline{K}$ a fixed separable closure of $K$
and by $\mathbb{C}_K$% \coloneqq \widehat{\overline{K}}$ 
its completion.
Finally we will denote by $\mathscr{G}_K \coloneqq \gal\left( \overline{K} / K \right)$ 
the absolute Galois group of $K$ and notice that its action on $\overline{K}$
extends to $\mathbb{C}_K$ by continuity.



\subsection{Galois representations}
\begin{defn}[$p$-adic representation]
	A \emph{$p$-adic representation} of a profinite group $\Gamma$ is a
	representation $\rho\colon \Gamma \to \mathrm{Aut}_{\mathbb{Q}_p}(V)$
	of $\Gamma$ on a finite-dimensional $\mathbb{Q}_p$ vector space
	$V$, where $\rho$ is continuous.
	Here the topology on $\mathrm{Aut}_{\mathbb{Q}_p}(V)$ is
	that of $\mathrm{GL}_n(\mathbb{Q}_p)$, which is well defined,
	independently of the chosen basis.

	A \emph{morphism of $p$-adic representations} $V_1, V_2$ of $\Gamma$
	is a $\Gamma$-equivariant linear map $f\colon V_1 \to V_2$, 
	i.e. a linear map that commutes with the action of $\Gamma$.
	More explicitly, denoting $\rho_i(\gamma)$ simply by $\gamma$ for
	an element $\gamma \in \Gamma$, a $\Gamma$-equivariant
	map satisfies $\gamma(f(v)) = f(\gamma(v))$
	for all $v \in V_1$ and all $\gamma \in \Gamma$.

	We denote the category of $p$-adic representations of $\Gamma$,
	whose objects and morphism have just been described,
	by $\mathsf{Rep}_{\mathbb{Q}_p}(\Gamma)$.
\end{defn}


\begin{rem}[]
	In general the above definition is used for representations of Galois groups.
	In particular $\mathscr{G}_K$ is profinite and one studies
	$p$-adic representations of $\Gamma = \mathscr{G}_K$.
\end{rem}


%\begin{rem}[]
%	tk: maybe one can add the example of Tate module of an elliptic curve here.
%	Define a Tate module before though.
%	See page 131 of my notes.
%\end{rem}


\begin{defn}[$(F,\Gamma)$-regular algebra]
	Let $F$ be a field and $\Gamma$ a group.
	Let $B$ be an integral $F$-algebra
	equipped with an action of $\Gamma$ via automorphisms of $F$-algebras.
	Denote by $C \coloneqq \mathrm{Frac}(B)$ and by $E \coloneqq B^{\Gamma}$.
	Notice that $\Gamma$ acts on $C$ in a natural way.
	We say that $B$ is \emph{$(F,\Gamma)$-regular} iff
\begin{enumerate}
	\item $B^{\Gamma} = C^{\Gamma}$ and
	\item if $b \in B$ generates a vector space $F \cdot b$
		stable under the action of $\Gamma$, then $b \in B^{\times}$.
\end{enumerate}
\end{defn}


\begin{rem}[]
	Notice that, for any $(F,\Gamma)$-regular algebra \(B\), $E/F$ is a field extension.
	Moreover if $B$ is already a field, than it is clearly $(F,\Gamma)$-regular.
	Finally we will always be concerned with $\Gamma = \mathscr{G}_K$ and $F = \mathbb{Q}_p$,
	so we will fix them and assume that $B$ is a $(\mathbb{Q}_p, \mathscr{G}_K)$-regular
	algebra.
	Moreover in the following we will simply write 
	$\Gamma$-regular to mean $(\mathbb{Q}_p, \Gamma)$-regular.
\end{rem}


\begin{defn}[]
	One can define the functor
	\begin{equation*}
	\begin{tikzcd}[row sep = 0ex
		,/tikz/column 1/.append style={anchor=base east}
		,/tikz/column 2/.append style={anchor=base west}]
		\mathbf{D}_B\colon \mathsf{Rep}_{\mathbb{Q}_p}(\mathscr{G}_K) \arrow[r, "", rightarrow] &
		\mathsf{Vect}(E) \\
		V \arrow[r, "", mapsto] & 
		\mathbf{D}_B(V) \coloneqq \left( B \otimes_{\mathbb{Q}_p} V \right)^{\mathscr{G}_K}
	,\end{tikzcd}
	\end{equation*} 
	where we denoted by $\mathsf{Vect}(E)$ the category of $E$-vector spaces.
	Moreover we can define a natural $B$-linear, $\mathscr{G}_K$-equivariant map
	\begin{equation*}
	\begin{tikzcd}[row sep = 0ex
		,/tikz/column 1/.append style={anchor=base east}
		,/tikz/column 2/.append style={anchor=base west}]
		\alpha_B(V)\colon 
		B \otimes_E \mathbf{D}_B(V) \arrow[r, "", rightarrow] &
		B \otimes_{\mathbb{Q}_p} V \\
		b \otimes d
		\arrow[r, "", mapsto] & 
		bd
	.\end{tikzcd}
	\end{equation*}
\end{defn}


\begin{prop}[{\cite[Theorem 5.2.1]{Brinon}}]
	Fix $V \in \mathsf{Rep}_{\mathbb{Q}_p}(\mathscr{G}_K)$.
	Then the map $\alpha_B(V)$ is always injective and
	$\dim_E \mathbf{D}_B(V) \leq \dim_E V$.
	Moreover there is equality of dimensions iff $\alpha_B(V)$ is an isomorphism.
\end{prop}


\begin{defn}[Admissible representations]
	We say that $V \in \mathsf{Rep}_{\mathbb{Q}_p}(\mathscr{G}_K)$ is a
	\emph{$B$-admissible representation} iff 
	$\dim_E \mathbf{D}_B(V) = \dim_E V$.
	We denote by $\mathsf{Rep}_{\mathbb{Q}_p}^B(\mathscr{G}_K) \subset
	\mathsf{Rep}_{\mathbb{Q}_p}(\mathscr{G}_K)$
	the full subcategory of $B$-admissible representations.
\end{defn}


\begin{prop}[{\cite[Theorem 5.2.1]{Brinon}}]
	Let's denote by $\mathsf{Vect}_f(E)$ the category of finite dimensional $E$-vector spaces.
	Then, the functor
	\begin{equation*}
	\begin{tikzcd}[row sep = 0ex
		,/tikz/column 1/.append style={anchor=base east}
		,/tikz/column 2/.append style={anchor=base west}]
		\mathbf{D}_B\colon \mathsf{Rep}_{\mathbb{Q}_p}^B(\mathscr{G}_K) \arrow[r, "", rightarrow] &
		\mathsf{Vect}_f(E) \\
		V \arrow[r, "", mapsto] & 
		\mathbf{D}_B(V) \coloneqq \left( B \otimes_{\mathbb{Q}_p} V \right)^{\mathscr{G}_K}
	,\end{tikzcd}
	\end{equation*} 
	is exact and faithful.
	Moreover any subrepresentation and quotient of a $B$-admissible
	representation is $B$-admissible.
\end{prop}



\subsection{Period rings}
\begin{defn}[]
	Let $A$ be an $\mathbb{F}_p$-algebra.
	We can associate it the perfect $\mathbb{F}_p$-algebra
	\begin{equation*}
		R(A) \coloneqq \varprojlim_{x \mapsto x^p} A =
		\left\{ \mathbf{x} = \left( \mathbf{x}_0, \mathbf{x}_1, \ldots \right) 
			\in \prod_{n \in \mathbb{N}} A
		\ \middle|\ \mathbf{x}_{n+1}^p = \mathbf{x}_n \text{ for all } n \in \mathbb{N} \right\}
	\end{equation*}
	endowed with the product ring structure.
\end{defn}


\begin{rem}[]\leavevmode\vspace{-.2\baselineskip}
\begin{enumerate}
\item The above $\mathbb{F}_p$-algebra is perfect
	since the $p$th power map is clearly surjective by definition.
	Moreover it is injective since any element $\mathbf{x} = \left( \mathbf{x}_n \right)$
	satisfying $\mathbf{x}^p = 0$ has
	$\mathbf{x}_{n-1} = \mathbf{x}_n^p = 0$ for all $n \geq 1$.

\item We have a canonical morphism
	\begin{equation*}
	\begin{tikzcd}[row sep = 0ex
		,/tikz/column 1/.append style={anchor=base east}
		,/tikz/column 2/.append style={anchor=base west}]
		R(A) \arrow[r, "", rightarrow] &
		A \\
		\left( \mathbf{x}_n \right)_{n \in \mathbb{N}} \arrow[r, "", mapsto] & 
		\mathbf{x}_0
	.\end{tikzcd}
	\end{equation*} 
	Moreover any morphism from a perfect $\mathbb{F}_p$-algebra
	to $A$ factors through the above projection.

\item If $A$ is already perfect, then $R(A) \simeq A$.
	In particular the inverse map of the above projection is given by
	$a \mapsto \left( a^{1/p^n} \right)$.
	In particular, let's consider $\overline{k}$ a fixed separable closure of $k$.
	Since $\overline{k}$ is already perfect, we obtain $R(\overline{k}) \simeq \overline{k}$.

\item Given $F$ a field of characteristic $p$, it can be shown that 
	$R(F)$ is the largest perfect subfield of $F$.
\end{enumerate}
\end{rem}


\begin{ntt}[]
	We introduce the following ring
	\begin{equation*}
		R \coloneqq R(O_{\mathbb{C}_K}/pO_{\mathbb{C}_K}) =
		R(O_{\overline{K}}/pO_{\overline{K}})
	.\end{equation*}
	It is a perfect $\mathbb{F}_p$-algebra and also canonically an
	algebra over $\overline{k}$, since $O_{\mathbb{C}_K}$ is.
\end{ntt}


\begin{prop}[{\cite[Proposition 4.3.1]{Brinon}}]
	Let $O$ be a $p$-adically separated and complete ring, and $\mathfrak{a} \triangleleft O$
	an ideal of $O$ containing $p$ and such that $\mathfrak{a}^N \subset pO$ for some $N \in \mathbb{N}$
	(i.e. the $\mathfrak{a}$-adic and $p$-adic topologies coincide).
	Then we have a map
	\begin{equation*}
	\begin{tikzcd}[row sep = 0ex
		,/tikz/column 1/.append style={anchor=base east}
		,/tikz/column 2/.append style={anchor=base west}]
		R(O/\mathfrak{a}) \arrow[r, "", rightarrow] &
		\varprojlim_{x \mapsto x^p} O \\
		\left( \mathbf{x}_n \right)_{n \in \mathbb{N}} \arrow[r, "", mapsto] & 
		\left( \mathbf{x}^{(n)} \right)_{n \in \mathbb{N}}
	,\end{tikzcd}
	\end{equation*} 
	where we define $\mathbf{x}^{(n)} \coloneqq \lim_{m \to \infty} \widehat{\mathbf{x}_{n+m}}^{p^m}$,
	in which $\widehat{\mathbf{x}_m}$ is any lift of $\mathbf{x}_m$ to $O$.
	This map does not depend on the choice of lift, it is bijective
	and its inverse is given by
	\begin{equation*}
	\begin{tikzcd}[row sep = 0ex
		,/tikz/column 1/.append style={anchor=base east}
		,/tikz/column 2/.append style={anchor=base west}]
		\varprojlim_{n \mapsto x^p} O \arrow[r, "", rightarrow] &
		R(O/\mathfrak{a}) \\
		\left( \mathbf{x}^{(n)} \right)_{n \in \mathbb{N}} \arrow[r, "", mapsto] & 
		\left( \mathbf{x}^{(n)} \mathrm{mod}\, \mathfrak{a} \right)_{n \in \mathbb{N}}
	.\end{tikzcd}
	\end{equation*} 
	Moreover $R(O/pO) \simeq R(O/\mathfrak{a})$ and this common ring
	is a domain as soon as $O$ is.
\end{prop}


\begin{rem}[]\label{tiltingOperations}
	If we endow $\varprojlim_{x \mapsto x^p} O$ with the ring structure
	given, for any $\mathbf{x} \coloneqq \left( \mathbf{x}^{(n)} \right)$ and 
	$\mathbf{y} \coloneqq \left( \mathbf{y}^{(n)} \right)$, 
	by
	\begin{equation*}
		\left( \mathbf{xy} \right)^{(n)} \coloneqq \mathbf{x}^{(n)} \mathbf{y}^{(n)}
		\qquad \text{ and } \qquad
		\left( \mathbf{x} + \mathbf{y} \right)^{(n)} \coloneqq
		\lim_{m \to \infty} \big( \mathbf{x}^{(n+m)} +
		\mathbf{y}^{(n+m)}\big)^{p^m}
	,\end{equation*}
	then the above bijection is an isomorphism of rings.
\end{rem}


\begin{ntt}[]\label{not:tiltingElts}
	In view of the above isomorphism we might see an element $\mathbf{x} \in R$
	either as an element 
	$\left( \mathbf{x}_n \right) \in \varprojlim_{x \mapsto x^p} O_{\mathbb{C}_K}/ (p)$
	or as an element
	$\left( \mathbf{x}^{(n)} \right) \in \varprojlim_{x \mapsto x^p} O_{\mathbb{C}_K}$.
	We will use high and low indices accordingly.
\end{ntt}


\begin{rem}[]\label{rem:GKActionR}
	Let's now notice that $\mathscr{G}_K \coloneqq \gal\left( \overline{K} / K \right)$
	acts naturally on $O_{\mathbb{C}_K}$, since it acts via isometries on
	$\overline{K}$.
	Moreover, since it acts via morphisms of rings, which commute with $x \mapsto x^p$,
	its action can be naturally extended to
	$R = \varprojlim_{x \mapsto x^p} O_{\mathbb{C}_K}$.
\end{rem}


\begin{lem}[{\cite[Lemma 4.3.3]{Brinon}}]
	Denote by $\left| \ \cdot \ \right|_p$ the absolute value on $\mathbb{C}_K$
	normalized by $\left| p \right|_p = 1/p$.
	The map
	\begin{equation*}
	\begin{tikzcd}[row sep = 0ex
		,/tikz/column 1/.append style={anchor=base east}
		,/tikz/column 2/.append style={anchor=base west}]
		\left| \ \cdot \ \right|_R\colon R \arrow[r, "", rightarrow] &
		p^{\mathbb{Q}} \cup \left\{ 0 \right\} \\
		\left( \mathbf{x}^{(n)} \right)_{n \in \mathbb{N}} \arrow[r, "", mapsto] & 
		\left| \mathbf{x}^{(0)} \right|_p
	\end{tikzcd}
	\end{equation*} 
	is a \emph{$\mathscr{G}_K$-equivariant absolute value} on $R$ that makes
	$R$ the valuation ring for the unique valuation
	$\nu_R$ on $\mathrm{Frac}\, R$ extending
	$-\log_p \left| \ \cdot \ \right|_R$ on $R$
	and having value group $\mathbb{Q}$.
	Moreover $R$ is $\nu_R$-adically separated and complete
	and the subfield $\overline{k}$ of $R$ maps
	isomorphically onto the residue field of $R$.
\end{lem} 


\begin{ex}[]\leavevmode\vspace{-.2\baselineskip}\label{ExampleEltsTilt}
\begin{enumerate}
\item Fix $\big( p^{1/p^n} \big)_{n \in \mathbb{N}}$ 
	a compatible family of $p^n$th roots of $p$
	in $O_{\mathbb{C}_K}$.
	Denote by $\mathbf{p}$ the element of $R$ given by
	\begin{equation*}
		\mathbf{p} \coloneqq \big( p^{(n)} \big)_{n \in \mathbb{N}} =
		\big( p, p^{1/p}, p^{1/p^2}, \ldots \big) \in R
	.\end{equation*}
	Its valuation is easily computed by
	$\nu_R(\mathbf{p}) = \nu_p(p^{(0)}) = \nu_p(p) = 1$.

\item Fix a compatible family of primitive $p^n$th roots of unity
	$\left( \zeta_{p^n} \right)_{n \in \mathbb{N}}$ in $O_{\mathbb{C}_K}$.
	We denote by $\boldsymbol\varepsilon$ the special element of $R$ given by
	\begin{equation*}
		\boldsymbol\varepsilon \coloneqq \big( \boldsymbol\varepsilon^{(n)} \big)_{n \in \mathbb{N}} =
		\left( 1, \zeta_p, \zeta_{p^2}, \ldots \right) \in R
	.\end{equation*}
	The element $\boldsymbol\varepsilon$ depends on the chosen
	compatible family
	and any two such $\boldsymbol\varepsilon$
	are $\mathbb{Z}_{p}^{\times}$-powers of each other.
	Moreover we have
	$\nu_R \left( \boldsymbol\varepsilon - 1 \right) = p/ (p - 1)$.
	Let's show this for $p > 2$: by definition we have
	$\nu_R(\boldsymbol\varepsilon - 1) = \nu_p \left( (\boldsymbol\varepsilon -1 )^{(0)} \right)$.
	By \cref{tiltingOperations} we have
	\begin{equation*}
		\left( \boldsymbol\varepsilon -1 \right)^{(0)} =
		\lim_{n \to \infty} \left( \zeta_{p^n} + (-1)^{(n)} \right)^{p^n}
	.\end{equation*}
	Let's notice that $\left( -1 \right)^{(n)} = -1$ for all $n$
	and that $\zeta_{p^n} - 1$ is a root of $\Phi_{p^n}(1+X)$,
	where $\Phi_m$ denotes the $m$th cyclotomic polynomial.
	In particular $\Phi_{p^n}(1+X)$ is Eisenstein of degree
	$p^{n-1}(p-1)$.
	Then
	\begin{equation*}
		\nu_R(\boldsymbol\varepsilon - 1) =
		\lim_{n \to \infty} \frac{ p^n }{ p^{n-1}(p-1) } =
		\frac{ p }{ p-1 }
	.\end{equation*}
	Finally we recall that $\mathscr{G}_K$ acts on $\zeta_{p^n}$ via
	the cyclotomic character, which is defined by
	$g(\zeta_{p^n}) = \zeta_{p^n}^{\chi(g)}$ for any $g \in \mathscr{G}_K$.
	As a consequence, since the induced action is component-wise,
	$\mathscr{G}_K$ acts also on $\boldsymbol\varepsilon$ via the cyclotomic character,
	i.e. $g(\boldsymbol\varepsilon) = \boldsymbol\varepsilon^{\chi(g)}$
	for all $g \in \mathscr{G}_K$.
\end{enumerate}
\end{ex}


\begin{thm}[{\cite[Theorem 4.3.5]{Brinon}}]
	The field $\mathrm{Frac}\, R = R[1/\mathbf{p}]$ is algebraically closed.
\end{thm}


\begin{rem}[]
	There is a natural family of ring homomorphisms
	\begin{equation*}
	\begin{tikzcd}[row sep = 0ex
		,/tikz/column 1/.append style={anchor=base east}
		,/tikz/column 2/.append style={anchor=base west}]
		\theta_n\colon R \arrow[r, "", rightarrow] &
		O_{\mathbb{C}_K} \\
		\left( \mathbf{x}_m \right)_{m \in \mathbb{N}} \arrow[r, "", mapsto] & 
		\mathbf{x}_n
	.\end{tikzcd}
	\end{equation*} 
	Let's give $R$ a $\overline{k}$-algebra structure
	via the $k$-embedding
	\begin{equation*}
	\begin{tikzcd}[row sep = 0ex
		,/tikz/column 1/.append style={anchor=base east}
		,/tikz/column 2/.append style={anchor=base west}]
		\overline{k} = R(\overline{k}) \arrow[r, "", rightarrow] &
		R(O_{\overline{K}}/pO_{\overline{K}}) =R\\
		c \arrow[r, "", mapsto] & 
		\left( j(c), j(c^{1/p}), j(c^{1/p^2}), \ldots \right)
	,\end{tikzcd}
	\end{equation*} 
	where $j\colon \overline{k} \to O_{\overline{K}}/ (p)$ is the canonical
	section to the reduction map $O_{\overline{K}}/ (p) \twoheadrightarrow \overline{k}$.
	Then $\theta_0$ is a morphism of $\overline{k}$-algebras.
	We wish to lift it to a ring map $W(R) \to O_{\mathbb{C}_K}$,
	but we cannot use universal property of the Witt vectors construction
	since $O_{\mathbb{C}_K}/ (p)$ is not perfect (in particular the Frobenius
	is not injective).
\end{rem}


\begin{defn}[]
	We define, set theoretically, the map
	\begin{equation*}
	\begin{tikzcd}[row sep = 0ex
		,/tikz/column 1/.append style={anchor=base east}
		,/tikz/column 2/.append style={anchor=base west}]
		\theta\colon W(R) \arrow[r, "", rightarrow] &
		O_{\mathbb{C}_K} \\
		\sum_{n \in \mathbb{N} }^{  } [\mathbf{c}_n] p^n \arrow[r, "", mapsto] & 
		\sum_{n \in \mathbb{N} }^{  } \mathbf{c}_n^{(0)} p^n
	,\end{tikzcd}
	\end{equation*} 
	where \cref{TeichmullerExpansionWitt} allows us to write any element of $W(R)$
	in a unique Teichmüller expansion and $\mathbf{c}_n^{(0)}$ is defined as in \cref{not:tiltingElts}.
\end{defn}


\begin{rem}[]
	In \cref{TeichmullerExpansionWitt} we explicitly computed the Teichmüller
	expansion of $\left( \mathbf{r}_n \right)_{n \in \mathbb{N}}$
	to be $\sum_{n \in \mathbb{N} }^{  } [\mathbf{r}_n^{p^{-n}}] p^n$.
	Moreover, by compatibility of the elements in $\varprojlim_{x \mapsto x^p} O_{\mathbb{C}_K}$
	and multiplicativity of $\mathbf{r} \mapsto \mathbf{r}^{(n)}$,
	we have that $\big( \mathbf{r}^{p^{-n}} \big)^{(0)} = \big( (\mathbf{r}^{p^{-n}})^{(n)} \big)^{p^n} =
	\mathbf{r}^{(n)}$.
	Hence we can compute $\theta$ also via
	\begin{equation*}
	\begin{tikzcd}
		\theta\colon 
		\left( \mathbf{r}_0, \mathbf{r}_1, \ldots \right)
		\arrow[r, "", mapsto] &
		\sum_{n \in \mathbb{N} }^{  } \mathbf{r}_n^{(n)} p^n
	.\end{tikzcd}
	\end{equation*}
\end{rem}


\begin{rem}[]\label{GKActionWR}
	Let's recall that in \cref{rem:GKActionR} we saw that
	the action of $\mathscr{G}_K$ extends naturally from $O_{\mathbb{C}_K}$ to $R$.
	Then, thanks to \cref{UPWittVectors}, this naturally induces
	an action of $\mathscr{G}_K$ on $W(R)$.
	More explicitly the action of $\mathscr{G}_K$ is defined,
	for all \(g \in \mathscr{G}_K\), by
	\begin{equation*}
		g \left( \sum_{n \in \mathbb{N} }^{  } [\mathbf{c}_n] p^n \right) =
		\sum_{n \in \mathbb{N} }^{  } [g(\mathbf{c}_n)] p^n
	.\end{equation*}
	Moreover, recalling the explicit description of the Teichmüller
	expansion given in \cref{TeichmullerExpansionWitt}, we see that this
	action of $\mathscr{G}_K$ on $W(R)$ corresponds with the component-wise action
	(again, since it commutes with Frobenius on $R$).
\end{rem}


\begin{lem}[{\cite[Lemma 4.4.1]{Brinon}}]
	The map $\theta\colon W(R) \to O_{\mathbb{C}_K}$ is a
	$\mathscr{G}_K$-equivariant surjective ring homomorphism.
\end{lem} 


\begin{rem}[]
	Notice that $\theta$ is clearly $\mathscr{G}_K$-equivariant,
	since $\mathscr{G}_K$ acts on $O_{\mathbb{C}_K}$
	via isometries (hence via continuous maps).
	Moreover, inverting $p$, we obtain another $\mathscr{G}_K$-equivariant surjective ring homomorphism
	\begin{equation*}
	\begin{tikzcd}[row sep = 0ex
		,/tikz/column 1/.append style={anchor=base east}
		,/tikz/column 2/.append style={anchor=base west}]
		\theta_{\mathbf{Q}}\colon W(R)[1/p] \arrow[r, "", rightarrow] &
		O_{\mathbb{C}_K}[1/p] = \mathbb{C}_K
	.\end{tikzcd}
	\end{equation*} 
	It is important to notice, though, that the source ring is not a complete valuation ring.
\end{rem}


\begin{prop}[{\cite[Proposition 4.4.3]{Brinon}}]
	Let $\mathbf{p}$ be as in \cref{ExampleEltsTilt} and let
	\begin{equation*}
		\xi \coloneqq \xi_{\mathbf{p}} = [\mathbf{p}] - p \in W(R)
	.\end{equation*}
\begin{enumerate}
	\item The ideal $\ker \theta \triangleleft W(R)$ is principal and it is generated
		by $\xi$.
	\item An element $\mathbf{w} = (\mathbf{w}_0, \mathbf{w}_1, \ldots) \in \ker\theta$ generates
		the ideal if and only if $\mathbf{w}_1 \in R^{\times}$.
\end{enumerate}
\end{prop}


\begin{cor}[{\cite[Corollary 4.4.5]{Brinon}}]
	For all $j \geq 1$ we have $W(R) \cap \left( \ker \theta_{\mathbf{Q}} \right)^j =
	\left( \ker \theta \right)^j$.
	Moreover $\bigcap_{j \geq 1} \left( \ker \theta \right)^j =
	\bigcap_{j \geq 1} \left( \ker \theta_{\mathbf{Q}} \right)^j = 0$.
\end{cor} 


\begin{rem}[]\label{kerQGKStable}
	As a consequence of the above we see that $W(R)[1/p]$
	injects into its $\ker \theta_{\mathbf{Q}}$-completion
	\begin{equation*}
	B^+_{\mathrm{dR}} \coloneqq \varprojlim_{j \geq 1}
	\frac{ W(R)[1/p] }{ \left( \ker \theta_{\mathbf{Q}} \right)^j }
	.\end{equation*}
	Recall that $\mathscr{G}_K$ acts component-wise on the ring of Witt vectors,
	so $\ker \theta_{\mathbf{Q}}$ is stable under the action of $\mathscr{G}_K$
	and the transition maps of the above projective limit are $\mathscr{G}_K$-equivariant.
	As a consequence $B^+_{\mathrm{dR}}$ inherits a natural action of $\mathscr{G}_K$
	which is compatible with that on its subring $W(R)[1/p]$.
	Then $B^+_{\mathrm{dR}}$ projects $\mathscr{G}_K$-equivariantly on its quotients
	by $\left( \ker \theta_{\mathbf{Q}} \right)^j$.
	In particular, for $j = 1$, we obtain a lift of $\theta$ 
	to a $\mathscr{G}_K$-equivariant surjection
	\begin{equation*}
	\begin{tikzcd}[row sep = 0ex
		,/tikz/column 1/.append style={anchor=base east}
		,/tikz/column 2/.append style={anchor=base west}]
		\theta^+_{\mathrm{dR}}\colon B^+_{\mathrm{dR}} \arrow[r, "", rightarrow] &
		\mathbb{C}_K
	.\end{tikzcd}
	\end{equation*} 
	Finally we see that the action of the Frobenius on $W(R)[1/p]$
	does not naturally extend to $B^+_{\mathrm{dR}}$.
	In fact $\ker \theta_{\mathbf{Q}}$ is not stable under its action,
	since $\varphi(\xi) = [\mathbf{p}^p] - p \notin \ker \theta_{\mathbf{Q}}$.
\end{rem}


\noindent
By construction $B_{\mathrm{dR}}^+$ is a discrete valuation ring,
so we want to find a uniformizer which behaves well under the action of $\mathscr{G}_K$
\begin{defn}[]
	Let $\boldsymbol\varepsilon$ be as in \cref{ExampleEltsTilt}.
	We define
	\begin{equation*}
		t \coloneqq \log \left( [\boldsymbol\varepsilon] \right) =
		\log \left( 1 + ([\boldsymbol\varepsilon] - 1)\right) =
		\sum_{n = 1 }^{ \infty } (-1)^{n+1} 
		\frac{ ([\boldsymbol\varepsilon] - 1)^n }{ n }
		\in B_{\mathrm{dR}}^+
	.\end{equation*}
\end{defn}


\newpage
\begin{rem}[{\cite[pp. 60--62]{Brinon}}]\leavevmode\vspace{-0.2\baselineskip}
\begin{enumerate}
	\item The element $[\boldsymbol\varepsilon] - 1 $ lies in
		$\ker \theta$, hence the element $\log ([\boldsymbol\varepsilon])$
		is well defined in $B_{\mathrm{dR}}^+$.

	\item One can show, via careful topological arguments, that,
		since any two different choices of $\boldsymbol\varepsilon$
		are related by $\boldsymbol\varepsilon' = \boldsymbol\varepsilon^a$
		for $a \in \mathbb{Z}_{p}^{\times}$, then
		$t' = a t$.

	\item $\mathscr{G}_K$ acts multiplicatively on $t$ via the cyclotomic character,
		i.e. for all $g \in \mathscr{G}_K$ we have
		\begin{equation*}
		g(t) = \chi(g) t
		.\end{equation*}

	\item The element $t$ is a uniformizer of $B_{\mathrm{dR}}^+$.
\end{enumerate}
\end{rem}


\begin{defn}[Field of $p$-adic periods]
	We define the \emph{field of $p$-adic periods}, also called
	the \emph{de Rham period ring},
	\begin{equation*}
	B_{\mathrm{dR}} \coloneqq \mathrm{Frac}\, B_{\mathrm{dR}}^+ =
	B_{\mathrm{dR}}^+[1/t]
	.\end{equation*}
\end{defn}


\begin{rem}[]
	Notice that, just like $B_{\mathrm{dR}}^+$, the field of $p$-adic periods
	$B_{\mathrm{dR}}$ is endowed with a natural action of $\mathscr{G}_K$.
	Moreover we can notice that, set theoretically, the construction of
	$B_{\mathrm{dR}}$ depends only on $\mathbb{C}_K$ and not on $K$,
	though the choice of $K$ changes, functorially, the Galois group acting on it,
	up to restriction to a closed subgroup.
\end{rem}


\begin{prop}[{\cite[Theorem 4.4.13 and example 5.1.3]{Brinon}}]
	$B_{\mathrm{dR}}$ is $\mathscr{G}_K$-regular
	(being a field) and $B_{\mathrm{dR}}^{\mathscr{G}_K} = K$.
\end{prop}


\begin{defn}[]
	We define $A_{\mathrm{cris}}$ to be the $p$-adic completion of the
	divided power envelope of $W(R)$ with respect to the ideal $\ker \theta$.
	More explicitly
	\begin{equation*}
		A_{\mathrm{cris}} = \varprojlim_{n \in \mathbb{N}} 
		\mathcal{D}_{W(R)}(\ker\theta)/p^n\mathcal{D}_{W(R)}(\ker\theta)
	.\end{equation*}
\end{defn}


\begin{rem}[{\cite[\S9.1]{Brinon}}]
	The ring $A_{\mathrm{cris}}$ is identified with a subring
	of $B_{\mathrm{dR}}^+$, whose elements are given by
	\begin{equation*}
	A_{\mathrm{cris}} = 
	\left\{ \sum_{n \in \mathbb{N} }^{  } \mathbf{a}_n \frac{ \xi^n }{ n! } \ \middle|\ 
	\mathbf{a}_n \in W(R) \text{ s.t. } \lim_{n \to \infty} \mathbf{a}_n = 0
	\text{ for the $p$-adic topology}\right\}
	.\end{equation*}
	In particular this grants that $A_{\mathrm{cris}}$ is a domain.
	Moreover the composite $A_{\mathrm{cris}} \hookrightarrow B_{\mathrm{dR}}^+ 
	\twoheadrightarrow \mathbb{C}_K$ lands in $O_{\mathbb{C}_K}$
	and induces a surjective ring homomorphism $A_{\mathrm{cris}} \twoheadrightarrow O_{\mathbb{C}_K}$.
	Also, by {\cite[Proposition 9.1.3]{Brinon}}, we see that the important
	element $t = \log ([\boldsymbol\varepsilon])$ is in $A_{\mathrm{cris}}$.
\end{rem}


\begin{rem}[]
	As seen in \cref{GKActionWR}, $\mathscr{G}_K$ acts on $W(R)$ sending $pW(R)$ to $pW(R)$
	and, as seen in \cref{kerQGKStable}, $\ker\theta$ to $\ker\theta$.
	Then by universal property of divided powers envelope the action extends to
	$\mathcal{D}_{W(R)}(\ker\theta)$.
	More explicitly any $g \in \mathscr{G}_K$ induces the unique dashed arrow
	\begin{equation}\label{eqn:ExtensionGKActionACris}
	\begin{tikzcd}
		&
		\mathcal{D}_{W(R)}(\ker\theta)
		\arrow[rd, "", dashrightarrow,
		start anchor=south east] & \\
		\left(W(R), \ker\theta\right) \arrow[ru, "", rightarrow,
		end anchor=south west] 
		\arrow[rr, "", rightarrow] & &
		W(R) \subset \mathcal{D}_{W(R)}(\ker\theta) 
	.\end{tikzcd}
	\end{equation}
	This then extends to $A_{\mathrm{cris}}$.
	By {\cite[Proposition 9.1.2]{Brinon}} the action
	of $\mathscr{G}_K$ on $A_{\mathrm{cris}}$ is continuous
	for the $p$-adic topology.
	Also, following {\cite[Lemmas 9.1.7-9.1.8]{Brinon}},
	we can extend the Frobenius morphism $\varphi$ on $W(R)$ to 
	$A_{\mathrm{cris}}$.
	In fact one can check on the generator $\xi = [\mathbf{p}] - p$
	of $\ker \theta$ that $\varphi$ on $W(R)$ sends $\ker\theta + (p)$ to
	$\ker\theta + (p)$ and argue as in \cref{eqn:ExtensionGKActionACris}
	that the Frobenius extends to $\mathcal{D}_{W(R)}(\ker\theta)$
	and then to $A_{\mathrm{cris}}$.
	In particular the above holds since
	\begin{equation*}
		\varphi(\xi) = [\mathbf{p}^p] - p =
		[\mathbf{p}]^p - p =
		\underbrace{[\mathbf{p}]^p - p^p}_{\in \ker\theta} + 
		\underbrace{p^p - p}_{\in (p)}
	.\end{equation*}
	Moreover one can check that $\varphi(t) = pt$.
\end{rem}


\begin{rem}[]\label{AcrisAlgebraStructures}
	We can give to $A_{\mathrm{cris}}$ the structure of $W$-algebra and then also
	of $S$-algebra.
	The construction goes as follows.
	We have the injections
	\begin{equation*}
	\begin{tikzcd}
		W(R) \arrow[r, "", hookrightarrow] &
		\mathcal{D}_{W(R)}(\ker\theta) \arrow[r, "", hookrightarrow] &
		\varprojlim_{n \in \mathbb{N}} \mathcal{D}_{W(R)}(\ker\theta)/
		p^n \mathcal{D}_{W(R)}(\ker\theta) = A_{\mathrm{cris}}
	.\end{tikzcd}
	\end{equation*}
	Since, moreover, $k \hookrightarrow R$ we obtain an injection $W \hookrightarrow W(R)$.
	Composed with the above it gives a canonical $W$-algebra structure to $A_{\mathrm{cris}}$.
	We then use this structure to define the $W$-algebra morphism
	\begin{equation*}
	\begin{tikzcd}[row sep = 0ex
		,/tikz/column 1/.append style={anchor=base east}
		,/tikz/column 2/.append style={anchor=base west}]
		\alpha\colon W[u] \arrow[r, "", rightarrow] &
		A_{\mathrm{cris}} \\
		u \arrow[r, "", mapsto] & \left[ \boldsymbol\pi \right]
	,\end{tikzcd}
	\end{equation*}
	where $\boldsymbol\pi$ is defined, like $\mathbf{p}$, as
	$\boldsymbol\pi \coloneqq \big( \pi, \pi^{1/p}, \pi^{1/p^2}, \ldots \big)
	\in \varprojlim O_{\mathbb{C}_K} = R$ for \(\pi\) a uniformizer of \(K\).
	Notice, moreover, that by definition of Frobenius on both $W[u]$ and
	$A_{\mathrm{cris}}$, the map $\alpha$ is compatible with Frobenius.
	In fact the image of $\alpha$ lies in $W(R) \subset A_{\mathrm{cris}}$, hence we can check it
	here, since Frobenius of $A_{\mathrm{cris}}$ extends that of $W(R)$.
	Then, by definition we have the following square
	\begin{equation*}
	\begin{tikzcd}
		\sum_{n \in \mathbb{N} }^{  } a_n u^n
		\arrow[r, "\alpha", mapsto] \arrow[d, "\varphi"', mapsto] &
		\sum_{n \in \mathbb{N} }^{  } a_n \left[ \boldsymbol\pi \right]^n
		\arrow[d, "\varphi", mapsto] \\
		\sum_{n \in \mathbb{N} }^{  } a_n^p u^{np}
		\arrow[r, "\alpha"', mapsto] &
		\sum_{n \in \mathbb{N} }^{  } a_n^p \left[ \boldsymbol\pi \right]^{np}
	,\end{tikzcd}
	\end{equation*}
	which commutes since the Frobenius on $W(R)$ is a ring homomorphism.
	Now we notice that $\alpha(E(u)) \in \ker\theta$
	by definition of $\theta$.
	In fact $\alpha(E(u)) = E(\left[ \boldsymbol\pi \right])$
	and $\theta([\boldsymbol\pi]) = \pi$ imply that $\theta(\alpha(E(u))) = E(\pi) = 0$.
	This implies that, by universal property of divided powers envelope,
	the morphism $W[u] \to W(R) \subset A_{\mathrm{cris}}$
	induces a morphism $\mathcal{D}_{W[u]}(E(u)) \to W(R)$.
	Now we can see $W(R) \subset \mathcal{D}_{W(R)}(\ker\theta)$
	which, combined with the above comments, induces the diagram
	\begin{equation*}
	\begin{tikzcd}
		\mathcal{D}_{W[u]}(E(u)) \arrow[r, "", rightarrow] 
		\arrow[d, "", hookrightarrow] 
		\arrow[rd, "", rightarrow] &
		\mathcal{D}_{W(R)}(\ker\theta) \arrow[d, "", hookrightarrow] \\
		S \arrow[r, "", dashrightarrow] &
		A_{\mathrm{cris}}
	,\end{tikzcd}
	\end{equation*}
	where the dashed arrow exists by universal property of the completion
	of a ring with respect to the $p$-adic topology.
\end{rem}


\begin{defn}[]
	We denote by $B_{\mathrm{cris}}^+$ the $\mathscr{G}_K$-stable $W(R)[1/p]$ subalgebra 
	\begin{equation*}
	B_{\mathrm{cris}}^+ \coloneqq A_{\mathrm{cris}}[1/p] \subset B_{\mathrm{dR}}^+
	.\end{equation*}
	We define the \emph{crystalline period ring} for $K$
	to be the $\mathscr{G}_K$-stable $W(R)[1/p]$-subalgebra of $B_{\mathrm{dR}}$
	given by $B_{\mathrm{cris}} \coloneqq B_{\mathrm{cris}}^+[1/t]$.
\end{defn}


\begin{rem}[]\leavevmode\vspace{-.2\baselineskip}
\begin{enumerate}
\item With some computations (see \cite[Proposition 9.1.3]{Brinon})
	one can show that $t^{p-1} \in p A_{\mathrm{cris}}$, hence that 
	inverting $t$ makes $p$ a unit.
	Then $B_{\mathrm{cris}} = B_{\mathrm{cris}}^+[1/t] = A_{\mathrm{cris}}[1/t]$.

\item As for $B_{\mathrm{dR}}$, set theoretically the definition of $B_{\mathrm{cris}}^+$ and
	of $B_{\mathrm{cris}}$ only depends on $\mathbb{C}_{K}$.
	Again the choice of $K$ changes, functorially, the Galois group $\mathscr{G}_K$ acting on the
	period rings.
\end{enumerate}
\end{rem}


\begin{prop}[{\cite[Proposition 9.1.6]{Brinon}}]
	$B_{\mathrm{cris}}$ is $\mathscr{G}_K$-regular
	and $B_{\mathrm{cris}}^{\mathscr{G}_K} = K_0$.
\end{prop}


\noindent
Finally we give a characterization of $A_{\mathrm{cris}}$ via
a universal property.
\begin{defn}[Formal divided power thickening]
	Let $A$ be a ring, with a principal ideal $\mathfrak{p}$
	equipped with divided powers and $V$ be an $A$-algebra.
	A \emph{$\mathfrak{p}$-adic divided power thickening of $V$}
	is a surjective homomorphism of $A$-algebras
	$\theta\colon D \to V$ such that $\ker \theta$ has a divided power structure
	compatible with those on $\mathfrak{p}$, similarly to \cref{defn:PDThickening}.
	Morphisms between divided power thickening are divided power morphisms
	making the obvious diagram commute.
	If the category of $\mathfrak{p}$-adic divided power $A$-thickening of $V$,
	whose objects and morphisms have just been defined, admits an initial object
	we call it the \emph{universal $\mathfrak{p}$-adic divided power $A$-thickening of $V$}.
	
	If, moreover, $V$ is separated and complete with respect to the $\mathfrak{p}$-adic topology
	we define \emph{formal $\mathfrak{p}$-adic divided power $A$-thickenings of $V$} to be
	$\mathfrak{p}$-adic divided power $A$-thickenings of $V$ which are separated and complete
	with respect to the $\mathfrak{p}$-adic topology.
	An initial object in the category of formal $\mathfrak{p}$-adic divided power $A$-thickening
	of $V$ is called universal as before.
\end{defn}


\begin{prop}[{\cite[\S2.3.2]{Fontaine}}]\label{UPACris}
	$A_{\mathrm{cris}}$ is a universal formal $p$-adic divided power $W$-thickening of
	$O_{\mathbb{C}_K}$.
\end{prop}
\begin{proof}
	It is clear that, by construction, $A_{\mathrm{cris}}$ is a
	formal $p$-adic divided power $W$-thickening of $O_{\mathbb{C}_K}$.
	In fact $\ker (A_{\mathrm{cris}} \to O_{\mathbb{C}_K})$
	is the $p$-adic completion of the P.D. ideal generated by $\ker \theta$
	in $\mathcal{D}_{W(R)}(\ker\theta)$.
	The latter is equipped with divided powers by construction, whereas
	the former is equipped with divided powers by \cref{PDExtendCompletion}.
	We then need to show universality.
	Let $\left(D, \theta_D, \gamma\right)$ be another formal $p$-adic divided power
	$W$-thickening of $O_{\mathbb{C}_K}$.
	Giving a morphism of formal $p$-adic divided power $W$-thickening
	from $A_{\mathrm{cris}}$ to $D$ is equivalent to giving
	a continuous morphism
	\begin{equation*}
	\begin{tikzcd}[row sep = 0ex
		,/tikz/column 1/.append style={anchor=base east}
		,/tikz/column 2/.append style={anchor=base west}]
		\alpha\colon W(R) \arrow[r, "", rightarrow] &
		D
	\end{tikzcd}
	\end{equation*} 
	between $p$-adic rings, such that $\theta_D \circ \alpha = \left.\theta\right|_{ W(R) }$.
	This is because universal property of divided powers envelope
	allows to extend $\alpha$ uniquely to $\mathcal{D}_{W(R)}(\ker\theta)$
	and continuity to $A_{\mathrm{cris}}$.
	Let's denote $J_D \coloneqq \ker\theta_D$.
	We now need to notice that, given $d_1 \equiv d_2 \mod J_D$ in $D$,
	then $d_1^p \equiv d_2^p \mod pJ_D$.
	In fact, given $d_1 \equiv d_2 \mod J_D$ there is $\lambda \in J_D$
	such that $d_2 = d_1 + \lambda$.
	Then
	\begin{equation*}
		d_2^p = d_1^p + \sum_{i=1}^{ p-1 } \binom{p}{i} d_1^i \lambda^{p-i} + \lambda^p
	.\end{equation*}
	Here we notice that $\binom{p}{i} d_1^i \lambda^{p-i} \in pJ_D$ for all $2 \leq i \leq p-1$
	and that $\lambda^p = p! \gamma_p(\lambda)$, since $J_D$ is equipped with divided
	powers.
	All in all we have $d_1^p - d_2^p \in pJ_D$ as we wished.
%	If $p \in J$ it is simple, since $pJ \subset J^2$ and
%	it is known that $d_1^p \equiv d_2^p \mod J^2$.
%	If, instead, $p \notin J$, we can see that 
%	\begin{equation*}
%		d_1^p - d_2^p = \left( d_1 - d_2 \right) 
%		\left( d_1^{p-1} + \cdots + d_2^{p-1} \right)
%	\end{equation*}
%	implies that $d_1^p - d_2^p \in J$.
%	Moreover, since $J$ has divided powers, we see that
%	\begin{equation*}
%		d_1^p - d_2^p = p! \left( \gamma_p(d_1) - \gamma_p(d_2) \right) = p \cdot y
%	.\end{equation*}
%	Since, moreover, $J$ is a prime ideal and $p \notin J$
%	we have $y \in J$ and $d_1^p - d_2^p \in pJ$.
	Fix now $\mathbf{x} \in R$ and take, for any $m \in \mathbb{N}$, 
	$\xi_m \in D$ a lift via $\theta_D$ of $\mathbf{x}^{(m)} \in O_{\mathbb{C}_K}$, defined
	as in \cref{not:tiltingElts}.
	Then, from what we stated before, the sequence $\xi_m^{p^m}$ converges
	$p$-adically to an element $\rho(\mathbf{x}) \in D$ which does not depend on the chosen lift.
	To prove this let's notice that, since $(x^{(m+1)})^p = x^{(m)}$ in $O_{\mathbb{C}_K}$,
	we have $\xi_{m+1}^p \equiv \xi_m \mod J_D$.
	But, by what we just proved, this implies that $\xi_{m+1}^{p^{m+1}} \equiv \xi_{m}^{p^m}
	\mod p^m J_D$, hence that the sequence $\left\{ \xi_m^{p^m} \right\}$
	is $p$-adically Cauchy.
	Since $D$ is $p$-adically complete this is enough to grant convergence.
	With regards to independence from the chosen lift, fix another family 
	$\left\{ \zeta_m \right\}$ of lifts of $x^{(m)}$ in $D$.
	Then, by construction, $\xi_m \equiv \zeta_m \mod J_D$ for all $m \in \mathbb{N}$, hence
	$\xi_m^{p^m} \equiv \zeta_{m}^{p^m} \mod p^m J_D$ by the above.
	This means that the sequence $\left\{ \xi_m^{p^m} - \zeta_m^{p^m} \right\}$
	is Cauchy, moreover it clearly converges to zero.
	Then, being $D$ a topological ring (hence sum is continuous),
	we obtain that the two sequences $\left\{ \zeta_m^{p^m} \right\}$ and
	$\left\{ \xi_m^{p^m} \right\}$ converge to the same point,
	which is then independent from the choice of lifts.
	Let's now recall that
	\begin{equation*}
	\theta([\mathbf{x}]) = \mathbf{x}^{(0)} =
	\lim_{n \to \infty} \widehat{\mathbf{x}_n}^{p^n}
	.\end{equation*}
	We want to construct a family of lifts
	$\xi_n$ such that $\xi_n \to \alpha([\mathbf{x}])$, in order
	to show that $\alpha([\mathbf{x}]) = \rho(\mathbf{x})$.
	We denote by $y_n \coloneqq \widehat{\mathbf{x}_n}^{p^n} \in O_{\mathbb{C}_K}$
	and by
	\begin{equation*}
		\mathbf{y}_n \coloneqq \left( y_n, y_n^{1/p}, y_n^{1/p^2}, \ldots \right) \in 
		\varprojlim_{x \mapsto x^p} O_{\mathbb{C}_K} \simeq R
	.\end{equation*}
	Then $\theta([\mathbf{y}_n]) = \widehat{\mathbf{x}_n}^{p^n}$ by definition of $\theta$.
	Moreover it is clear that $\mathbf{y}_n \to \mathbf{x}$ in $R$,
	hence that $[\mathbf{y}_n] \to [\mathbf{x}]$, if we endow $W(R)$ with the weak topology.
	The first statement is true since, by definition of the element $\mathbf{x}^{(0)}$
	(and independence of the chosen lift in its definition), we have
	$y_n \to \mathbf{x}^{(0)}$. Then an easy induction argument
	shows that $y_n^{1/p^m} \to \mathbf{x}^{(m)}$, hence that we have convergence
	in $R \simeq \varprojlim_{x \mapsto x^p}O_{\mathbb{C}_K}$ (notice that we
	can uniformly bound the distance of each component from its limit).
	By continuity of $\alpha$, then, we get that
	\begin{equation*}
		\alpha([\mathbf{x}]) = \alpha\left(\lim_{n \to \infty} [\mathbf{y}_n]\right) =
		\lim_{n \to \infty} \alpha([\mathbf{y}_n]) = \rho(\mathbf{x})
	,\end{equation*}
	where the last equality holds by compatibility of $\alpha$ with $\theta$,
	which implies that $\alpha([\mathbf{y}_n])$ is a lift of $\widehat{\mathbf{x}_n}^{p^n}$.
	Then, since $\alpha$ is a continuous morphism of $p$-adic rings,
	it acts on the general element of $W(R)$ by
	\begin{equation*}
	\begin{tikzcd}[row sep = 0ex
		,/tikz/column 1/.append style={anchor=base east}
		,/tikz/column 2/.append style={anchor=base west}]
		\alpha\colon 
		(\mathbf{x}_1, \mathbf{x}_2, \ldots) =
		\sum_{n \in \mathbb{N} }^{  } [\mathbf{x}_n^{p^{-n}}] p^n
		\arrow[r, "", mapsto] &
		\sum_{n }^{  } \rho (\mathbf{x}_n^{p^{-n}}) p^n
	.\end{tikzcd}
	\end{equation*} 
	This proves uniqueness, but it is also an explicit description of $\alpha$.
	As a consequence also existence is clear, since the above $\alpha$ is
	a continuous homomorphism which commutes with the morphisms $\theta$.
\end{proof}



\subsection{Comparison morphisms}
This last section will be dedicated to constructing and studying a few properties
of our desired comparison morphism and related ones.


\begin{rem}[]\label{elementsTateModule}
	Consider $G \in \mathsf{BT}/O_K$, seen as the inductive limit of $G_v$, 
	as in \cref{defn:pDivGroupFormalSchemes}.
	Let's notice that, in \cref{AlternativeDefnTateModule} we could have
	carried out the proof in $O_{\mathbb{C}_K}$ instead of $O_{\overline{K}}$.
	This follows from \cref{VCPBTGroups}, in which we saw that
	$G_v(\overline{K}) = G_v(O_{\mathbb{C}_K})$.
	Then we have
	\begin{equation*}
		T_p(G) \simeq \Hom_{\mathsf{BT}/O_{\mathbb{C}_K}} 
		\big( \underline{\mathbb{Q}_p/\mathbb{Z}_{p}}, G_{O_{\mathbb{C}_K}} \big)
	.\end{equation*}
\end{rem}


\begin{rem}[]\label{BaseChangeAcris}
	Among other things, \cref{PDExtendCompletion} states that
	$A_{\mathrm{cris}} \twoheadrightarrow O_{\mathbb{C}_K}$
	is a divided power thickening.
	Since we will need to compute $\mathbb{D}^*(G_{O_{\mathbb{C}_K}})(A_{\mathrm{cris}})$
	for certain $G \in \mathsf{BT}/O_K$, we wish to
	reduce its computation to that of $\mathbb{D}^*(G)(S)$.
	In order to do so, let's recall \cref{PBCrystals}, where we
	explicitly defined the pullback of our crystals.
	In particular consider the cartesian diagram
	\begin{equation*}
	\begin{tikzcd}
		V \coloneqq \spec(A_{\mathrm{cris}})
		\arrow[r, "", rightarrow] &
		U \\
		V_0 \coloneqq \spec(O_{O_{\mathbb{C}_K}})
		\arrow[u, "", hookrightarrow] \arrow[r, "f", rightarrow] &
		\spec(O_K) \eqqcolon U_0
		\arrow[u, "", hookrightarrow]
	.\end{tikzcd}
	\end{equation*}
	In this context we have $f^*G = G_{O_{\mathbb{C}_K}}$.
	Then stability under base change means 
	\begin{equation*}
	f^* \mathbb{D}^*(G)_{U_0 \hookrightarrow U} \simeq
	\mathbb{D}^*(G_{O_{\mathbb{C}_K}})_{V_0 \hookrightarrow V}
	\end{equation*}
	as Zariski sheaves.
	Moreover, since both $A_{\mathrm{cris}}$ and $O_K$ are endowed
	with a morphism from $S$ (hence with an $S$-module structure),
	from the universal property of pushouts, we obtain
	a morphism in $\mathsf{Crys}(O_K)$
	\begin{equation*}
	\begin{tikzcd}
		U \arrow[r, "\overline{\alpha}", rightarrow] &
		\spec(S) \\
		\spec(O_K)
		\arrow[u, "", hookrightarrow] \arrow[r, "", equals] &
		\spec(O_K) \arrow[u, "", hookrightarrow] 
	\end{tikzcd}
	\end{equation*}
	which we will denote by $\alpha$.
	Since crystals are special sheaves on $\mathsf{Crys}(O_K)$,
	as of \cref{defn:SpecialQCSheavesCrys},
	we finally have $\alpha^* \mathbb{D}^*(G)_{S \twoheadrightarrow O_K} =
	\mathbb{D}^*(G)_{U_0 \hookrightarrow U}$.
	Now we can conclude since, taking evaluations, we have
	\begin{equation*}
	\mathbb{D}^*(G_{O_{\mathbb{C}_K}})(A_{\mathrm{cris}}) \simeq
	f^* \mathbb{D}^*(G)(S) = A_{\mathrm{cris}} \otimes_S \mathbb{D}^*(G)(S)
	.\end{equation*}
\end{rem}


%\begin{rem}
%	tk: do I keep this?
%	Notice that, in the above, we don't really have a notion of
%	crystalline site \(\mathsf{Crys}(O_K)\), nor we can construct
%	\(\mathbb{D}^*(G)\) for \(G \in \mathsf{BT}/O_K\).
%	Though, thanks to \Cref{lem:BT/admissible,defn:NotNilpotentEvaluation},
%	we can carry out the above computations for the restriction
%	of \(G\) to \(O_K/pO_K\).
%	Then, the evaluations, will be given by a limit process over the
%	quotients of \(A_{\mathrm{cris}}\) and \(S\) by \((p^n)\).
%\end{rem}


\begin{rem}[]\label{constr:ComparisonMorphism}
	From \cref{elementsTateModule} we see that any $f \in T_p(G)$
	can be interpreted as a morphism of Barsotti-Tate groups
	on $O_{\mathbb{C}_K}$ between $\underline{\mathbb{Q}_p/\mathbb{Z}_{p}}$
	(base changed to $O_{\mathbb{C}_K}$) and $G_{O_{\mathbb{C}_K}}$.
	Since $\mathbb{D}^*$ acts contravariantly on Barsotti-Tate groups
	and covariantly on rings, it associates to \(f\) the map
%	\begin{equation*}
%	\begin{tikzcd}[row sep = 0ex
%		,/tikz/column 1/.append style={anchor=base east}
%		,/tikz/column 2/.append style={anchor=base west}]
%		\mathbb{D}^*(f)\colon 
%		\mathbb{D}^*(G_{O_{\mathbb{C}_K}})\arrow[r, "", rightarrow] &
%		\mathbb{D}^*(\underline{\mathbb{Q}_p/\mathbb{Z}_{p}})
%	.\end{tikzcd}
%	\end{equation*} 
%	Then we can evaluate the above at $A_{\mathrm{cris}}$ and obtain
	\begin{equation*}
	\begin{tikzcd}[row sep = 0ex
		,/tikz/column 1/.append style={anchor=base east}
		,/tikz/column 2/.append style={anchor=base west}]
		\mathbb{D}^*(f)(A_{\mathrm{cris}})\colon 
		\mathbb{D}^*(G_{O_{\mathbb{C}_K}})(A_{\mathrm{cris}})
		\arrow[r, "", rightarrow] &
		\mathbb{D}^*(\underline{\mathbb{Q}_p/\mathbb{Z}_{p}})(A_{\mathrm{cris}})
	.\end{tikzcd}
	\end{equation*} 
	We use \cref{BaseChangeAcris} to compute those crystals
	in terms of the modules $M(G) = \mathbb{D}^*(G)(S)$ of \cref{prop:A6Kisin}.
	In particular, from \cref{DieudonneModulesSimpleExamples}, we get
	$\mathbb{D}^*(\underline{\mathbb{Q}_p/\mathbb{Z}_{p}})(S) = S$.
	Then we have
	\begin{equation*}
	\mathbb{D}^*(\underline{\mathbb{Q}_p/\mathbb{Z}_{p}})(A_{\mathrm{cris}})
	\simeq A_{\mathrm{cris}}
	\qquad \text{ and } \qquad
	\mathbb{D}^*(G_{O_{\mathbb{C}_K}})(A_{\mathrm{cris}}) \simeq
	A_{\mathrm{cris}} \otimes_S \mathbb{D}^*(G)(S)
	.\end{equation*}
	Recalling that $A_{\mathrm{cris}}$ has an $S$-algebra structure,
	defined in \cref{AcrisAlgebraStructures}, we obtain a pairing
	\begin{equation*}
	\begin{tikzcd}[row sep = 0ex
		,/tikz/column 1/.append style={anchor=base east}
		,/tikz/column 2/.append style={anchor=base west}]
		T_p(G) \times \mathbb{D}^*(G_{O_{\mathbb{C}_K}})(A_{\mathrm{cris}}) 
		\arrow[r, "", rightarrow] &
		A_{\mathrm{cris}} \\
		(f,a) \arrow[r, "", mapsto] &
		\mathbb{D}^*(f)(A_{\mathrm{cris}})(a)
	.\end{tikzcd}
	\end{equation*} 
	But this allows to associate to each $a \in \mathbb{D}^*(G_{O_{\mathbb{C}_K}})(A_{\mathrm{cris}})$
	the evaluation morphism
	\begin{equation*}
	\begin{tikzcd}[row sep = 0ex
		,/tikz/column 1/.append style={anchor=base east}
		,/tikz/column 2/.append style={anchor=base west}]
		\rho_a\colon T_p(G)
		\arrow[r, "", rightarrow] &
		A_{\mathrm{cris}} \\
		f \arrow[r, "", mapsto] & 
		\mathbb{D}^*(f)(A_{\mathrm{cris}})(a)
	.\end{tikzcd}
	\end{equation*} 
	Since $\mathbb{D}^*$ is an additive functor this morphism
	is $\mathbb{Z}$-linear, moreover it can be shown to be $\mathbb{Z}_{p}$-linear too.
	As stated in \cref{rem:TateModuleProperties}, $T_p(G)$ is a free
	$\mathbb{Z}_{p}$-module of finite rank.
	As a consequence we have a canonical isomorphism
	\begin{equation*}
	\begin{tikzcd}[row sep = 0ex
		,/tikz/column 1/.append style={anchor=base east}
		,/tikz/column 2/.append style={anchor=base west}]
		\Hom_{\mathbb{Z}_{p}\text{-}\mathsf{Mod}}
		\left( T_p(G), \mathbb{Z}_{p} \right) \otimes_{\mathbb{Z}_{p}} A_{\mathrm{cris}}
		\arrow[r, "\sim", rightarrow] &
		\Hom_{ \mathbb{Z}_{p}\text{-}\mathsf{Mod} } \left( T_p(G), A_{\mathrm{cris}} \right)
	.\end{tikzcd}
	\end{equation*} 
	All in all, the above allows us to define the following
	homomorphism
	\begin{equation*}
	\begin{tikzcd}[row sep = 0ex
		,/tikz/column 1/.append style={anchor=base east}
		,/tikz/column 2/.append style={anchor=base west}]
		\rho_G\colon 
		A_{\mathrm{cris}} \otimes_S \mathbb{D}^*(G)(S)
		\arrow[r, "", rightarrow] &
		A_{\mathrm{cris}} \otimes_{\mathbb{Z}_{p}} T_p(G)^\vee
	.\end{tikzcd}
	\end{equation*} 
\end{rem}


\begin{defn}[]
	Fix $\big\{ \pi^{1/p^n} \big\}$,
	a compatible family of roots of the uniformizer $\pi$ of $K$.
	By compatible we mean that $(\pi^{1/p^n})^p = \pi^{1/p^{n-1}}$
	for all $n \in \mathbb{N}$.
	We define the algebraic extension $K_{\infty}/K$ as the extension given
	by $K_\infty \coloneqq \bigcup_{n \in \mathbb{N}} K(\pi^{1/p^n})$.
	As usual we will denote by $\mathscr{G}_{K_\infty} \coloneqq \gal
	\left( \overline{K} / K_\infty \right)$ 
	the absolute Galois group of $K_\infty$.
\end{defn}


\begin{thm}[{\cite[\S6, theorem 7]{Faltings}}]\label{ComparisonTheorem}
	The morphism constructed in \cref{constr:ComparisonMorphism}
	\begin{equation*}
	\begin{tikzcd}[row sep = 0ex
		,/tikz/column 1/.append style={anchor=base east}
		,/tikz/column 2/.append style={anchor=base west}]
		\rho_G\colon A_{\mathrm{cris}} \otimes_S \mathbb{D}^*(G)(S) 
		\arrow[r, "", rightarrow] &
		A_{\mathrm{cris}} \otimes_{\mathbb{Z}_{p}} T_p(G)^\vee
	\end{tikzcd}
	\end{equation*} 
	is a functorial $\mathscr{G}_{K_\infty}$-equivariant injection 
	which respects Frobenius and filtrations.
	Moreover the cokernel of $\rho_G$ is annihilated by $t$.
\end{thm}
\begin{proof}
	The idea of the proof, as carried out in \cite[\S6]{Faltings},
	is to first concentrate on the particular case of $G = \mathbb{G}_m(p)$
	and then reduce, using functoriality of $\rho_G$, the general case to this one.
	Here, though, we will not be concerned with filtrations nor with Frobenius.

	Let's start by tackling functoriality and $\mathscr{G}_{K_\infty}$-equivariance.
	Both of them can be explicitly checked in the construction of 
	\cref{constr:ComparisonMorphism}.
	In fact each step is clearly functorial
	and, upon inspecting the various action at each step, it turns out
	that each is also $\mathscr{G}_{K_\infty}$-equivariant.

	Let's now concentrate on the particular case, so let's fix $G = \mathbb{G}_m(p)$ for now.
	We need to explicitly compute the morphism $\rho_G$, using Messing's theory,
	in order to prove that $t$ kills $\coker \rho_G$.
	Recall that $T_p(G) = \mathbb{Z}_{p}(1)$, hence that $T_p(G)^\vee = \mathbb{Z}_{p}(-1)$.
	Now, since $\mathscr{G}_{K}$ acts on $t \in A_{\mathrm{cris}}$ via the cyclotomic
	character, we have an isomorphism of $\mathscr{G}_{K}$-modules $\mathbb{Z}_{p}(-1)
	\simeq \mathbb{Z}_{p} t^{-1}$.
	Let's recall that we denoted $t = \log([\boldsymbol\varepsilon]) \in A_{\mathrm{cris}}$.
	Moreover, as defined in \cref{ExampleEltsTilt}, we will use
	the notation $\boldsymbol\varepsilon^{(n)}$ to denote
	our fixed $p^n$th root of unity $\zeta_{p^n} \in O_{\mathbb{C}_K}$,
	Then $t \in T_p(G)$ corresponds, as shown in \cref{AlternativeDefnTateModule},
	to a morphism of Barsotti-Tate groups $\underline{\mathbb{Q}_p/\mathbb{Z}_{p}} \to \mathbb{G}_m(p)$.
	In particular, when evaluated at $O_{\mathbb{C}_K}$, it gives rise to
	\begin{equation*}
	\begin{tikzcd}[row sep = 0ex
		,/tikz/column 1/.append style={anchor=base east}
		,/tikz/column 2/.append style={anchor=base west}]
		u_0\colon 
		\mathbb{Q}_p/\mathbb{Z}_{p}
		\arrow[r, "", rightarrow] &
		\mu_{p^\infty}(O_{\mathbb{C}_K}) \\
		\frac{ 1 }{ p^n } \arrow[r, "", mapsto] & 
		\boldsymbol\varepsilon^{(n)}
	,\end{tikzcd}
	\end{equation*} 
	where by $\mu_{p^\infty}(O_{\mathbb{C}_K})$ we mean the multiplicative group
	of all $p^\infty$ roots of unity in $O_{\mathbb{C}_K}$.
	Starting from the construction of \cref{UniversalExtensionConstruction} 
	one can compute the universal extension of $G$ and $G^D$.
	In particular they are
	\begin{equation*}
	\begin{tikzcd}
		0 \arrow[r, "", rightarrow] &
		\mathbb{G}_a \arrow[r, "", rightarrow] &
		(\mathbb{G}_a \oplus \underline{\mathbb{Q}_p}) / \underline{\mathbb{Z}_{p}} 
		\arrow[r, "", rightarrow] &
		\underline{\mathbb{Q}_p/\mathbb{Z}_{p}} \arrow[r, "", rightarrow] &
		0
	\end{tikzcd}
	\end{equation*}
	for $\underline{\mathbb{Q}_p/\mathbb{Z}_{p}}$, where the quotient
	is given by the pushout of $\mathbb{G}_a$ and $\underline{\mathbb{Q}_p}$
	over their common subgroup $\underline{\mathbb{Z}_{p}}$, and
	\begin{equation*}
	\begin{tikzcd}
		1 \arrow[r, "", rightarrow] &
		1 \arrow[r, "", rightarrow] &
		\mathbb{G}_m(p) \arrow[r, "", rightarrow] &
		\mathbb{G}_m(p) \arrow[r, "", rightarrow] &
		1
	\end{tikzcd}
	\end{equation*}
	for $\mathbb{G}_m(p)$.
	For this last notice that dual of $\mathbb{G}_m(p)$
	is étale, hence $\omega_{G^D} = 0$.
	Considering $t$ as a morphism of Barsotti-Tate groups, \cref{prop:MorUnivExts}
	grants the existence of a morphism of extensions which, when
	evaluated at $O_{\mathbb{C}_K}$, gives rise to the following commutative diagram
	\begin{equation*}
	\begin{tikzcd}
		0 \arrow[r, "", rightarrow] &
		O_{\mathbb{C}_K} 
		\arrow[d, "0", rightarrow] 
		\arrow[r, "", rightarrow] &
		\frac{ O_{\mathbb{C}_K} \oplus \mathbb{Q}_p }{ \mathbb{Z}_{p} }
		\arrow[d, "v_0", rightarrow] 
		\arrow[r, "", rightarrow] &
		\mathbb{Q}_p/\mathbb{Z}_{p}
		\arrow[d, "u_0", rightarrow] 
		\arrow[r, "", rightarrow] &
		0 \\
		1 \arrow[r, "", rightarrow] &
		1 \arrow[r, "", rightarrow] &
		\mu_{p^\infty}(O_{\mathbb{C}_K})
		\arrow[r, "", rightarrow] &
		\mu_{p^\infty}(O_{\mathbb{C}_K})
		\arrow[r, "", rightarrow] &
		1
	.\end{tikzcd}
	\end{equation*}
	Here it is clear that the map $v_0$, making the diagram commute, is defined by
	\begin{equation*}
	\begin{tikzcd}[row sep = 0ex
		,/tikz/column 1/.append style={anchor=base east}
		,/tikz/column 2/.append style={anchor=base west}]
		v_0\colon 
		(\lambda, x) \arrow[r, "", mapsto] & 
		u_0(x)
	.\end{tikzcd}
	\end{equation*} 
	Henceforth we will denote by $\boldsymbol\varepsilon^x \coloneqq u_0(x)$,
	for $x \in \mathbb{Q}_p/\mathbb{Z}_{p}$.
	In particular, if we write $x = a/p^m$, for $a \in \mathbb{Z}_{p}$ and $-m = \nu_p(x)$,
	we can explicitly compute $\boldsymbol\varepsilon^x \coloneqq 
	\left( \boldsymbol\varepsilon^{(m)} \right)^a$.
	\Cref{thm:UniqueLiftingCrystals}, recalling \cref{rem:NonNilpotentCrystalEvaluation}
	and that $A_{\mathrm{cris}}$ is $p$-adically complete,
	grants that $v_0$ can be uniquely
	lifted to a morphism
	\begin{equation*}
	\begin{tikzcd}[row sep = 0ex
		,/tikz/column 1/.append style={anchor=base east}
		,/tikz/column 2/.append style={anchor=base west}]
		v\colon ( A_{\mathrm{cris}} \oplus \mathbb{Q}_p) / \mathbb{Z}_{p} 
		\arrow[r, "", rightarrow] &
		A_{\mathrm{cris}}^{\times}
	\end{tikzcd}
	\end{equation*} 
	such that $\left.-v\right|_{ \underline{\mathcal{V}}(\mathbb{G}_m(p)_{A_{\mathrm{cris}}}) }$
	is an exponential (notice that, following the notation of \cref{thm:UniqueLiftingCrystals},
	$j = 0$ in our case).
	Moreover we can define a morphism
	\begin{equation*}
	\begin{tikzcd}[row sep = 0ex
		,/tikz/column 1/.append style={anchor=base east}
		,/tikz/column 2/.append style={anchor=base west}]
		\widetilde{v}\colon 
		(A_{\mathrm{cris}} \oplus \mathbb{Q}_p) / \mathbb{Z}_{p}
		\arrow[r, "", rightarrow] &
		A_{\mathrm{cris}}^{\times} \\
		(a,x) \arrow[r, "", mapsto] & 
		\exp(-at)\left[ \alpha(x) \right]
	,\end{tikzcd}
	\end{equation*} 
	where, writing as before $x = a/p^m$ with $-m = \nu_p(x)$,
	\begin{equation*}
	\alpha(x) \coloneqq \big( (\boldsymbol\varepsilon^{(m+n)})^a \big)_{n \in \mathbb{N}} \in
	R = \varprojlim O_{\mathbb{C}_K}
	.\end{equation*}
	Here it is important to notice that $A_{\mathrm{cris}}$ is complete
	with respect to the $p$-adic topology and that $\ker\theta$
	is equipped with divided powers, hence that the above map is well defined.
	Then, since $t \in \ker \theta$, ideal with divided powers, we also obtain that
	$\exp(-at) \in \ker\theta$.
	As a consequence we obtain that $\theta(\widetilde{v}(x)) = \boldsymbol\varepsilon^x$,
	i.e. that $\widetilde{v}$ lifts $v_0$.
	Moreover, restricting $v$ to $\underline{\mathcal{V}}
	(\underline{\mathbb{Q}_p/\mathbb{Z}_{p}}_{A_{\mathrm{cris}}})$ 
	is the same as computing it at $x=0$, which is an exponential.
	By uniqueness this grants that $v = \widetilde{v}$.
	Then we have explicitly computed $\mathbb{E}(u_0) \coloneqq \mathbb{E}(t) = \widetilde{v}$.
	Now, taking $\lie$ of all that, we get (what else could it be)
	\begin{equation*}
		\mathbb{D}^*(t)(A_{\mathrm{cris}})(a) = \log \left( \exp (-at) \right) = -at
	\end{equation*}
	for all $a \in A_{\mathrm{cris}}$.
	And this is exactly the required explicit construction of $\rho_G$, which
	then acts by
	\begin{equation*}
	\begin{tikzcd}[row sep = 0ex
		,/tikz/column 1/.append style={anchor=base east}
		,/tikz/column 2/.append style={anchor=base west}]
		\rho_G\colon 
		A_{\mathrm{cris}} \otimes_S \mathbb{D}^*(\mathbb{G}_m(p))(S)
		\arrow[r, "", rightarrow] &
		A_{\mathrm{cris}} \otimes_{\mathbb{Z}_{p}} \mathbb{Z}_{p} t^{-1} \\
		1 \otimes 1 \arrow[r, "", mapsto] & -t \otimes t^{-1}
	.\end{tikzcd}
	\end{equation*} 
	Here we clearly see that $t$ kills $\coker \rho_G$, since
	$\ima \rho_G = t \cdot (A_{\mathrm{cris}} \otimes_{\mathbb{Z}_{p}} \mathbb{Z}_{p} t^{-1})$.

	Let's now switch to the general case, in which $G \in \mathsf{BT}/O_K$
	is arbitrary.
	We want to explicitly construct a morphism $G_{O_{\overline{K}}} \to \mathbb{G}_m(p)$
	which will allow us to reduce the study of $\rho_G$ to the one we have just defined.
	To achieve this goal we recall that $T_p(G)^\vee = T_p(G^D)(-1) = t^{-1}T_p(G^D)$.
	Hence, given any $y \in T_p(G)^\vee$, we obtain $ty \in T_p(G^D)$,
	which can be interpreted as a map in 
	$\Hom_{ \mathsf{BT}/O_{\overline{K}} } 
	\big( \underline{\mathbb{Q}_p/\mathbb{Z}_{p}}, G^D_{O_{\overline{K}}}\big)$.
	By \cref{thm:CartierDuality} this gives a morphism
	\begin{equation*}
	\begin{tikzcd}[row sep = 0ex
		,/tikz/column 1/.append style={anchor=base east}
		,/tikz/column 2/.append style={anchor=base west}]
		\left( ty \right)^D\colon G_{O_{\overline{K}}}
		\arrow[r, "", rightarrow] &
		\mathbb{G}_m(p)_{O_{\overline{K}}}
	\end{tikzcd}
	\end{equation*} 
	Now, by functoriality of the comparison morphism $\rho_G$, we obtain the commutative
	diagram
	\begin{equation*}
	\begin{tikzcd}[column sep=4.0em]
		A_{\mathrm{cris}} \otimes_S \mathbb{D}^*(G)(S)
		\arrow[r, "\rho_G", rightarrow] &
		A_{\mathrm{cris}} \otimes_{\mathbb{Z}_{p}} \left( T_p(G) \right)^\vee \\
		A_{\mathrm{cris}} \otimes_S \mathbb{D}^*(\mathbb{G}_m(p))(S)
		\arrow[r, "\rho_{\mathbb{G}_m(p)}", rightarrow] 
		\arrow[u, "\id_{ A_{\mathrm{cris}} } \otimes \mathbb{D}^*((ty)^D)(A_{\mathrm{cris}})", rightarrow] &
		A_{\mathrm{cris}} \otimes_{\mathbb{Z}_{p}} \mathbb{Z}_{p} t^{-1}
		\arrow[u, "\id_{ A_{\mathrm{cris}} } \otimes T_p ((ty)^D)^\vee"', rightarrow] 
	,\end{tikzcd}
	\end{equation*}
	where we recall that both $\mathbb{D}^*$ and $T_p(G)^\vee$ act contravariantly on morphisms.
	Now we need to determine the morphism $T_p((ty)^D)^\vee$. %in order to reduce to the previous case.
	To do so let's notice that 
	\begin{equation*}
	\begin{tikzcd}[row sep = 0ex
		,/tikz/column 1/.append style={anchor=base east}
		,/tikz/column 2/.append style={anchor=base west}]
		T_p((ty)^D)^\vee(1) = T_p(ty)\colon 
		T_p \big( \underline{\mathbb{Q}_p/\mathbb{Z}_{p}} \big)
		\arrow[r, "", rightarrow] &
		T_p(G^D)\\
		1 = \left( 1/p^v \right)_{v \in \mathbb{N}}
		\arrow[r, "", mapsto] &
		ty
	.\end{tikzcd}
	\end{equation*} 
	Here, as computed before, we have $T_p ( \underline{\mathbb{Q}_p/\mathbb{Z}_{p}} ) =
	\mathbb{Z}_{p}$ and $T_p(G^D) = T_p(G)^\vee(1)$.
	Combined with the above this determines the morphism
	\begin{equation*}
	\begin{tikzcd}[row sep = 0ex
		,/tikz/column 1/.append style={anchor=base east}
		,/tikz/column 2/.append style={anchor=base west}]
		T_p((ty)^D)^\vee\colon 
		t^{-1}
		\arrow[r, "", mapsto] &
		y
	.\end{tikzcd}
	\end{equation*}
	Then, in the above square, we see that
	\begin{equation*}
	\begin{tikzcd}[column sep=4em]
		1 \otimes \mathbb{D}^*((ty)^D)(A_{\mathrm{cris}})(1) 
		\arrow[r, "\rho_G", mapsto] &
		-t \otimes y \\
		1 \otimes 1
		\arrow[u, "\id_{ A_{\mathrm{cris}} } \otimes \mathbb{D}^*((ty)^D)(A_{\mathrm{cris}})", mapsto] 
		\arrow[r, "\rho_{\mathbb{G}_m(p)}", mapsto] &
		-t \otimes t^{-1}
		\arrow[u, "\id_{ A_{\mathrm{cris}} } \otimes T_p((ty)^D)^\vee"', mapsto] 
	.\end{tikzcd}
	\end{equation*}
	Now, since we imposed no restriction to the choice of $y \in T_p(G)^\vee$, 
	commutativity of the square tells us that $t$ kills $\coker \rho_G$.

	We are finally left to prove injectivity.
	Notice that, thanks to classical Dieudonné theory,
	$\mathbb{D}^*(G)(S)$ is a free $S$-module of rank $h$.
	Then, inverting $t$ in $A_{\mathrm{cris}}$, i.e. extending scalars
	to $B_{\mathrm{cris}}$, we obtain a surjective map
	\begin{equation*}
	\begin{tikzcd}[row sep = 0ex
		,/tikz/column 1/.append style={anchor=base east}
		,/tikz/column 2/.append style={anchor=base west}]
		B_{\mathrm{cris}} \otimes_S \mathbb{D}^*(G)(S)
		\arrow[r, "", rightarrow] &
		B_{\mathrm{cris}} \otimes_S T_p(G)^\vee
	\end{tikzcd}
	\end{equation*} 
	between finitely generated free $B_{\mathrm{cris}}$-modules.
	But then, applying Cayley-Hamilton, we see that this is an isomorphism,
	so in particular it is injective.
	Now, since $\mathbb{D}^*(G)(S)$ and $T_p(G)^\vee$ are free modules,
	they are also flat, hence the inclusion $A_{\mathrm{cris}} \hookrightarrow B_{\mathrm{cris}}$
	induces the inclusions
	\begin{equation*}
	\begin{tikzcd}
		A_{\mathrm{cris}} \otimes_S 
		\mathbb{D}^*(G)(S)
		\arrow[r, "", hookrightarrow] &
		B_{\mathrm{cris}} \otimes_S
		\mathbb{D}^*(G)(S)
	\end{tikzcd}
	\end{equation*}
	and
	\begin{equation*}
	\begin{tikzcd}
		A_{\mathrm{cris}} \otimes_S 
		T_p(G)^\vee
		\arrow[r, "", hookrightarrow] &
		B_{\mathrm{cris}} \otimes_S
		T_p(G)^\vee
	.\end{tikzcd}
	\end{equation*} 
	This allows us to see $\rho_G$ as the restriction of an isomorphism to a subspace,
	which grants its injectivity.
	And we win.
\end{proof}


\begin{rem}[]
	Notice that it is important to consider only the action of $\mathscr{G}_{K_\infty}$.
	In fact, the $S$-module structure of $A_{\mathrm{cris}}$ is induced from
	the map $u \mapsto \left[ \boldsymbol\pi \right]$ from $W[u]$ to $W(R)$.
	Then an action on the tensor product $A_{\mathrm{cris}} \otimes_S \mathbb{D}^*(G)(S)$
	has to be compatible with this $S$-module structure.
	In particular, since $W \subset K$, it has to fix the element $\left[ \boldsymbol\pi \right]$,
	which is not the case for $\mathscr{G}_K$.
\end{rem}


\noindent
We remark that, base changing to $B_{\mathrm{cris}}$, the above result can actually be
improved upon.
\begin{thm}[]
	The base change to $B_{\mathrm{cris}}$ of the morphism
	constructed in \cref{constr:ComparisonMorphism}
	\begin{equation*}
	\begin{tikzcd}[row sep = 0ex
		,/tikz/column 1/.append style={anchor=base east}
		,/tikz/column 2/.append style={anchor=base west}]
		B_{\mathrm{cris}} \otimes_S \mathbb{D}^*(G)(S) \arrow[r, "\sim", rightarrow] &
		B_{\mathrm{cris}} \otimes_{\mathbb{Z}_{p}} T_p(G)^\vee
	\end{tikzcd}
	\end{equation*} 
	is a $\mathscr{G}_{K_\infty}$-equivariant isomorphism compatible with filtrations and Frobenius.
\end{thm}


\begin{rem}[]
	This result actually extends previously known 
	results in the context of classical Dieudonné theory.
	In fact, given $G \in \mathsf{BT}/O_K$, let's denote by $G_k$
	its base change to $k = O_K / \mathfrak{m}$ the residue field of $O_K$.
	Then, denoted by $\mathbb{D}^*(G_k)$ the Dieudonné module associated to
	$G_k$, we have the isomorphism of modules
	\begin{equation*}
		\mathbb{D}^*(G)(S) \simeq S \otimes_W \mathbb{D}^*(G_k)
	.\end{equation*}
	Though here we have not defined a natural filtration.
	As a consequence it will not be of concern when looking at the classical
	comparison morphism.
\end{rem}


\begin{thm}[]
	Let $G \in \mathsf{BT}/O_K$ and $G_k$ be as above.
	Then we have a functorial $\mathscr{G}_K$-equivariant injection
	\begin{equation*}
	\begin{tikzcd}[row sep = 0ex
		,/tikz/column 1/.append style={anchor=base east}
		,/tikz/column 2/.append style={anchor=base west}]
		\rho_G\colon 
		A_{\mathrm{cris}} \otimes_W \mathbb{D}^*(G_k)
		\arrow[r, "", rightarrow] &
		A_{\mathrm{cris}} \otimes_{\mathbb{Z}_{p}} \left( T_p(G) \right)^\vee
	\end{tikzcd}
	\end{equation*} 
	which is also compatible with Frobenius.
	Inverting $t$ we obtain a functorial $\mathscr{G}_{K}$-equivariant isomorphism
	\begin{equation*}
	\begin{tikzcd}[row sep = 0ex
		,/tikz/column 1/.append style={anchor=base east}
		,/tikz/column 2/.append style={anchor=base west}]
		\rho\colon B_{\mathrm{cris}} \otimes_W \mathbb{D}^*(G_k)
		\arrow[r, "", rightarrow] &
		B_{\mathrm{cris}} \otimes_{\mathbb{Z}_{p}} \left( T_p(G) \right)^\vee
	\end{tikzcd}
	\end{equation*} 
	which again is compatible with Frobenius.
\end{thm}


\begin{rem}[]
	Classically, in fact, in order to recover a natural filtration on $\mathbb{D}^*(G_k)$,
	one needs to base change to $B_{\mathrm{dR}}$.
	In fact one can define $D_{K} \coloneqq O_K \otimes_W \mathbb{D}^*(G_k)$.
	Then 
	\begin{equation*}
		D_K = \mathbb{D}^*(G)(S)/\left( \operatorname{Fil}^1S \otimes_W \mathbb{D}^*(G_k)\right)
	,\end{equation*}
	hence it inherits a filtration from that on $\mathbb{D}^*(G)(S)$, via
	\begin{equation*}
	\operatorname{Fil}^1 D_K \coloneqq 
	\operatorname{Fil}^1 / \left( \operatorname{Fil}^1S \otimes_W \mathbb{D}^*(G_k) \right)
	.\end{equation*}
	And now this induces the usual filtration on the scalar extension
	$B_{\mathrm{dR}} \otimes_{W} D_K$, via 
	\begin{align*}
		\operatorname{Fil}^i \left( B_{\mathrm{dR}} \otimes_{W} D_K \right) &=
		\sum_{j \in \mathbb{Z} }^{  } \ima \left( 
		\operatorname{Fil}^j B_{\mathrm{dR}} \otimes_{W} \operatorname{Fil}^{i-j} D_K \right) \\
		&=
		\operatorname{Fil}^{i-1} B_{\mathrm{dR}} \otimes_{W} \operatorname{Fil}^1 D_K +
		\operatorname{Fil}^{i} B_{\mathrm{dR}} \otimes_{W} D_K
	,\end{align*}
	since the (decreasing) filtration on $D_K$ satisfies
	$\operatorname{Fil}^0 D_K = D_K$ and $\operatorname{Fil}^2 D_K = 0$.
	Moreover one defines
	\begin{equation*}
		\operatorname{Fil}^i \left( B_{\mathrm{dR}} \otimes_{\mathbb{Z}_{p}} T_p(G) \right) \coloneqq
		\operatorname{Fil}^i B_{\mathrm{dR}} \otimes_{\mathbb{Z}_{p}} T_p(G)
	.\end{equation*}
	With all this in mind one can prove that the base change of the
	above isomorphism $\rho$ to $B_{\mathrm{dR}}$, i.e.
	\begin{equation*}
	\begin{tikzcd}[row sep = 0ex
		,/tikz/column 1/.append style={anchor=base east}
		,/tikz/column 2/.append style={anchor=base west}]
		\id_{ B_{\mathrm{dR}} } \otimes \rho\colon 
		B_{\mathrm{dR}} \otimes_W D \arrow[r, "", rightarrow] &
		B_{\mathrm{dR}} \otimes_{\mathbb{Z}_{p}} T_p(G)
	\end{tikzcd}
	\end{equation*} 
	is an isomorphism compatible with the filtrations we have just defined.
\end{rem}

\documentclass[../Main]{subfiles}
\begin{document}
\section{\texorpdfstring{$p$}{p}-divisible groups}
In all that follows $p$ will be a fixed prime number.
Moreover $R$ denotes a commutative ring.

The aim of this section is to introduce, from two different points of view,
the notions of {\em $p$-divisible group} and of {\em formal Lie group},
and to show how the two concepts are related to one another.
tk: do we keep it? Before doing so, though, we need to introduce a new notion, that
of formal scheme.


\subsection{Formal schemes}
These concepts are thought to allow for the defintion of
infinitesimal neighbourhoods of subschemes of schemes.
Let's start by recalling a few useful algebra defintions.

\begin{defn}[Topological rings and modules]\leavevmode\vspace{-1\baselineskip}
\begin{enumerate}
\item We say that a ring $R$ is a {\em topological ring} iff it is a ring endowed with a topology
	such that both addition and multiplication are continuous maps
	$R \cross R \to R$, where $R \cross R$ is taken with the product topology.

\item We say that an $R$-module $M$, where $R$ is a topological ring,
	is a {\em topological module} iff $M$ is enodwed with a topology such that
	addition and scalar multiplication are both continuous, again with their source
	taken with the product topology.

\item $R$ is {\em linearly topologized} iff $0$ has a fundamental system
	of neighbourhoods consisting of ideals.
	Analogously $M$ is {\em linearly topologized} iff $0$ has a fundamental
	system of neighbourhood consisting of submodules.

\item If $R$ is linearly topologized, we say that the ideal $I \triangleleft R$
	is an {\em ideal of definition} iff $I$ is open and every neighbourhood
	of $0$ contains $I^n$ for an appropriate $n \in \mathbb{N}$.
\end{enumerate}
\end{defn}


\begin{defn}[Complete tensor product]
	Let $R$ be a topological ring and $M, N$ be linearly topologized $R$-modules.
	Let $M_\mu \triangleleft M$ and $N_\nu \triangleleft N$ run through
	fundamental systems of open submodules of $M$ and $N$ respectively.
	We endow the tensor product of $M$ and $N$ with the linear topology
	defined by the fundamental system of open submodules
	\begin{equation*}
	\begin{tikzcd}
	\ima \{ M_\mu \otimes_R N + M \otimes_R N_\nu 
		\arrow[r, "", rightarrow] &
	M \otimes_R N \}
	.\end{tikzcd}
	\end{equation*}
	Then we define the {\em completed tensor product} as the completion
	of the tensor product, i.e.
	\begin{equation*}
		M \widehat{\otimes}_R N = 
		\varprojlim \frac{M \otimes_R N}{M_\mu \otimes_R N +
		M \otimes_R N_\nu} =
		\varprojlim M / M_\mu \otimes_R N / N_\nu
	.\end{equation*}
\end{defn}


\begin{rem}[]
	In the case where $R$ is a complete topological ring,
	$M = R [\![ X_1, \ldots, X_{ n } ]\!]$
	and $N = R [\![ Y_1, \ldots, Y_{ m } ]\!]$,
	one obtains the isomorphism
	\begin{equation*}
		R [\![ X_1, \ldots, X_{ n } ]\!] \widehat{\otimes}_R
		R [\![ Y_1, \ldots, Y_{ m } ]\!] \simeq
		R [\![ S_1, \ldots, S_n, T_1 \ldots, T_{ m } ]\!]
	.\end{equation*}
	Above we denoted by $S_j \coloneqq X_j \otimes 1$
	and by $T_j \coloneqq 1 \otimes Y_j$.
\end{rem}


\begin{thm}[{\cite[\S5]{Shatz}}, Grothendieck]
	A functor from profinite $R$-algebras to sets (resp$.$ groups)
	is representable iff it is {\em left exact}, i.e.
	iff it commutes with fibered products and final elements.
\end{thm}
In what follows this theorem will be applied, sometimes
to formal schemes. In such case, the anti-equivalence of categories (tk: make sure you define it)
implies that one need to check {\em right exactness} instead of left exactness.



\subsection{Formal Lie groups}
In this section we will define the concept of {\em formal Lie group},
generalizing that of Lie group, i.e. that of group (tk: I'd look for a description
on wikipedia, if I'm not mistaken it is the infinitesimal linear expansion of
the action of a VS on a manifold), without the restriction of convergence.

As will be usual (tk: has it become so?) throughout this section we will define
concepts both in terms of formal schemes, borrowing from \cite{Shatz},
and in terms of {\em fppf} sheaves, borrowing from \cite{Messing}.


tk: the following is just a draft. Please check its correctness

\subsubsection{Formal scheme point of view}
Here we fix $R$ a complete local ring and $S = \mathrm{Spec}(R)$.
tk: for this definition I need the connected-etale short exact sequence...
\begin{defn}[Formal Lie variety]\leavevmode\vspace{-.2\baselineskip}
\begin{enumerate}
	\item A formal $S$-scheme $G$ is said to be {\em smooth} iff
		$G^0$ is the formal spectrum of a power series ring over $R$.
	\item A {\em formal Lie variety} over $S$ is a smooth, connected formal $S$-scheme.
\end{enumerate}
\end{defn}

tk: these definitions will need to be checked once you settle down on how 
to approach the previous ones.
\begin{defn}[Formal Lie group]
%	Let $R$ be a complete, noetherian, local ring,
%	with residue field $k$ of characteristic $p > 0$.
%	Let $\mathscr{A} \coloneqq R [\![ X_1, \ldots, X_{ n } ]\!]$
%	be the ring of formal power series in $n$ variables,
%	and $\mathscr{A} \widehat{\otimes}_R \mathscr{A}$ the ring of formal power series 
%	in $2n$ variables.
%	An {\em $n$-dimensional formal Lie group $\Gamma$ over $R$} is 
%	given by $\Gamma \coloneqq \mathrm{Spec}\, \mathscr{A}$,
%	(tk: are you sure? The more I read, the more I am {\em un}sure...)
%	where the group law is defined by a homomorphism of $R$-algebras 
	A {\em formal Lie group} is a group object in the category of
	formal Lie varieties.
	More explicitly it is $\Gamma \coloneqq \mathrm{Spf}\left( \mathscr{A} \right)$,
	where $\mathscr{A} \coloneqq R [\![ X_1, \ldots, X_{ n } ]\!]$
	and $n$ is its dimension. 
	Moreover its group law is given by a ring homomorphism
	\begin{equation*}
	\begin{tikzcd}[row sep = 0ex
		,/tikz/column 1/.append style={anchor=base east}
		,/tikz/column 2/.append style={anchor=base west}]
		f\colon \mathscr{A} \arrow[r, "", rightarrow] &
		\mathscr{A} \widehat{\otimes}_R \mathscr{A} =
		R [\![ X_1, \ldots, X_{ 2n } ]\!]
	,\end{tikzcd}
	\end{equation*} 
	satisfying the following conditions:
	\begin{enumerate}
		\item {\em $\varepsilon$ axiom:} $X = f(X,0) = f(0,X)$,
		\item {\em coassociativity:} $f(X, f(Y,Z)) = f(f(X,Y), Z)$,
		\item {\em commutativity:} $f(X,Y) = f(Y,X)$.
	\end{enumerate}
	Notice that any such morphism $f$ is just the data of 
	$\left( f_i(Y,Z) \right)_{i=1}^n$, power series in $2n$ variables, where
	$f_i$ is the image of $X_i$ via $f$.
	Then one, at the level of $R$-algebras, introduces the notation
	\begin{equation*}
		X \ast Y \coloneqq f(X,Y)
	.\end{equation*} 
\end{defn}

\begin{rem}[]
	These axioms are enough to grant the existance of the inverse for any element of $\Gamma$,
	hence they suffice to give $\Gamma$ a group scheme structure.
\end{rem}


\begin{defn}[divisible formal groups]
	We define the map multiplication by $p$ on $\Gamma$ as the map $p\colon \Gamma \to \Gamma$
	associated to 
	\begin{equation*}
	\begin{tikzcd}[row sep = 0ex
		,/tikz/column 1/.append style={anchor=base east}
		,/tikz/column 2/.append style={anchor=base west}]
		\psi\colon \mathscr{A} \arrow[r, "", rightarrow] &
		\mathscr{A} \widehat{\otimes}_R \mathscr{A} \\
		X \arrow[r, "", mapsto] & 
		X \ast \ldots \ast X
	\quad \text{($p$ times)}
	.\end{tikzcd}
	\end{equation*} 
	A formal Lie group $\Gamma$ is said to be {\em divisible} iff the map $p$ is 
	an isogeny, i.e. it is surjective and has finite kernel.
\end{defn}


\subsection{\texorpdfstring{$p$}{p}-divisible groups}
\begin{defn}[$p$-divisible group]
	A {\em $p$-divisible group} over $R$ of height $h \in \mathbb{N}_+$ is an inductive system
	\begin{equation*}
	G \coloneqq \left(G_\nu, i_\nu\right)_{\nu \in \mathbb{N}}
	,\end{equation*} 
	satisfying:
	\begin{enumerate}
		\item for each $\nu \in \mathbb{N}$, $G_\nu$ is a finite, commutative group scheme over $R$
			of order $p^{\nu h}$,
		\item for each $\nu \in \mathbb{N}$, there is an exact sequence
			\begin{equation*}
			\begin{tikzcd}
				0 \arrow[r, "", rightarrow] &
				G_\nu \arrow[r, "i_\nu", rightarrow] &
				G_{\nu + 1} \arrow[r, "p^\nu", rightarrow] &
				G_{\nu + 1} 
			,\end{tikzcd}
			\end{equation*}
			where the second map is the multiplication by $p^\nu$ in $G_{\nu + 1}$,
			hence the first is a closed immersion.
	\end{enumerate}
\end{defn}

\begin{rem}[]
	In the case of ordinary abelian groups this would give rise just to 
	$G_\nu = \left( \mathbb{Z}/p^\nu\mathbb{Z} \right)^h$, hence to the $p$-divisible group
	\begin{equation*}
		G = \varinjlim G_\nu = \left( \mathbb{Q}_p / \mathbb{Z}_p \right)^h
	.\end{equation*} 
\end{rem}

\begin{ex}
	In such case one can define a $p$-divisible group of height $h$
	over $R$, starting from $\Gamma$, by:
	\begin{equation*}
		\Gamma(p) \coloneqq \left(\Gamma_{p^\nu}, i_{p^\nu}\right)_\nu
	.\end{equation*} 
	In the above $\Gamma_{p^\nu}$ is the kernel of the multiplication by $p^\nu$ 
	in $\Gamma$.
	By questions of connectedness (tk: see short exact sequence of connected
	and etale groups) one sees that the order of $\Gamma_{p^\nu}$
	is a power of $p$, that $\Gamma_{p^\nu}$ is connected.
	hence that $\Gamma(p)$ is a connected $p$-divisible group.

	tk: complete it.
\end{ex}

\begin{prop}[]
	Let $R$ be a complete noetherian local ring whose residue field $k$
	is of characteristic $p > 0$.
	Then $\Gamma \mapsto \Gamma(p)$ is an equivalence of categories
	between the category of divisible commutative formal Lie groups over $R$
	and the category of connected $p$-divisible groups over $R$.
\end{prop}

\begin{defn}[Dimension of a $p$-divisible group]
	Let $G \coloneqq \left(G_{\nu}, i_\nu\right)_{\nu \in \mathbb{N}}$ be a $p$-divisible group
	over $R$ as before.
	The connected components $G^0_\nu$ determine a connected $p$-divisible
	group $G^0$.
	Moreover, from the short exact sequence
	\begin{equation*}
	\begin{tikzcd}
		0 \arrow[r, "", rightarrow] &
		G^0_\nu \arrow[r, "", rightarrow] &
		G_\nu \arrow[r, "", rightarrow] &
		G^{et}_\nu \arrow[r, "", rightarrow] &
		0
	\end{tikzcd}
	\end{equation*}
	one gets the exact sequence
	\begin{equation*}
	\begin{tikzcd}
		0 \arrow[r, "", rightarrow] &
		G^0 \arrow[r, "", rightarrow] &
		G \arrow[r, "", rightarrow] &
		G^{et} \arrow[r, "", rightarrow] &
		0
	,\end{tikzcd}
	\end{equation*}
	where $G^{et}$ is an e\'tale $p$-divisible group.
	One then defines the dimension of $G$ to be the dimension 
	(i.e. the number of variables of $\mathscr{A}$)
	of the formal Lie group corresponding, as of proposition (tk: reference to the above one),
	to $G^0$.
\end{defn}


tk: add section on dual of $p$-divisible group. Maybe add a remark about
the definition from Messing, in order to prepare to
the comparison with formal Lie groups.

\end{document}

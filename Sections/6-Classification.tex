\section{Classification of \texorpdfstring{$p$}{p}-divisible groups
	over \texorpdfstring{$\mathcal{O}_{ K }$}{the ring of integers of K}}
This section will follow \cite[Appendix A]{Kisin}.
We fix $k$ a perfect field of characteristic $p$,
$W \coloneqq W(k)$ its ring of Witt vectors, $K_0 \coloneqq W[1/p]$
its field of fractions and $K/K_0$ a finite totally ramified extension.
Fix now a uniformizer $\pi$ of $K$ with (Eisenstein) minimal polynomial
$E(u) \in K_0[u]$.


\begin{rem}[]
	The formation of $\mathbb{D}(G)$ and of $\mathbb{D}^*(G)$ is compatible
	with all base changes. %(tk: why?).
	In particular, if $p = 0$ on $T$ we can pull $G$ back by the Frobenius $\varphi$ on $T$.
	Then the relative Frobenius on $G$ gives a map $G \to \varphi^*(G)$
	hence a map of crystals
	\begin{equation*}
	\begin{tikzcd}
		\varphi^* (\mathbb{D}^*(G)) \arrow[r, "\sim", rightarrow] &
		\mathbb{D}^*(\varphi^*(G)) \arrow[r, "", rightarrow] &
		\mathbb{D}^*(G)
	.\end{tikzcd}
	\end{equation*}
\end{rem}


\begin{ntt}[]
	In the following we will mainly be interested in the valuation of
	crystals on objects.
	In particular, given $G_0 \in \mathsf{BT}(T_0)$ and 
	$T_0 \hookrightarrow T \in \mathsf{Crys}(T_0)$, we will be interested
	in $\mathbb{D}^*(G_0)_{T}(T) = \mathbb{D}^*(G_0)_{T_0 \hookrightarrow T}(T)$,
	where $\mathbb{D}^*(G_0)_T$ is defined in \cref{CrystalAssociatedZarSheafS}
	and its evaluation in \cref{rem:SheavesCrysX}.
	Then we introduce the shorter notation
	$\mathbb{D}^*(G_0)(T) \coloneqq \mathbb{D}^*(G_0)_T(T)$.
	Finally, if $T = \mathrm{Spec}(A)$ is affine, we introduce the notation
	$\mathbb{D}^*(G)(A)$ for $\mathbb{D}^*(G)(T)$.
\end{ntt}


\begin{rem}[]\label{rem:DGValuationRestriction}
	Suppose that $T_0$ is a scheme over $W$ and that $p = 0$ on $T_0$.
	Consider $G_0 \in \mathsf{BT}(T_0)$ and $T_0 \hookrightarrow T \in \mathsf{Crys}(T_0/W)$
	an object on which $p$ is locally nilpotent, and $G$ a lifting of $G_0$ to $T$.
	By construction of $\mathbb{D}^*$ we have an isomorphism
	\begin{equation*}
	\begin{tikzcd}[row sep = 0ex
		,/tikz/column 1/.append style={anchor=base east}
		,/tikz/column 2/.append style={anchor=base west}]
		\mathbb{D}^*(G_0)(T) \arrow[r, "\sim", rightarrow] &
		\mathbb{D}^*(G)(T)
	.\end{tikzcd}
	\end{equation*} 
	%tk: prove it.
	%In fact, as defined above, the left hand side is just
	%$\mathbb{D}^*(G_0)_{T_0 \hookrightarrow T}(T)$ and the right hand side
	%$\mathbb{D}^*(G)_{T \hookrightarrow T}(T)$.
	%Then it suffices to prove the equality of Zariski sheaves on $T$
	%\begin{equation*}
	%\mathbb{D}^*(G_0)_{T_0 \hookrightarrow T} = \mathbb{D}^*(G)_{T \hookrightarrow T}
	%.\end{equation*}
	%In fact, in general, a nilpotent closed immersion $U_0 \hookrightarrow U$
	%induces a morphism $\mathsf{Crys}(U/S) \to \mathsf{Crys}(U_0/S)$
	%by base changing $T \to U$ an open immersion to $U_0$.

	Moreover the $\mathcal{O}_{ T }$-module $\mathbb{D}^*(G)(T)$ sits in an exact sequence
	\begin{equation*}
	\begin{tikzcd}
		0 \arrow[r, "", rightarrow] &
		\left( \mathrm{Lie}(G) \right)^D \arrow[r, "", rightarrow] &
		\mathbb{D}^*(G)(T) \arrow[r, "", rightarrow] &
		\mathrm{Lie}(G^D) \arrow[r, "", rightarrow] &
		0
	.\end{tikzcd}
	\end{equation*}
	%tk: prove it.
\end{rem}


\noindent
We recall that elements of $\mathsf{Crys}(X/S)$ are defined by
locally nilpotent sheaf of ideals.
In what follows, though, we want to evaluate crystals on
surjections of $p$-adically complete rings, whose kernel is endowed with divided powers.
\begin{defn}[]\label{defn:NotNilpotentEvaluation}
	tk: this definition seems to make sense, but I wrote it myself.
	Let $A \twoheadrightarrow A_0$ be a surjective homomorphism
	of $p$-adically complete and separated $\mathbb{Z}_{p}$-algebras
	whose kernel is equipped with divided powers, compatible with those on $p \mathbb{Z}_{p}$.
	Take $G \in \mathsf{BT}(A_0)$.
	Denote by $G_n$ the restriction of $G$ to $A_0/p^nA_0$
	and by $A_n \coloneqq A/p^nA$.
	Then we define
	\begin{equation*}
		\mathbb{D}^*(G)(A) \coloneqq
		\varprojlim_{n \in \mathbb{N}} \mathbb{D}^*(G_n)(A_n)
	.\end{equation*}
\end{defn}


\begin{rem}[]
	Notice that, in the above definition, $A_n \twoheadrightarrow A_0/p^nA_0$
	has kernel equipped with divided powers.
	In fact it is the projection of the kernel of $A \twoheadrightarrow A_0$, over
	$p^nA$, an ideal with divided powers compatible with those on the kernel.
	And analogously, since the divided powers on the kernel are compatible with those 
	on $(p)$, we see that the kernel of $A_n \twoheadrightarrow A_0/p^mA_0$
	has divided powers for all $m \leq n$.
	Now, combining this remark with \cref{rem:DGValuationRestriction},
	we see that the above definition could have easily been swapped out with
	\begin{equation*}
		\mathbb{D}^*(G)(A) \coloneqq
		\varprojlim_{n \geq m} \mathbb{D}^*(G_m)(A_n)
	\end{equation*}
	by cofinality of the family $n \geq m$ in $\mathbb{N}$.
\end{rem}



\begin{rem}[]
	tk: add a remark on the technical lemmas before going into the statements.
	The idea is that we need to iteratively lift the additional structure we are defining
	on our modules (so Frobenius and filtration) to extend the classical theory over a field.
\end{rem}


\begin{lem}[{\cite[Lemma A.2]{Kisin}}]\label{lem:A2Kisin}
	Let $A \to A_0$ be a surjection of $p$-adically complete
	and separated local $\mathbb{Z}_{p}$-algebras with residue field
	$k$ and kernel $\mathrm{Fil}^1 A$ equipped with divided powers.
	Suppose moreover that
\begin{enumerate}
	\item $A$ is $p$-torsion-free and it is equipped with an endomorphism
		$\varphi\colon A \to A$ lifting the Frobenius on $A/pA$;
		
	\item the following map, induced on the pullback, is surjective
		\begin{equation*}
		\begin{tikzcd}
			\mathrm{id}_{ A } \otimes \varphi/p \colon
			\varphi^*(\mathrm{Fil}^1 A)
			\arrow[r, "", twoheadrightarrow] &
			A
		.\end{tikzcd}
		\end{equation*}
\end{enumerate}
	If $G \in \mathsf{BT}(A_0)$ we write $\mathrm{Fil}^1 \mathbb{D}^*(G)(A) \subset \mathbb{D}^*(G)(A)$
	for the preimage of $\left( \mathrm{Lie}(G) \right)^D$ inside $\mathbb{D}^*(G)(A_0)$.
	Then the restriction of $\varphi\colon \mathbb{D}^*(G)(A) \to \mathbb{D}^*(G)(A)$
	to $\mathrm{Fil}^1 \mathbb{D}^*(G)(A)$ is divisible by $p$ and the following induced map
	is a surjection
	\begin{equation*}
	\begin{tikzcd}
		1 \otimes \varphi/p \colon
		\varphi^*\mathrm{Fil}^1 \mathbb{D}^*(G)(A)
		\arrow[r, "", twoheadrightarrow] &
		\mathbb{D}^*(G)(A)
	.\end{tikzcd}
	\end{equation*}
\end{lem} 


\begin{rem}[]\label{rem:A2Kisin}
	Even though we will not prove the theorem, we feel it is important to remark on
	the most important points used in the proof.
\begin{enumerate}
	\item Given an ideal with divided powers $I \triangleleft A$, then $\varphi(I) \subset pA$.
		In fact 
		\begin{equation*}
			\varphi(x) = x^p = \gamma_p(x) \cdot p! \in pA
		.\end{equation*}

	\item tk: think about proving it.
		Let $\widetilde{G}$ be a lift of $G$ to $A$, then we have
		\begin{equation*}
			\mathrm{Fil}^1(\mathbb{D}^*(\widetilde{G})(A) =
			( \mathrm{Lie} \widetilde{G} )^D +
			\mathrm{Fil}^1A \cdot \mathbb{D}^*(\widetilde{G})(A)
		.\end{equation*}
		
	\item Let $H \in \mathsf{BT}(\mathrm{Spec}(W(k))$. Denote by $H_0$ the restriction
		of $H$ to $k$ and by $V$ the Verschiebung morphism.
		Then $W(k) \twoheadrightarrow k$ satisfies the hypothesis of our lemma,
		and the kernel $\mathrm{Fil}^1(W(k)) = (p)$ is equipped with divided powers.
		Then, using the theory of Dieudonné modules, one can see that
		\begin{equation*}
			( \mathrm{Lie} H)^D + p \mathbb{D}^*(H)(W(k)) =
			V \mathbb{D}^{*}(H)(W(k))
		,\end{equation*}
		i.e. that
		$V \mathbb{D}^*(H)(W(k)) = \mathrm{Fil}^1 \mathbb{D}^*(H)(W(k))$.
\end{enumerate}
\end{rem}


\begin{defn}[Special ring]
	We define a {\em special ring} to be a $p$-adically complete, separated, $p$-torsion-free,
	local $\mathbb{Z}_{p}$-algebra $A$ equipped with an endomorphism $\varphi$
	lifting the Frobenius on $A/pA$.
	Moreover we call {\em map of special rings} a morphism
	of $\mathbb{Z}_{p}$-algebras compatible with $\varphi$.
\end{defn}


\begin{defn}[]
	Let $A$ be a special ring. We define the category
	$\mathsf{C}_A$ whose objects are finite, free $A$-modules
	$M$ equipped with a semilinear Frobenius map $\varphi\colon M \to M$
	and an $A$-submodule $M_1 \subset M$ such that $\varphi(M_1) \subset pM$
	and such that the map
	\begin{equation*}
	\begin{tikzcd}[row sep = 0ex
		,/tikz/column 1/.append style={anchor=base east}
		,/tikz/column 2/.append style={anchor=base west}]
		\mathrm{id}_{ A }\otimes\varphi/p\colon 
		\varphi^*(M_1) \arrow[r, "", twoheadrightarrow] &
		M
	\end{tikzcd}
	\end{equation*} 
	is surjective.
	Its morphisms are morphisms of $A$-modules compatible with
	the Frobenius and submodules.
\end{defn}


\begin{rem}[]
	Notice that \cref{lem:A2Kisin} allows to endow,
	for $G \in \mathsf{BT}(A_0)$ and $A \twoheadrightarrow A_0$ as in the hypothesis,
	the module $\mathbb{D}^*(G)(A)$ with the structure of an object in $\mathsf{C}_A$.
\end{rem}


\begin{defn}[]
	Consider a map of special rings $A \to B$ and $M \in \mathsf{C}_A$.
	Then $M \otimes_A B \in \mathsf{C}_B$, when equipped with
	the induced Frobenius %(tk: what is it?) 
	and setting
	$\left( M \otimes_A B \right)_1$ to be the image of
	$M_1 \otimes_A B$ in $M \otimes_A B$.
\end{defn}


\begin{lem}[{\cite[Lemma A.4]{Kisin}}]\label{lem:A4Kisin}
	Let $h\colon A \to B$ be a surjection of special rings with kernel $J$.
	Suppose that, for all $i \geq 1$, $\varphi^i(J) \subset p^{i + j_i} J$,
	where $\left\{ j_i \right\}_{i \geq 1}$ is a sequence of integers
	such that $\lim_{i \to \infty} j_i = \infty$.
	Consider $M, M' \in \mathsf{C}_A$ and
	\begin{equation*}
	\begin{tikzcd}[row sep = 0ex
		,/tikz/column 1/.append style={anchor=base east}
		,/tikz/column 2/.append style={anchor=base west}]
		\theta_B\colon M \otimes_A B 
		\arrow[r, "\sim", rightarrow] &
		M' \otimes_A B
	\end{tikzcd}
	\end{equation*} 
	an isomorphism in $\mathsf{C}_B$.
	Then there exists a unique isomorphism of $A$-modules
	$\theta_A\colon M \to M'$ lifting $\theta_B$ and compatible with $\varphi$.
\end{lem} 


\begin{rem}[]
	Again we will not prove the theorem, though try to give an idea of
	how the proof works.
	In this case we deform iteratively the map
	\begin{equation*}
	\begin{tikzcd}[column sep=3.0em]
		\varphi^*(M_1) \arrow[r, "\varphi^*(\left.\theta_0\right|_{ M_1 })", rightarrow] &
		\varphi^*(M_1') \arrow[r, "1 \otimes \varphi/p", rightarrow] &
		M'
	\end{tikzcd}
	\end{equation*}
	obtaining a succession of maps $\theta_i$ whose successive difference
	has image lying in $p^{j_i}M'$.
	The fact that $j_i \to \infty$ allows to obtain a well defined limit.
\end{rem}


\noindent
In order to apply the above results to the proof of the main theorem of the section
we still need some remarks and definitions.


\begin{ntt}[]
	Consider $W$ as a divided power ring, endowing the maximal ideal $(p)$
	with the divided powers structure defined in \cref{PDex:DVR} of \cref{PDexamples}.
	Consider $W[u]$ as a $W$-algebra, and take $\mathcal{D}_{W[u]}(E(u))$
	its divided powers envelope, as outlined in \cref{thm:PDEnvelope}.
	Let's notice that, as outlined in \cref{rem:PDEnvelope},
	the ideal $\overline{J}$ of $\mathcal{D}_{W[u]}(E(u))$ contains $p$.
	We denote by $S$ the $p$-adic completion of $\mathcal{D}_{W[u]}(E(u))$.
	The ring $S$ is equipped with an endomorphism $\varphi$ given by the Frobenius
	on $W$ and by $\varphi(u) \coloneqq u^p$.
	We denote by $\mathrm{Fil}^1 S \subset S$ the closure of the ideal
	generated by $E(u)$ and its divided powers.
	Since $\mathrm{Fil}^1(S)$ is equipped with divided powers, 
	as seen in \cref{rem:A2Kisin} we have $\varphi( \mathrm{Fil}^1 S) \subset p S$.
	Then we set $\varphi_1 \coloneqq \varphi/p$.
\end{ntt}


\begin{rem}[]
	tk: you need to argue for the extension of $G_e$ to $S$,
	since the kernel seems to have topologically nilpotent divided powers,
	but not nilpotnet ones. So follow De joung.
\end{rem}


\begin{rem}[]\label{rem:j_iTheoremA6}
	We will apply \cref{lem:A4Kisin} in the situation where $J$ is equipped
	with divided powers structure and there exists a finite set of 
	elements $x_1, \ldots, x_n \in J$ such that $J$, in the $p$-adic topology, 
	is topologically generated by the $x_i$ and their divided powers.
	We will have, moreover, that $\varphi(x_i) = x_1^p$.
	Then the integers $j_i$ above are $v_p \left( (p-1)! \right) - 1$.
	In fact, since $J$ is equipped with divided powers, we have
	\begin{equation*}
		\varphi^i(x_j) =
		x_j^{p^i} =
		x_j \cdot \gamma_{p^i-1}(x_j) \cdot (p^i-1)!
	.\end{equation*}
	Since both $x_j$ and $\gamma_{p^i-1}(x_j)$ are in $J$,
	we see that $\varphi^i(x_j) \in p^{\nu_p((p^i-1)!)}J$.
\end{rem}


\begin{defn}[]
	We denote by $\mathsf{BT}^{\varphi}_{/S}$ the category
	whose objects are finite free $S$-modules $M$
	equipped with an $S$-submodule $\mathrm{Fil}^1 M$
	and a $\varphi$-semilinear map $\varphi_1\colon  \mathrm{Fil}^1 M \to M$
	such that
\begin{enumerate}
	\item $\mathrm{Fil}^1 S \cdot M \subset \mathrm{Fil}^1 M$
		and the quotient $M/\mathrm{Fil}^1 M$ is a free $\mathcal{O}_{ K }$-module;
	\item the map $1 \otimes \varphi_1\colon 
		\varphi^* \left( \mathrm{Fil}^1 M \right) \to M$
		is surjective.
\end{enumerate}
\end{defn}


\begin{rem}[]\label{invertibilityphi1E(u))}
	Notice that any $M \in \mathsf{BT}^{\varphi}_{/S}$ is equipped with a Frobenius
	semilinear map $\varphi\colon M \to M$ defined by
	\begin{equation*}
		\varphi (x) \coloneqq \varphi_1 \left( E(u) \right)^{-1} \varphi_1 (E(u)x)
	.\end{equation*}
	(tk: this is my interpretation, I'm not sure it is correct.)
	In fact $\varphi_1(E(u))$ is invertible in $S$.
	To show this we recall that, by definition, $\varphi_1$ is
	linear on $W[u]$, and that $W[u]$ embeds in $S$.
	Then it is enough to show that the element
	\begin{equation*}
		\varphi_1(E(u)) =
		\frac{ u^{ep} }{ p } + \varphi_1(a_{e-1}) u^{(e-1)p} +
		\cdots + \varphi_1(a_0)
	\end{equation*}
	is invertible in $W[u]$.
	Recall that,
	for all $L/K_0$ finite and totally ramified,
	the following morphism is surjective 
	\begin{equation*}
	\begin{tikzcd}[row sep = 0ex
		,/tikz/column 1/.append style={anchor=base east}
		,/tikz/column 2/.append style={anchor=base west}]
		W[u] \arrow[r, "", twoheadrightarrow] &
		O_{L} \\
		u \arrow[r, "", mapsto] & 
		\pi_L
	.\end{tikzcd}
	\end{equation*} 
	This is because, as an $O_{K_0}$-module, $O_L$ is generated by $\pi_L$, its uniformizer,
	and every element of $W$ can be seen as an element in $O_{K_0}$.
	Let's now take $[ L : K_0 ] = e \cdot p$, where $e = [ K : K_0 ]$.
	Then the image of $\varphi_1(E(u))$ via the above morphism is given by a sum of
	elements with strictly positive valuation 
	(we have $\nu_L(\varphi_1(a_i)) > 0$ for all $i < e$) and $\pi_L^{ep}/p$.
	Since the absolute ramification of $L$ is exactly $ep$, we see that this
	last term is invertible, hence the whole sum is.
	By surjectivity $\varphi_1(E(u))$ is invertible also in $W[u]$,
	hence in $S$.
\end{rem}


\begin{prop}[{\cite[Proposition A.6]{Kisin}}]
	There is an exact contravariant functor
	\begin{equation*}
	\begin{tikzcd}[row sep = 0ex
		,/tikz/column 1/.append style={anchor=base east}
		,/tikz/column 2/.append style={anchor=base west}]
		\mathsf{BT}(\mathcal{O}_{ K }) \arrow[r, "", rightarrow] &
		\mathsf{BT}^{\varphi}_{/S} \\
		G \arrow[r, "", mapsto] & 
		\mathbb{D}^*(G)(S)
	.\end{tikzcd}
	\end{equation*} 
	If $p > 2$ this functor is an anti equivalence,
	whereas if $p = 2$ it induces an anti-equivalence of the 
	corresponding isogeny categories.
\end{prop}
\begin{proof}\leavevmode\vspace{-.2\baselineskip}
\begin{enumerate}
\item We start by showing that this functor actually takes values in $\mathsf{BT}^{\varphi}_{/S}$.
	To do so we need to show that $M(G) \coloneqq \mathbb{D}^{*}(G)(S)$
	has a structure of object in $\mathsf{BT}^{\varphi}_{/S}$.
	If we can show that $S$ and $O_K$ satisfy the hypothesis of \cref{lem:A2Kisin}
	we can conclude by invoking that result.
	Notice that the map $S \to O_K$ is induced by $W[u] \twoheadrightarrow O_K$
	which we already remarked being a surjection.
	Then, by construction, we see that $S \twoheadrightarrow O_K$
	is surjective too.
	Then it is clear that both $S$ and $O_K$ are 
	$p$-adically complete, separated, local $\mathbb{Z}_{p}$-algebras.
	Moreover, by definition, $\ker (W[u] \to O_K) = (E(u))$.
	Then, by construction of $S$ and completeness of $O_K$,
	we see that $\mathrm{Fil}^1 S$, the kernel of $S \twoheadrightarrow O_K$, is generated
	by $E(u)$ and its divided powers, i.e. it is equipped with divided powers.
	Moreover, by construction, it is clear that $S$ is $p$-torsion-free
	and it is also equipped with an endomorphism lifting the Frobenius on $S/pS$.

	Finally we are left to prove that the map
	\begin{equation*}
	\begin{tikzcd}[row sep = 0ex
		,/tikz/column 1/.append style={anchor=base east}
		,/tikz/column 2/.append style={anchor=base west}]
		\mathrm{id}_{ S } \otimes \varphi/p \colon 
		\varphi^*(\mathrm{Fil}^1 S) \arrow[r, "", twoheadrightarrow] &
		S
	\end{tikzcd}
	\end{equation*} 
	is surjective.
	Here, thanks to the argument in \cref{invertibilityphi1E(u))}, we easily conclude.
	In fact we see that, for all $s \in S$,
	\begin{equation*}
	\begin{tikzcd}[row sep = 0ex
		,/tikz/column 1/.append style={anchor=base east}
		,/tikz/column 2/.append style={anchor=base west}]
		s \cdot \left( \varphi_1(E(u)) \right)^{-1} \otimes E(u) 
		\arrow[r, "", mapsto] & 
		s
	.\end{tikzcd}
	\end{equation*} 


\item The converse is definitely more tricky.
	It will require to use classical Dieudonné theory to construct a
	Barsotti-Tate group over $k$ associated to a module $M \in \mathsf{BT}^{\varphi}_{/S}$
	and then to iteratively lift it from $k = O_K/\pi O_K$ to $O_K$.
	This difficulty is due to the fact that, in general, the maximal
	ideal of $O_K$, i.e. the kernel of $O_K \twoheadrightarrow k$, 
	does not admit divided powers.
	Then the only hope to lift the classical construction is to
	procede iteratively, reducing the above projection in a sequence of
	smaller projections, whose kernels all have divided powers structures.
	(tk: might be a good idea to outline the two main steps now)

	Let's start by introducing the necessary notation:
	let $i$ vary in $1, \ldots, e$ and set $R_i \coloneqq W[u]/u^i$.
	Clearly $R_i$ is equipped with a Frobenius endomorphism $\varphi$
	given by the usual one on $W$ and $u \mapsto u^p$ on the indeterminate.
	Then $O_K/ (\pi^i)$ is an $R_i$-algebra by $u \mapsto \pi$,
	seeing $W = O_{K_0} \subset O_K$.
	As can be seen writing the elements of $O_K$ in Teichmüller expansion
	the map $R_i \twoheadrightarrow O_K/ (\pi^i)$ is a surjection
	with kernel $pR_i$, which is equipped with divided powers by \cref{lem:PDFlatExtension}.
	Then, given a $p$-divisible group $G_i \in \mathsf{BT}(O_K/\pi^iO_K)$,
	we can evaluate $\mathbb{D}^*(G_i)(R_i)$ thanks to \cref{defn:NotNilpotentEvaluation}.
	Following the notation of \cref{lem:A2Kisin}
	we denote by $\mathrm{Fil}^1\mathbb{D}^*(G_i)(R_i)$
	the preimage of $\left( \mathrm{Lie}G_i \right)^D \subset \mathbb{D}^*(G_i)(O_K/\pi^iO_K)$
	in $\mathbb{D}^*(G_i)(R_i)$.
	Notice that universal property of divided powers envelope
	induces a $\varphi$-compatible map $S \to R_i$ as the unique map associated to
	\begin{equation*}
	\begin{tikzcd}[row sep = 0ex
		,/tikz/column 1/.append style={anchor=base east}
		,/tikz/column 2/.append style={anchor=base west}]
		W[u] \arrow[r, "", rightarrow] &
		R_i \\
		u \arrow[r, "", mapsto] & u \\
		u^{ej} / j!  \arrow[r, "", mapsto] & 0
	.\end{tikzcd}
	\end{equation*} 
	This map is compatible with $\varphi$, hence by uniqueness also
	the induced map on $S$ is.
	We can then define the module $M_i \coloneqq M \otimes_S R_i$,
	which is a restriction of scalars of our original one and
	it is also equipped with the diagonal action of $\varphi$.
	We set $\mathrm{Fil}^1 M_i \subset M_i$
	to be the image of $\mathrm{Fil}^1 M$ in $M_i$, which is a submodule
	by surjectivity of $S \twoheadrightarrow R_i$.
	Notice that the above corresponds to defining $\mathrm{Fil}^1 M_i \coloneqq
	\mathrm{Fil}^1 M \otimes_S R_i$.
	Then, by right exactness of tensor product, 
	\begin{equation*}
	\begin{tikzcd}[row sep = 0ex
		,/tikz/column 1/.append style={anchor=base east}
		,/tikz/column 2/.append style={anchor=base west}]
		\mathrm{id}_{ S } \otimes \varphi_1
		\colon 
		\varphi^*(\mathrm{Fil}^1 M)\arrow[r, "", rightarrow] &
		M
	\end{tikzcd}
	\end{equation*} 
	induces an surjective map $\varphi^*(\mathrm{Fil}^1 M_i \twoheadrightarrow M_i$
	for all $i$.
	In other words we have just seen that, for $i \in [1,e]$,
	$M_i \in \mathsf{C}_{R_i}$.

	With all this in mind, the idea of the proof, is to first construct from
	$M_1$ a barsotti-Tate group $G_1$ on $k = O_K/\pi O_k$.
	Then, using the previous lemmas and Grothendieck-Messing deformation theory,
	lift it, iteratively, to $G_i \in \mathsf{BT}(O_K/\pi^iO_K)$ until $i = e$.
	Finally, using completeness of $O_K$, using another inductive process of lifting
	to define $G \in \mathsf{BT}(O_K)$.
	The idea is that at each step we obtain a group with an isomorphism
	\begin{equation*}
	\begin{tikzcd}[row sep = 0ex
		,/tikz/column 1/.append style={anchor=base east}
		,/tikz/column 2/.append style={anchor=base west}]
		\mathbb{D}^*(G_i)(R_i) \arrow[r, "\sim", rightarrow] &
		M_i
	\end{tikzcd}
	\end{equation*} 
	compatible with both filtration and $\varphi$.

	As already hinted above, the first part of the construction
	is mainly due to the classical Dieudonné theory.
	More explicitly, let's denote by $F\colon M_1 \to M_1$ the map induced by
	$\varphi\colon M \to M$.
	Then we see that both sides of the surjective map
	$\varphi^* ( \mathrm{Fil}^1 M_1) \to M_1$ are free $W$-modules of the same (finite) rank,
	which can be seen by base changing to $k$, which means that the map is an isomorphism.
	Now we consider the composition
	\begin{equation*}
	\begin{tikzcd}
		M_1 \arrow[r, "\sim", rightarrow] &
		\varphi^* (\mathrm{Fil}^1 M) \arrow[r, "", rightarrow] &
		\varphi^* (M_1) \arrow[r, "\sim", rightarrow] &
		M_1
	,\end{tikzcd}
	\end{equation*}
	where the first arrow is just the inverse of the above iso, the second map
	is induced by the inclusion $\mathrm{Fil}^1 M \hookrightarrow M$ and the
	third one is given by $a \otimes m \mapsto \varphi^{-1}(a)m$.
	This composition gives a $\varphi^{-1}$-semilinear map $V\colon M_1 \to M_1$
	such that $FV = VF = p$.
	Let's denote by $G_1$ the Barsotti-Tate group on $k$ associated to this
	Dieudonné module.
	Classical theory grants that the isomorphism
	\begin{equation*}
	\begin{tikzcd}[row sep = 0ex
		,/tikz/column 1/.append style={anchor=base east}
		,/tikz/column 2/.append style={anchor=base west}]
		\mathbb{D}^*(G_1)(W) \arrow[r, "\sim", rightarrow] &
		M_1
	\end{tikzcd}
	\end{equation*} 
	is compatible with Frobenius.
	Moreover, thanks to \cref{rem:A2Kisin}, we see that $\mathrm{Fil}^1 \mathbb{D}^*(G_1)(W)$
	is identified with $V \mathbb{D}^*(G_1)(W)$, which allows to say that the isomorphism
	is compatible also with Filtration.

	We now have to start, iteratively, lift this construction.
	We assume, for $i \in [2,e]$, that we have an isomorphism
	\begin{equation}\label{eqn:PartialIsoCRi}
	\begin{tikzcd}
	\mathbb{D}^*(G_{i-1})(R_{i-1}) 
	\arrow[r, "\sim", rightarrow] &
	M_{i-1}
	\end{tikzcd}
	\end{equation}
	compatible with Frobenius and filtrations, i.e. a morphism in $\mathsf{C}_{R_{i-1}}$.
	We can notice, moreover, that the kernel of $R_i \twoheadrightarrow O_K/ (\pi^{i-1})$
	is $(p, u^{i-1})$ which is still equipped with divided powers.
	In fact we simply put $\gamma_1(u^{i-1}) = u^{i-1}$ and, for all $n \geq 2$,
	$\gamma_n(u^{i-1}) = 0$, since $u^{n(i-1)} \in (u^i)$.
	Then we are done invoking \cref{lem:PDExt2}.
	Again, this means that we can compute $\mathbb{D}^*(G_{i-1})(R_i)$.
	We have already seen that $M_i \in \mathsf{C}_{R_i}$, moreover \cref{lem:A2Kisin}
	applied to the surjection $R_i \twoheadrightarrow O_k/ (\pi^{i-1})$ implies that
	also $\mathbb{D}^*(G_{i-1})(R_i)$ is in $\mathsf{C}_{R_i}$.
	Recalling that the isomorphism in \cref{eqn:PartialIsoCRi}
	is a morphism in $\mathsf{C}_{R_{i-1}}$ and \cref{rem:j_iTheoremA6}, 
	we can apply \cref{lem:A4Kisin} to the surjection $R_i \twoheadrightarrow R_{i-1}$
	and obtain a lift to an isomorphism
	\begin{equation*}
	\begin{tikzcd}
	\mathbb{D}^*(G_{i-1})(R_{i}) 
	\arrow[r, "\sim", rightarrow] &
	M_{i}
	\end{tikzcd}
	\end{equation*}
	compatible with Frobenius.
	Finally, since the kernel of $R_i \twoheadrightarrow R_{i-1}$ is nilpotent,
	we can invoke \cref{thm:EquivCatBTS} and obtain that
	there is a unique $G_i \in \mathsf{BT}(O_K/\pi^iO_K)$
	lifting $G_{i-1}$ and such that $(\mathrm{Lie}G_i)^D \subset \mathbb{D}^*(G_{i-1})(O_K/\pi^i O_K)$
	is equal to the image of $\mathrm{Fil}^i M_i$ under the composite
	\begin{equation*}
	\begin{tikzcd}
		\mathrm{Fil}^1 M_i \subset M_i
		\arrow[r, "\sim", rightarrow] &
		\mathbb{D}^*(G_{i-1})(R_i)
		\arrow[r, "", rightarrow] &
		\mathbb{D}^*(G_{i-1})(O_K/\pi^iO_K)
	.\end{tikzcd}
	\end{equation*}
	Then, by construction, we obtain that the isomorphism
	$\mathbb{D}^*(G_i)(R_i) \simeq M_i$, where we recall the implicit use of 
	\cref{rem:DGValuationRestriction}, is compatible also with filtrations,
	i.e. is in $\mathsf{C}_{R_i}$.
	This concludes the first step of induction.
\end{enumerate}
\end{proof}


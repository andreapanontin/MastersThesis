\section{Classification of \texorpdfstring{$p$}{p}-divisible groups
	over \texorpdfstring{$O_{ K }$}{the ring of integers of K}}
In this section we will review the results seen in \cite[Appendix A]{Kisin},
which generalize the results of Dieudonné of classification of Barsotti-Tate groups.
In fact we will introduce some technical lemmas and use them to
generalize the classification of Barsotti-Tate groups over a perfect field
of characteristic $p$
to that of Barsotti-Tate groups over the ring of integers of a local field.



\subsection{Witt vectors}
In order to work over a finite extension $K/\mathbb{Q}_p$ it is convenient to
have some familiarity with the formalism of Witt vectors, so we will dedicate some 
space to recall the basic definitions and results.
As a typographical convention, in the following sections we will
denote vectors using a boldface character.


\begin{rem}[Motivation]
	Consider $K/\mathbb{Q}_p$ a finite and unramified extension.
	Denote by $k \coloneqq O_K/pO_K$ the residue field of $K$ and by $q \coloneqq \left| k \right|$
	its cardinality.
	Then there is multiplicative map $[\ \cdot \ ]\colon k \to O_K$,
	called the Teichmüller lifting, with image $\mu_{q-1}(O_K)$
	the $(q-1)$-st root of unity of $O_K$.
	In particular it is well known that any element $a \in O_K$
	can be uniquely written in a Teichmüller expansion
	\begin{equation*}
		a = \sum_{n \in \mathbb{N}} [c_n] p^n
	,\end{equation*}
	with $c_n \in k$ for all $n \in \mathbb{N}$.
	The theory of Witt vectors allows the explicit computation of sums and
	products between Teichmüller expansions, using only
	algebraic operations of the sequences of elements of $k$.
	Clearly its interest doesn't stop there, so we will highlight some 
	other properties of the construction.

	Moreover we wish to remark that the following constructions
	can be carried out for any ring $A$, though the most interesting
	results for us will be in case $A$ is a perfect $\mathbb{F}_p$-algebra,
	i.e. an $\mathbb{F}_p$ algebra on which the map $x \mapsto x^p$
	is an automorphism.
	The results we will introduce in this section
	revolve around this assumption, but we will highlight when
	it is not strictly necessary.
\end{rem}


\begin{defn}[Witt polynomials]%tk: widow
	We define a family of polynomials $\left\{ w_n(\mathbf{x}) \right\}_{n \in \mathbb{N}} \subset
	\mathbb{Z}[\mathbf{x}] = \mathbb{Z}[x_i]_{i \in \mathbb{N}}$ by
	\begin{align*}
		w_0(\mathbf{x}) \coloneqq w_0(x_0) &\coloneqq x_0 \\
		w_1(\mathbf{x}) \coloneqq w_1(x_0, x_1) &\coloneqq x_0^p + px_1 \\
				&\ \ \ \ \! \vdots \\
		w_n(\mathbf{x}) \coloneqq w_n(x_0, \ldots, x_n) &\coloneqq
		\sum_{ i=0 }^{ n } p^i x_i^{p^{n-i}} =
		x_0^{p^n} + px_1^{p^{n-1}} + \cdots + p^{n-1}x_{n-1}^p + p^n x_n
	.\end{align*}
\end{defn}


\begin{lem}[{\cite[Chapter II, \S6, theorem 6]{Serre}}]\label{thm:WittSumProdPolys}
	For every $\phi \in \mathbb{Z}[x,y]$ there exists a unique sequence
	$\left\{ \varphi_n \right\}_{n \in \mathbb{N}} \subset \mathbb{Z}[\mathbf{x}, \mathbf{y}]$,
	where $\mathbf{x} = \left\{ x_n \right\}_{n \in \mathbb{N}}$
	$\mathbf{y} = \left\{ y_n \right\}_{n \in \mathbb{N}}$, such that,
	for all $n \in \mathbb{N}$,
	\begin{equation*}
		w_n(\varphi_0, \ldots, \varphi_n) =
		\phi \left( w_n(x_0, \ldots, x_n), w_n(y_0, \ldots, y_n) \right)
	.\end{equation*}
\end{lem} 


\begin{ntt}[]\label{not:WittSumProd}
	Denote by $\left\{ S_n \right\}_{n \in \mathbb{N}}$ and
	$\left\{ P_n \right\}_{n \in \mathbb{N}}$ the polynomials associated 
	by \cref{thm:WittSumProdPolys} to 
	$\phi(x,y) = x + y$ and $\phi(x,y) = x \cdot y$ respectively.
	Then we define, for $\mathbf{a},\mathbf{b} \in A^{\mathbb{N}}$ the following
	composition laws:
	\begin{align*}
		\mathbf{a} + \mathbf{b} &\coloneqq
		\left( S_n(\mathbf{a},\mathbf{b}) \right)_{n \in \mathbb{N}}\\
		\mathbf{a} \cdot \mathbf{b} &\coloneqq 
		\left( P_n(\mathbf{a},\mathbf{b}) \right)_{n \in \mathbb{N}}
	.\end{align*}
\end{ntt}


\begin{thm}[{\cite[Chapter II, \S6, theorem 7]{Serre}}]
	The laws of composition on $A^{\mathbb{N}}$ defined in \cref{not:WittSumProd}
	make it into a commutative unitary ring.
\end{thm}


\begin{defn}[Ring of Witt vectors]
	We define the {\em ring of Witt vectors} with coefficients in $A$,
	denoted by $W(A)$, to be the commutative unitary ring $A^{\mathbb{N}}$
	endowed with the composition laws defined in \cref{not:WittSumProd}.
\end{defn}


\begin{rem}[]
	By definition, the map
	\begin{equation*}
	\begin{tikzcd}[row sep = 0ex
		,/tikz/column 1/.append style={anchor=base east}
		,/tikz/column 2/.append style={anchor=base west}]
		W\colon W(A) \arrow[r, "", rightarrow] &
		A^{\mathbb{N}} \\
		\mathbf{a} \arrow[r, "", mapsto] & 
		\left( w_n(\mathbf{a}) \right)_{n \in \mathbb{N}}
	\end{tikzcd}
	\end{equation*} 
	is a ring homomorphism.
	Moreover it is easy to check that it is 
	a monomorphism if $p$ is not a zero divisor and
	an isomorphism as soon as $p$ is invertible. 
\end{rem}


\begin{defn}[Frobenius and Verschiebung]
	Since $A$ is of characteristic $p$ we can define on $W(A)$ the following two
	maps
	\begin{equation*}
	\begin{tikzcd}[row sep = 0ex
		,/tikz/column 1/.append style={anchor=base east}
		,/tikz/column 2/.append style={anchor=base west}]
		V\colon W(A) \arrow[r, "", rightarrow] &
		W(A) \\
		(a_0, a_1, \ldots) \arrow[r, "", mapsto] & 
		(0, a_0, a_1, \ldots)
	\end{tikzcd}
	\qquad \text{ and } \qquad
	\begin{tikzcd}[row sep = 0ex
		,/tikz/column 1/.append style={anchor=base east}
		,/tikz/column 2/.append style={anchor=base west}]
		F\colon W(A) \arrow[r, "", rightarrow] &
		W(A) \\
		(a_0, a_1, \ldots) \arrow[r, "", mapsto] & 
		(a_0^p, a_1^p, \ldots)
	.\end{tikzcd}
	\end{equation*} 
	It can be easily shown that the first map, called {\em Verschiebung}
	(German for {\em shift}), is additive, whereas the second, called {\em Frobenius},
	is a ring homomorphism.
\end{defn}


\begin{defn}[(Strict) $p$-ring]
	A ring $B$ is called {\em $p$-ring} iff it is separated and complete
	for the topology induced by a decreasing collection of ideals
	$\left\{ \mathfrak{b}_i \right\}_{i \geq 1}$ such that 
	$\mathfrak{b}_n \mathfrak{b}_m \subset \mathfrak{b}_{n+m}$
	for all $n,m \geq 1$
	and $B/\mathfrak{b}_1$ is a perfect $\mathbb{F}_p$-algebra
	(hence $p \in \mathfrak{b}_1$).

	We say that $B$ is a {\em strict $p$-ring} iff it is a $p$-ring
	and $\mathfrak{b}_i = p^iB$ for all $i \geq 1$
	and $p\colon B \to B$ is an injective map.
\end{defn}


\begin{rem}[]
	Notice that a strict $p$-ring is a $p$-adically separated and complete
	ring such that $B/pB$ is a perfect $\mathbb{F}_p$-algebra.
\end{rem}


\begin{rem}[]
	Since $A$ is of characteristic $p$ we have the identities
	$VF = FV = p$.
	Moreover, in case $A$ is perfect, we see that $p^nW(A) = V^nW(A)$,
	since $p \cdot \left( a_0, a_1, \ldots \right) =
	(0, a_0^p, a_1^p, \ldots)$.
	This means that the $p$-adic topology of $W(A)$ corresponds
	to its natural product topology.
	This means that $W(A) \simeq \varprojlim_{n \in \mathbb{N}} W(A)/ (p^n)$.
	Then, since $W(A)/ (p) \simeq A$, that $W(A)$ is a strict $p$-ring.
\end{rem}


\begin{lem}[{\cite[Lemma 4.2.2]{Brinon}}]\label{pRingSection}
	Let $B$ be a $p$-ring.
	There is a unique set theoretic section
	$r_B\colon B/\mathfrak{b}_1 \to B$
	to the reduction map such that
	$r_B(x^p) = r_B(x)^p$
	for all $x \in B/\mathfrak{b}_1$.
	Moreover $r_B$ is multiplicative and $r_B(1) = 1$.
\end{lem} 


\begin{defn}[Teichmüller lift]
	For a ring $A$, we can define the section to the reduction map
	\begin{equation*}
	\begin{tikzcd}[row sep = 0ex
		,/tikz/column 1/.append style={anchor=base east}
		,/tikz/column 2/.append style={anchor=base west}]
		[\ \cdot \ ]\colon A \arrow[r, "", rightarrow] &
		W(A) \\
		a \arrow[r, "", mapsto] &
		[a] \coloneqq \left( a, 0, 0, \ldots \right)
	,\end{tikzcd}
	\end{equation*} 
	where $[a]$ is called the {\em Teichmüller lift} of $a$.
\end{defn}


\begin{rem}[]\leavevmode\vspace{-.2\baselineskip}\label{TeichmullerExpansionWitt}
\begin{enumerate}
\item Reducing to an appropriate universal case it is easy to prove that
	\begin{equation*}
		(x_0, x_1, x_2, \ldots) \cdot (y_0, 0, 0 \ldots) =
		(x_0y_0, x_1^py_0, x_2^{p^2}y_0, \ldots)
	\end{equation*}
	in $W(A)$.
	Hence $[\ \cdot\ ]$ is a multiplicative map
	and satisfies conditions of \cref{pRingSection}.

\item If $B$ is a strict $p$-ring endowed with the $p$-adic topology,
	each $b \in B$ can be written as
	\begin{equation*}
		b = \sum_{n \in \mathbb{N} }^{  } r_B(b_n) p^n
	\end{equation*}
	with $b_n \in B/\mathfrak{b}_1 = B/pB$.
	Let's recall that $B$ is complete and separated with respect to the $p$-adic
	topology.
	As a consequence the above series converges and the expansion is unique.

\item Let $A$ be a perfect $\mathbb{F}_p$-algebra, then any element $\mathbf{x} \in W(A)$
	can be uniquely written as
	\begin{equation*}
		\mathbf{x} = \sum_{n \in \mathbb{N} }^{  } [c_n] p^n
	,\end{equation*}
	where $c_n \in A$.
	We will refer to this expansion as the {\em Teichmüller expansion} in the future.
	Moreover, given $\mathbf{x} = \left( a_n \right)_{n \in \mathbb{N}}$, one can also write it as
	\begin{equation*}
		\mathbf{x} = \sum_{n \in \mathbb{N} }^{  } V^n([a_n])
	.\end{equation*}
	Then, since $A$ is perfect, $F$ is invertible on $W(A)$ too
	and we obtain $V^n = p^n F^{-n}$. Then we can rewrite the above sum as
	\begin{equation*}
		\mathbf{x} = 
		\sum_{n \in \mathbb{N} }^{  } p^n F^{-n}([a_n]) =
		\sum_{n \in \mathbb{N} } [a_n^{p^{-n}}] p^n
	.\end{equation*}
	All combined we can explicitly compute the coefficients
	of the Teichmüller expansion of $\mathbf{x} = \left( a_n \right)_{n \in \mathbb{N}}$,
	via $[c_n] = [a_n^{p^{-n}}]$.
\end{enumerate}
\end{rem}


\begin{prop}[{\cite[Proposition 4.2.3]{Brinon}}]\label{UPWittVectors}
	If $A$ is a perfect $\mathbb{F}_p$-algebra and $B$ is a $p$-ring,
	then the natural "reduction" map 
	\begin{equation*}
	\begin{tikzcd}
		\mathrm{Hom}_{  } \left( W(A), B \right)
		\arrow[r, "\sim", rightarrow] &
		\mathrm{Hom}_{  } \left( A, B/\mathfrak{b}_1 \right)
	\end{tikzcd}
	\end{equation*}
	is an isomorphism.
	More generally, for any strict $p$-ring $\mathcal{B}$, the natural map
	\begin{equation*}
	\begin{tikzcd}
		\mathrm{Hom}_{  } \left( \mathcal{B}, B \right)
		\arrow[r, "\sim", rightarrow] &
		\mathrm{Hom}_{  } \left( \mathcal{B}/ (p), B/\mathfrak{b}_1 \right)
	\end{tikzcd}
	\end{equation*}
	is bijective for every $p$-ring $B$.
\end{prop}


\begin{rem}[]\leavevmode\vspace{-.2\baselineskip}
\begin{enumerate}
\item Let's notice that, by the previous comments, we have
	$W(A)/ (p) = A$, which allows the definition of the first map above.
	Moreover the above shows that $\mathcal{B}$ and $W(\mathcal{B}/p \mathcal{B})$
	satisfy the same universal property in the category of $p$-rings.
	As a consequence all strict $p$-rings are isomorphic to $W(A)$ for
	a perfect $\mathbb{F}_p$-algebra $A$.

\item The inverse to the second bijection has the following form.
	Let $h \in \mathrm{Hom}_{  } \left( \mathcal{B}/ (p), B/\mathfrak{b}_1 \right)$
	and write an element $x \in \mathcal{B}$ using its expansion via $r_{\mathcal{B}}$.
	The induced map on $\mathcal{B}$ acts by
	\begin{equation*}
	\begin{tikzcd}[row sep = 0ex
		,/tikz/column 1/.append style={anchor=base east}
		,/tikz/column 2/.append style={anchor=base west}]
		x = \sum_{n \in \mathbb{N} }^{  } r_{\mathcal{B}}(x_n) p^n 
		\arrow[r, "", mapsto] & 
		\sum_{n \in \mathbb{N} }^{  } r_B(h(x_n)) p^n
	.\end{tikzcd}
	\end{equation*} 

\item Using the above proposition we can recover the theory of maximal unramified
	extensions for finite extensions $K/\mathbb{Q}_p$.
	Let $k \coloneqq O_k/ (\pi_K)$ be its residue field.
	In fact $O_K$ is a $p$-ring, when considered with the filtration
	given by $\left\{ \pi_K^i \right\}_{i \geq 1}$.
	Then there is a unique map of rings $\alpha\colon W(k) \to O_K$
	lifting the isomorphism $W(k)/ (p) \simeq k \simeq O_K/ (\pi_K)$.
	Since $p$ has image in the maximal ideal of $O_K$,
	the map $\alpha$ is local and injective.
	This means that $O_K/ (p)$ is a $W(k)/ (p) = k$ vector space
	with basis $\left(1, \pi_K, \ldots, \pi_K^e\right)$,
	where $e = e(K/\mathbb{Q}_p)$ is the absolute ramification
	index of $K$.
	Since $O_K$ is complete and separated with respect to the $p$-adic topology,
	by successive approximations, we see that the above is a
	$W(k)$-basis of $O_K$, which is then a free $W(k)$-module
	of rank $e$.
	As a consequence also $K = O_K[1/p]$ is a $W(k)[1/p] \eqqcolon K_0$
	vector space of dimension $e$.
	Then $K/K_0$ is a field extension of degree $e$, which is totally ramified
	since the two fields have isomorphic residue fields.
	Then we see that the Witt vector construction allows one to
	construct $K_0$ the {\em maximal unramified subextension} of any
	finite extension of $\mathbb{Q}_p$.

\item We can finally notice that the isomorphism $W(k) \simeq O_{K_0}$ preserves
	Teichmüller lifts. Hence, recalling the motivational remark at the beginning
	of the section, we can use the definition of sum and product via
	Witt polynomials on $W(k)$ to compute the operations on
	Teichmüller expansions on $O_{K_0}$.
\end{enumerate}
\end{rem}



\subsection{Classification of \texorpdfstring{$p$}{p}-divisible groups
	over \texorpdfstring{$O_{ K }$}{the ring of integers of K}}
This section will follow \cite[Appendix A]{Kisin}.
Fix $k$ a perfect field of characteristic $p$ and denote by
$W \coloneqq W(k)$ its ring of Witt vectors and by $K_0 \coloneqq W[1/p]$
its field of fractions.
Finally fix $K/K_0$ a finite totally ramified extension.
We denote by $\pi$ a fixxed uniformizer of $K$ and by $E(u) \in W[u]$
its (Eisenstein) minimal polynomial.


\begin{rem}[]
	Let $T$ be a scheme and $G \in \mathsf{BT}(T)$.
	The formation of $\mathbb{D}(G)$ and of $\mathbb{D}^*(G)$ is compatible
	with all base changes. %(tk: why?).
	In particular, if $p = 0$ on $T$ we can pull $G$ back by the Frobenius $\varphi$ on $T$.
	Then the relative Frobenius on $G$ gives a map $G \to \varphi^*(G)$
	hence a map of crystals
	\begin{equation*}
	\begin{tikzcd}
		\varphi^* (\mathbb{D}^*(G)) \arrow[r, "\sim", rightarrow] &
		\mathbb{D}^*(\varphi^*(G)) \arrow[r, "", rightarrow] &
		\mathbb{D}^*(G)
	.\end{tikzcd}
	\end{equation*}
\end{rem}


\begin{ntt}[]
	In the following we will mainly be interested in the evaluation of
	crystals on objects.
	In particular, given a scheme $T_0$, $G_0 \in \mathsf{BT}(T_0)$ and 
	$T_0 \hookrightarrow T \in \mathsf{Crys}(T_0)$, we will be interested
	in $\mathbb{D}^*(G_0)_{T}(T) = \mathbb{D}^*(G_0)_{T_0 \hookrightarrow T}(T)$,
	where $\mathbb{D}^*(G_0)_T$ is defined in \cref{CrystalAssociatedZarSheafS}
	and its evaluation in \cref{rem:SheavesCrysX}.
	Then we introduce the shorter notation
	$\mathbb{D}^*(G_0)(T) \coloneqq \mathbb{D}^*(G_0)_T(T)$.
	Moreover, if $T = \mathrm{Spec}(A)$ is affine, we introduce the notation
	$\mathbb{D}^*(G)(A)$ for $\mathbb{D}^*(G)(T)$.
	Finally we notice that, fixed $G$, the functor $\mathbb{D}^*(G)$
	is contravariant when evaluated on schemes, but covariant on rings.
\end{ntt}


\begin{rem}[]\label{rem:DGValuationRestriction}
	Suppose that $T_0$ is a scheme over $W$ and that $p = 0$ on $T_0$.
	Consider $G_0 \in \mathsf{BT}(T_0)$ and $T_0 \hookrightarrow T \in \mathsf{Crys}(T_0/W)$
	an object on which $p$ is locally nilpotent, and $G$ a lifting of $G_0$ to $T$.
	By construction of $\mathbb{D}^*$ we have an isomorphism
	\begin{equation*}
	\begin{tikzcd}[row sep = 0ex
		,/tikz/column 1/.append style={anchor=base east}
		,/tikz/column 2/.append style={anchor=base west}]
		\mathbb{D}^*(G_0)(T) \arrow[r, "\sim", rightarrow] &
		\mathbb{D}^*(G)(T)
	.\end{tikzcd}
	\end{equation*} 
	%In fact, as defined above, the left hand side is just
	%$\mathbb{D}^*(G_0)_{T_0 \hookrightarrow T}(T)$ and the right hand side
	%$\mathbb{D}^*(G)_{T \hookrightarrow T}(T)$.
	%Then it suffices to prove the equality of Zariski sheaves on $T$
	%\begin{equation*}
	%\mathbb{D}^*(G_0)_{T_0 \hookrightarrow T} = \mathbb{D}^*(G)_{T \hookrightarrow T}
	%.\end{equation*}
	%In fact, in general, a nilpotent closed immersion $U_0 \hookrightarrow U$
	%induces a morphism $\mathsf{Crys}(U/S) \to \mathsf{Crys}(U_0/S)$
	%by base changing $T \to U$ an open immersion to $U_0$.
	Moreover the $\mathcal{O}_{ T }$-module $\mathbb{D}^*(G)(T)$ sits in an exact sequence
	\begin{equation*}
	\begin{tikzcd}
		0 \arrow[r, "", rightarrow] &
		\left( \mathrm{Lie}(G) \right)^D \arrow[r, "", rightarrow] &
		\mathbb{D}^*(G)(T) \arrow[r, "", rightarrow] &
		\mathrm{Lie}(G^D) \arrow[r, "", rightarrow] &
		0
	.\end{tikzcd}
	\end{equation*}
\end{rem}


\noindent
We recall that elements of $\mathsf{Crys}(X/S)$ are defined by
locally nilpotent sheaf of ideals.
In what follows, though, we want to evaluate crystals on
surjections of $p$-adically complete rings, whose kernel is endowed with divided powers
(not necessarily nilpotent).
\begin{defn}[]\label{defn:NotNilpotentEvaluation}
	Let $A \twoheadrightarrow A_0$ be a surjective homomorphism
	of $p$-adically complete and separated $\mathbb{Z}_{p}$-algebras
	whose kernel is equipped with divided powers, compatible with those on $p \mathbb{Z}_{p}$.
	Take $G \in \mathsf{BT}(A_0)$.
	Denote by $G_n$ the restriction of $G$ to $A_0/p^nA_0$.
	Then we define
	\begin{equation*}
		\mathbb{D}^*(G)(A) \coloneqq
		\varprojlim_{n \in \mathbb{N}} \mathbb{D}^*(G_1)(A/p^nA)
	.\end{equation*}
\end{defn}


\begin{rem}[]
	Notice that, in the above definition, $A_n \twoheadrightarrow A_0/p^nA_0$
	has kernel equipped with divided powers.
	In fact it is the projection of the kernel of $A \twoheadrightarrow A_0/p^nA_0$, over
	$p^nA$.
	This ideal has divided powers since $pA$ and $\ker (A \to A_0)$ have compatible
	divided powers.
	Analogously we see that the kernel of $A_n \twoheadrightarrow A_0/p^mA_0$
	has divided powers for all $m \leq n$.
	Now, combining this remark with \cref{rem:DGValuationRestriction},
	we see that the above definition could have easily been swapped out with
	\begin{equation*}
		\mathbb{D}^*(G)(A) \coloneqq
		\varprojlim_{n \geq m} \mathbb{D}^*(G_m)(A/p^nA)
	\end{equation*}
	by cofinality of the family $n \geq m$ in $\mathbb{N}$.
\end{rem}


\begin{rem}[]
	We will now introduce some technical lemmas, which will play a crucial role
	in the proof of the main result of this section, \cref{prop:A6Kisin}.
	In fact these lemmas allow, in combination with \cref{thm:EquivCatBTS},
	to lift not only a Barsotti-Tate group along a thickening with divided
	powers, but to also lift some additional structure, i.e.
	a Frobenius morphism and a filtration, on the evaluation of $\mathbb{D}^*$.
	It is, in fact, this extra structure that is going to grant
	uniqueness for the lifts we are interested in.
\end{rem}


\begin{lem}[{\cite[Lemma A.2]{Kisin}}]\label{lem:A2Kisin}
	Let $A \to A_0$ be a surjection of $p$-adically complete
	and separated local $\mathbb{Z}_{p}$-algebras with residue field
	$k$ and kernel $\mathrm{Fil}^1 A$ equipped with divided powers.
	Suppose moreover that
\begin{enumerate}
	\item $A$ is $p$-torsion-free and it is equipped with an endomorphism
		$\varphi\colon A \to A$ lifting the Frobenius on $A/pA$;
		
	\item the following map, induced on the pullback, is surjective
		\begin{equation*}
		\begin{tikzcd}
			\mathrm{id}_{ A } \otimes \varphi/p \colon
			\varphi^*(\mathrm{Fil}^1 A)
			\arrow[r, "", twoheadrightarrow] &
			A
		.\end{tikzcd}
		\end{equation*}
\end{enumerate}
	If $G \in \mathsf{BT}(A_0)$ we write $\mathrm{Fil}^1 \mathbb{D}^*(G)(A) \subset \mathbb{D}^*(G)(A)$
	for the preimage of $\left( \mathrm{Lie}(G) \right)^D$ inside $\mathbb{D}^*(G)(A_0)$.
	Then the restriction of $\varphi\colon \mathbb{D}^*(G)(A) \to \mathbb{D}^*(G)(A)$
	to $\mathrm{Fil}^1 \mathbb{D}^*(G)(A)$ is divisible by $p$ and the following induced map
	is a surjection
	\begin{equation*}
	\begin{tikzcd}
		1 \otimes \varphi/p \colon
		\varphi^*\mathrm{Fil}^1 \mathbb{D}^*(G)(A)
		\arrow[r, "", twoheadrightarrow] &
		\mathbb{D}^*(G)(A)
	.\end{tikzcd}
	\end{equation*}
\end{lem} 


\begin{rem}[]\label{rem:A2Kisin}
	Here are the main ingredients of the proof.
\begin{enumerate}
	\item Given an ideal with divided powers $I \triangleleft A$, then $\varphi(I) \subset pA$.
		In fact 
		\begin{equation*}
			\varphi(x) = x^p = \gamma_p(x) \cdot p! \in pA
		.\end{equation*}

	\item Let $\widetilde{G}$ be a lift of $G$ to $A$, then we have
		\begin{equation*}
			\mathrm{Fil}^1(\mathbb{D}^*(\widetilde{G})(A)) =
			( \mathrm{Lie} \widetilde{G} )^D +
			\mathrm{Fil}^1A \cdot \mathbb{D}^*(\widetilde{G})(A)
		.\end{equation*}
		
	\item Let $H \in \mathsf{BT}(\mathrm{Spec}(W(k))$. Denote by $H_0$ the restriction
		of $H$ to $k$ and by $V$ the Verschiebung morphism.
		Then $W(k) \twoheadrightarrow k$ satisfies the hypothesis of our lemma,
		and the kernel $\mathrm{Fil}^1(W(k)) = (p)$ is equipped with divided powers.
		Then, using the theory of Dieudonné modules, one can see that
		\begin{equation*}
			( \mathrm{Lie} H)^D + p \mathbb{D}^*(H)(W(k)) =
			V \mathbb{D}^{*}(H)(W(k))
		,\end{equation*}
		i.e. that
		$V \mathbb{D}^*(H)(W(k)) = \mathrm{Fil}^1 \mathbb{D}^*(H)(W(k))$.
\end{enumerate}
\end{rem}


\begin{defn}[Special ring]
	A {\em special ring} $A$ is a $p$-adically complete, separated, $p$-torsion-free,
	local $\mathbb{Z}_{p}$-algebra equipped with an endomorphism $\varphi$
	lifting the Frobenius on $A/pA$.
	Moreover we call {\em map of special rings} a morphism
	of $\mathbb{Z}_{p}$-algebras compatible with $\varphi$.
\end{defn}


\begin{defn}[]
	Let $A$ be a special ring. We define the category
	$\mathsf{C}_A$ whose objects are finite, free $A$-modules
	$M$ equipped with a semilinear Frobenius map $\varphi\colon M \to M$
	and an $A$-submodule $M_1 \subset M$ such that $\varphi(M_1) \subset pM$
	and such that the map
	\begin{equation*}
	\begin{tikzcd}[row sep = 0ex
		,/tikz/column 1/.append style={anchor=base east}
		,/tikz/column 2/.append style={anchor=base west}]
		\mathrm{id}_{ A }\otimes\varphi/p\colon 
		\varphi^*(M_1) \arrow[r, "", twoheadrightarrow] &
		M
	\end{tikzcd}
	\end{equation*} 
	is surjective.
	Its morphisms are morphisms of $A$-modules compatible with
	the Frobenius and submodules.
\end{defn}


\begin{rem}[]
	Notice that \cref{lem:A2Kisin} allows to endow,
	for $G \in \mathsf{BT}(A_0)$ and $A \twoheadrightarrow A_0$ as in the hypothesis,
	the module $\mathbb{D}^*(G)(A)$ with the structure of an object in $\mathsf{C}_A$.
\end{rem}


\begin{defn}[]
	Consider a map of special rings $A \to B$ and $M \in \mathsf{C}_A$.
	Then $M \otimes_A B \in \mathsf{C}_B$, when equipped with
	the induced Frobenius %(tk: what is it?) 
	and setting
	$\left( M \otimes_A B \right)_1$ to be the image of
	$M_1 \otimes_A B$ in $M \otimes_A B$.
\end{defn}


\begin{lem}[{\cite[Lemma A.4]{Kisin}}]\label{lem:A4Kisin}
	Let $h\colon A \to B$ be a surjection of special rings with kernel $J$.
	Suppose that, for all $i \geq 1$, $\varphi^i(J) \subset p^{i + j_i} J$,
	where $\left\{ j_i \right\}_{i \geq 1}$ is a sequence of integers
	such that $\lim_{i \to \infty} j_i = \infty$.
	Consider $M, M' \in \mathsf{C}_A$ and
	\begin{equation*}
	\begin{tikzcd}[row sep = 0ex
		,/tikz/column 1/.append style={anchor=base east}
		,/tikz/column 2/.append style={anchor=base west}]
		\theta_B\colon M \otimes_A B 
		\arrow[r, "\sim", rightarrow] &
		M' \otimes_A B
	\end{tikzcd}
	\end{equation*} 
	an isomorphism in $\mathsf{C}_B$.
	Then there exists a unique isomorphism of $A$-modules
	$\theta_A\colon M \to M'$ lifting $\theta_B$ and compatible with $\varphi$.
\end{lem} 


\begin{rem}[]
	As before we will present only the main idea behind the proof.
	In this case, starting from any lift $\theta_0\colon M \to M'$
	of $\theta_B$, we deform iteratively the map
	\begin{equation*}
	\begin{tikzcd}[column sep=4.8em]
		\varphi^*(M_1) \arrow[r, "\varphi^*(\left.\theta_i\right|_{ M_1 })", rightarrow] &
		\varphi^*(M_1') \arrow[r, "1 \otimes \varphi/p", rightarrow] &
		M'
	\end{tikzcd}
	\end{equation*}
	obtaining a succession of maps $\theta_i$ whose successive difference
	has image lying in $p^{j_i}M'$.
	The fact that $j_i \to \infty$ allows to obtain a well defined limit.
\end{rem}


\noindent
In order to apply the above results to the proof of the main theorem of the section
we still need some remarks and definitions.
Let's start by defining the ring in which we will take
our ideals with divided powers.
\begin{defn}[]
	Consider $W$ as a divided power ring, endowing the maximal ideal $(p)$
	with the divided powers structure defined in \cref{PDex:DVR} of \cref{PDexamples}.
	Consider $W[u]$ as a $W$-algebra, and take $\mathcal{D}_{W[u]}(E(u))$
	its divided powers envelope, constructed in \cref{thm:PDEnvelope}.
	Let's notice that, as outlined in \cref{rem:PDEnvelope},
	we can take $\overline{J} \triangleleft \mathcal{D}_{W[u]}(E(u))$ 
	as the ideal generated by $E(u), p$ and their divided powers,
	also called the P.D. ideal generated by $E(u)$ and $p$.
	We denote by $S$ the $p$-adic completion of $\mathcal{D}_{W[u]}(E(u))$
	and by $\mathrm{Fil}^1 S \subset S$ the closure of the ideal
	generated by $E(u)$ and its divided powers.
\end{defn}


\begin{rem}[]\label{rem:liftFrobeniusS}
	The ring $S$ is equipped with an endomorphism $\varphi$ lifting that on $S/pS$.
	It is, in fact, the lift to $S$ of the Frobenius, $F$, on $W[u]$ which act naturally
	on $W$ and by $u \mapsto u^p$.
	In order to lift it we use the universal property of
	the divided powers envelope.
	In particular we need to notice that this map sends $(E(u))$
	into $\overline{J}$, which contains $p$ as stated before.
	But this is clear, since $F$ sends $pW[u]$ to $pW[u]$ and, 
	since the P.D. ideal generated by $E(u)$ in $\mathcal{D}_{W[u]}(E(u))$
	contains divided powers, it sends $E(u)$ in the P.D. ideal
	generated by $E(u)$ and $p$.
	Then, by universal property of divided powers envelope, we obtain the desired 
	lift $\varphi$ as the unique map making the following diagram commute
	\begin{equation*}
	\begin{tikzcd}
		&
		\mathcal{D}_{W[u]}(E(u))
		\arrow[rd, "\varphi", dashrightarrow,
		start anchor=south east] & \\
		\left( W[u], (E(u),p)\right) \arrow[ru, "", rightarrow,
		end anchor=south west] 
		\arrow[rr, "F", rightarrow] & &
		\left( W[u], (E(u),p)\right) \subset
		\mathcal{D}_{W[u]}(E(u)). 
	\end{tikzcd}
	\end{equation*}
	Then this extends to $S$ by continuity with respect to $p$-adic
	topology and it lifts the Frobenius defined on
	$S/pS = \mathcal{D}_{W[u]}(E(u))/ (p)$.
\end{rem}


\begin{ntt}[]\label{not:phi1S}
	Thanks to \cref{PDExtendCompletion}, the ideal 
	$\mathrm{Fil}^1(S)$ is equipped with divided powers.
	Then, as seen in \cref{rem:A2Kisin}, 
	we have $\varphi( \mathrm{Fil}^1 S) \subset p S$, which
	allows us to define $\varphi_1 \coloneqq \varphi/p$.
\end{ntt}


\begin{rem}[]\label{rem:j_iTheoremA6}
	We will apply \cref{lem:A4Kisin} in the situation where $J$ is equipped
	with divided powers structure and there exists a finite set of 
	elements $x_1, \ldots, x_n \in J$ such that $J$, in the $p$-adic topology, 
	is topologically generated by the $x_i$ and their divided powers
	and such that $\varphi(x) = x^p$ on all $x_i$ and their divided powers.
	Then the integers $j_i$ above are $v_p \left( (p-1)! \right) - 1$.
	In fact, since $J$ is equipped with divided powers, we have for any $x \in J$
	\begin{equation*}
		\varphi^i(x) =
		x^{p^i} =
		x \cdot \gamma_{p^i-1}(x) \cdot (p^i-1)!
	.\end{equation*}
	Since both $x$ and $\gamma_{p^i-1}(x)$ are in $J$,
	we see that $\varphi^i(x) \in p^{\nu_p((p^i-1)!)}J$.
\end{rem}


\begin{defn}[]
	We denote by $\mathsf{BT}^{\varphi}_{/S}$ the category
	whose objects are finite free $S$-modules $M$
	equipped with an $S$-submodule $\mathrm{Fil}^1 M$
	and a $\varphi$-semilinear map $\varphi_1\colon  \mathrm{Fil}^1 M \to M$
	such that
\begin{enumerate}
	\item $\mathrm{Fil}^1 S \cdot M \subset \mathrm{Fil}^1 M$
		and the quotient $M/\mathrm{Fil}^1 M$ is a free $\mathcal{O}_{ K }$-module;
	\item the map $1 \otimes \varphi_1\colon 
		\varphi^* \left( \mathrm{Fil}^1 M \right) \to M$
		is surjective.
\end{enumerate}
\end{defn}


\begin{rem}[]\label{invertibilityphi1E(u))}
	Notice that any $M \in \mathsf{BT}^{\varphi}_{/S}$ is equipped with a 
	$\varphi$-semilinear map $\varphi\colon M \to M$ defined by
	\begin{equation*}
		\varphi (x) \coloneqq \varphi_1 \left( E(u) \right)^{-1} \varphi_1 (E(u)x)
	.\end{equation*}
	In fact $\varphi_1(E(u))$ is invertible in $S$.
	To show this we recall that, by definition, $\varphi_1$ is
	linear on $W[u]$, and that $W[u]$ embeds in $S$.
	More explicitly we need to show that the element
	\begin{equation*}
		\varphi_1(E(u)) =
		\underbrace{\frac{ u^{ep} }{ p } + \varphi_1(a_{e-1}) u^{(e-1)p} +
		\cdots + \varphi_1(a_1)u^p}_{x} + \varphi_1(a_0)
	\end{equation*}
	is invertible in $S$.
	To do so it suffices to show that $\varphi_1(a_1) \in S^{\cross}$ and
	that $x$ is nilpotent modulo $p$. 
	In fact, in such case, since $S$ is complete with respect to the
	$p$-adic topology, we can compute an inverse of $\varphi_1(E(u))$
	in $S$ via successive approximations (essentially in the same
	way one proves this for power series).
	Let's start with $x$, for which it suffices to show that 
	$u^{ep}/p$ and all $\varphi_1(a_i)$ are in $\mathrm{Fil}^1S$.
	In such case, then $x \in \mathrm{Fil}^1S$, which has divided powers,
	and $x^p = p! \gamma_p(x) \in pS$.
	Let's recall that we defined $\mathrm{Fil}^1S$ as the topological
	closure of the ideal $(E(u),p) \triangleleft \mathcal{D}_{W[u]}(E(u))$
	and that it is equipped with divided powers.
	Then $\varphi_1(a_i) = a_i^p/p = p!/p \cdot \gamma_p(a_i) \in \mathrm{Fil}^1S$,
	and analogously for $\varphi_1(u^e)$, allows us to state
	that $x \in \mathrm{Fil}^1S$.
	With regards to $\varphi_1(a_0)$ we need to show that it 
	is invertible modulo $p$.
	But this is true, since $\varphi(p) = p$ on $W$, hence on $S$,
	and $a_0 = p \cdot \alpha$, for $\alpha \in W^{\cross}$.
	In fact this grants that $\varphi_1(a_0) = \varphi(p \alpha)/p = \varphi(\alpha)$.
	Since $\varphi$ lifts the Frobenius on $W$, and $\alpha \in W$,
	which itself lifts the Frobenius on $k$, we see that $\alpha$
	is invertible in $W/pW \simeq k$, and $\varphi(\alpha)$ too,
	which allows us to conclude.
\end{rem}


\begin{prop}[{\cite[Proposition A.6]{Kisin}}]\label{prop:A6Kisin}
	There is an exact contravariant functor
	\begin{equation*}
	\begin{tikzcd}[row sep = 0ex
		,/tikz/column 1/.append style={anchor=base east}
		,/tikz/column 2/.append style={anchor=base west}]
		M\colon \mathsf{BT}(\mathcal{O}_{ K }) \arrow[r, "", rightarrow] &
		\mathsf{BT}^{\varphi}_{/S} \\
		G \arrow[r, "", mapsto] & 
		\mathbb{D}^*(G)(S) \eqqcolon M(G)
	.\end{tikzcd}
	\end{equation*} 
	If $p > 2$ this functor is an anti equivalence,
	whereas if $p = 2$ it induces an anti-equivalence of the 
	corresponding isogeny categories.
\end{prop}
\begin{proof}\leavevmode\vspace{-.2\baselineskip}
We start by showing that this functor actually takes values in $\mathsf{BT}^{\varphi}_{/S}$.
To do so we need to show that $M(G) = \mathbb{D}^{*}(G)(S)$
has a structure of object in $\mathsf{BT}^{\varphi}_{/S}$.
It is enough to check that $S \twoheadrightarrow O_K$ satisfies
the hypothesis of \cref{lem:A2Kisin}.
Notice that the map $S \to O_K$ is induced by $W[u] \twoheadrightarrow O_K$,
which we already remarked being a surjection.
Then, by construction, we see that $S \twoheadrightarrow O_K$
is surjective too.
It is also clear that both $S$ and $O_K$ are 
$p$-adically complete, separated, local $\mathbb{Z}_{p}$-algebras.
Moreover, by definition, $\ker (W[u] \to O_K) = (E(u))$.
Then, by construction of $S$ and completeness of $O_K$,
we see that $\mathrm{Fil}^1 S$, the kernel of $S \twoheadrightarrow O_K$, 
as seen in \cref{not:phi1S}, is the P.D. ideal topologically generated by $E(u)$ and is
equipped with divided powers.
Moreover, by construction, it is clear that $S$ is $p$-torsion-free
and, thanks to \cref{rem:liftFrobeniusS},
that it is equipped with an endomorphism lifting the Frobenius on $S/pS$.
Finally we are left to prove that the map
\begin{equation*}
\begin{tikzcd}[row sep = 0ex
	,/tikz/column 1/.append style={anchor=base east}
	,/tikz/column 2/.append style={anchor=base west}]
	\mathrm{id}_{ S } \otimes \varphi/p \colon 
	\varphi^*(\mathrm{Fil}^1 S) \arrow[r, "", twoheadrightarrow] &
	S
\end{tikzcd}
\end{equation*} 
is surjective.
Here, thanks to the argument in \cref{invertibilityphi1E(u))}, we easily conclude.
In fact we see that, for all $s \in S$,
\begin{equation*}
\begin{tikzcd}[row sep = 0ex
	,/tikz/column 1/.append style={anchor=base east}
	,/tikz/column 2/.append style={anchor=base west}]
	s \cdot \left( \varphi_1(E(u)) \right)^{-1} \otimes E(u) 
	\arrow[r, "", mapsto] & 
	s
.\end{tikzcd}
\end{equation*} 



The construction of the quasi-inverse $M \mapsto G(M)$
is definitely more tricky and we will concentrate only on
the case $p > 2$, leaving the rest of the proof to \cite{Kisin}.
It will require the use of classical Dieudonné theory to construct a
Barsotti-Tate group over $k$ associated to a module $M \in \mathsf{BT}^{\varphi}_{/S}$
and then to iteratively lift it from $k = O_K/\pi O_K$ to $O_K$.
This difficulty is due to the fact that, in general, the maximal
ideal of $O_K$, i.e. the kernel of $O_K \twoheadrightarrow k$, 
does not admit divided powers.
Then the only hope to lift the classical construction is to
procede iteratively, reducing the above projection to a sequence of
smaller projections, whose kernels all have nilpotent divided powers structures.
In particular those lifting steps will make use of Grothendieck-Messing
deformation theory to lift the Barsotti-Tate group, and of the previous
technical lemmas to lift the structure of module in $\mathsf{BT}^{\varphi}_{/S}$
associated to the $p$-divisible group via the crystal $\mathbb{D}^*(G)$.

%More specifically, using the previous technical lemmas and Grothendieck-Messing deformation theory,
%we will iteratively lift $G_1$, the Barsotti-Tate group coming from calssical Dieudonné theory,
%to $G_i \in \mathsf{BT}(O_K/\pi^iO_K)$ until $i = e$.
%Finally, thanks to completeness of $O_K$ and \cite[Lemma 2.4.4]{deJong},
%we will carry out another inductive process of lifting
%to define $G \in \mathsf{BT}(O_K)$.
%The idea is that at each step we obtain a group with an isomorphism
%\begin{equation*}
%\begin{tikzcd}[row sep = 0ex
%	,/tikz/column 1/.append style={anchor=base east}
%	,/tikz/column 2/.append style={anchor=base west}]
%	\mathbb{D}^*(G_i)(R_i) \arrow[r, "\sim", rightarrow] &
%	M_i
%\end{tikzcd}
%\end{equation*} 
%compatible with both filtration and $\varphi$, in such a way that these isomorphisms
%lift to a global one, still compatible with all the necessary structure,
%which in turn grants uniqueness of this construction.

Let's now start the proof by introducing the necessary notation:
let $i$ vary in $1, \ldots, e$ and set $R_i \coloneqq W[u]/ (u^i)$.
Clearly $R_i$ is equipped with a Frobenius endomorphism $\varphi$
given by the usual one on $W$ and $u \mapsto u^p$ on the indeterminate.
These are all $S$-algebras.
In fact, by universal property of divided powers envelope
we have a ring homomorphism $S \to R_i$ given by the unique map associated to
\begin{equation*}
\begin{tikzcd}[row sep = 0ex
	,/tikz/column 1/.append style={anchor=base east}
	,/tikz/column 2/.append style={anchor=base west}]
	W[u] \arrow[r, "", rightarrow] &
	R_i \\
	u \arrow[r, "", mapsto] & u \\
	u^{ej} / j!  \arrow[r, "", mapsto] & 0
.\end{tikzcd}
\end{equation*} 
This map is compatible with $\varphi$ and by uniqueness also
the induced map on $S$ is.
Moreover also $O_K/ (\pi^i)$ is an $R_i$-algebra by $u \mapsto \pi$,
seeing $W = O_{K_0} \subset O_K$.
As one can check writing the elements of $O_K$ in Teichmüller expansion
the map $R_i \twoheadrightarrow O_K/ (\pi^i)$ is a surjection
with kernel $pR_i$, which is equipped with divided powers by \cref{lem:PDFlatExtension}.
Then, given a $p$-divisible group $G_i \in \mathsf{BT}(O_K/\pi^iO_K)$,
we can consider its evaluation $\mathbb{D}^*(G_i)(R_i)$ thanks to \cref{defn:NotNilpotentEvaluation}.
Following the notation of \cref{lem:A2Kisin}
we denote by $\mathrm{Fil}^1\mathbb{D}^*(G_i)(R_i)$
the preimage of $\left( \mathrm{Lie}G_i \right)^D \subset \mathbb{D}^*(G_i)(O_K/\pi^iO_K)$
in $\mathbb{D}^*(G_i)(R_i)$.

We define the module $M_i \coloneqq M \otimes_S R_i$,
which is a restriction of scalars of our original one and
it is also equipped with the diagonal action of $\varphi$.
We set $\mathrm{Fil}^1 M_i \subset M_i$
to be the image of $\mathrm{Fil}^1 M$ in $M_i$, which is a submodule
by surjectivity of $S \twoheadrightarrow R_i$.
Notice that the above corresponds to defining $\mathrm{Fil}^1 M_i \coloneqq
\mathrm{Fil}^1 M \otimes_S R_i$.
Then, by right exactness of tensor product, 
\begin{equation*}
\begin{tikzcd}[row sep = 0ex
	,/tikz/column 1/.append style={anchor=base east}
	,/tikz/column 2/.append style={anchor=base west}]
	\mathrm{id}_{ S } \otimes \varphi_1
	\colon 
	\varphi^*(\mathrm{Fil}^1 M)\arrow[r, "", rightarrow] &
	M
\end{tikzcd}
\end{equation*} 
induces a surjective map $\varphi^*(\mathrm{Fil}^1 M_i) \twoheadrightarrow M_i$
for all $i$.
In other words we have just seen that
$M_i \in \mathsf{C}_{R_i}$ for all $i \in [1,e]$.

With all this in mind we can start with the lifting process.
\begin{enumerate}
\item	At first we need to construct, using classical Dieudonné theory,
	a Barsotti-Tate group $G_1$ on $k = O_K/\pi O_k$ from $M_1$.
	Let's denote by $F\colon M_1 \to M_1$ the map induced by
	$\varphi\colon M \to M$.
	Then we see that both sides of the surjective map
	$\varphi^* ( \mathrm{Fil}^1 M_1) \to M_1$ are free $W$-modules of the same (finite) rank,
	as can be seen by base changing to $k$, which means that the map is an isomorphism.
	Now we consider the composition
	\begin{equation*}
	\begin{tikzcd}
		M_1 \arrow[r, "\sim", rightarrow] &
		\varphi^* (\mathrm{Fil}^1 M) \arrow[r, "", rightarrow] &
		\varphi^* (M_1) \arrow[r, "\sim", rightarrow] &
		M_1
	,\end{tikzcd}
	\end{equation*}
	where the first arrow is just the inverse of the above isomorphism, the second map
	is induced by the inclusion $\mathrm{Fil}^1 M \hookrightarrow M$ and the
	third one is given by $a \otimes m \mapsto \varphi^{-1}(a)m$.
	This composition gives a $\varphi^{-1}$-semilinear map $V\colon M_1 \to M_1$
	such that $FV = VF = p$.
	Let's denote by $G_1$ the Barsotti-Tate group on $k$ associated to this
	Dieudonné module, see e.g. \cite[Proposition 7.2.6]{Brinon} for a reference.
	In particular the isomorphism
	\begin{equation*}
	\begin{tikzcd}[row sep = 0ex
		,/tikz/column 1/.append style={anchor=base east}
		,/tikz/column 2/.append style={anchor=base west}]
		\mathbb{D}^*(G_1)(W) \arrow[r, "\sim", rightarrow] &
		M_1
	\end{tikzcd}
	\end{equation*} 
	is compatible with Frobenius.
	Moreover, thanks to \cref{rem:A2Kisin}, $\mathrm{Fil}^1 \mathbb{D}^*(G_1)(W)$
	can be identified with $V \mathbb{D}^*(G_1)(W)$, which grants that the isomorphism
	is also compatible with filtrations.

\item We now iteratively lift this construction.
	Assume, for $i \in [2,e]$, that we have an isomorphism
	\begin{equation}\label{eqn:PartialIsoCRi}
	\begin{tikzcd}
	\mathbb{D}^*(G_{i-1})(R_{i-1}) 
	\arrow[r, "\sim", rightarrow] &
	M_{i-1}
	\end{tikzcd}
	\end{equation}
	compatible with Frobenius and filtrations, i.e. a morphism in $\mathsf{C}_{R_{i-1}}$.
	We can notice that the kernel of $R_i \twoheadrightarrow O_K/ (\pi^{i-1})$
	is $(p, u^{i-1})$ which is still equipped with divided powers.
	In fact we simply put $\gamma_1(u^{i-1}) = u^{i-1}$ and
	$\gamma_n(u^{i-1}) = 0$ for all $n \geq 2$,
	since $u^{n(i-1)} \in (u^i)$.
	Then we are done invoking \cref{lem:PDExt2}.
	Again, this means that we can compute $\mathbb{D}^*(G_{i-1})(R_i)$.
	We have already seen that $M_i \in \mathsf{C}_{R_i}$, moreover \cref{lem:A2Kisin}
	applied to the surjection $R_i \twoheadrightarrow O_k/ (\pi^{i-1})$ implies that
	also $\mathbb{D}^*(G_{i-1})(R_i)$ is in $\mathsf{C}_{R_i}$.
	Recalling \cref{rem:j_iTheoremA6} and that the isomorphism in \cref{eqn:PartialIsoCRi}
	is a morphism in $\mathsf{C}_{R_{i-1}}$, 
	we can apply \cref{lem:A4Kisin} to the surjection $R_i \twoheadrightarrow R_{i-1}$
	and obtain a lift to an isomorphism
	\begin{equation*}
	\begin{tikzcd}
	\mathbb{D}^*(G_{i-1})(R_{i}) 
	\arrow[r, "\sim", rightarrow] &
	M_{i}
	\end{tikzcd}
	\end{equation*}
	compatible with Frobenius.
	Finally, since the kernel of $R_i \twoheadrightarrow R_{i-1}$ is nilpotent,
	we can invoke \cref{thm:EquivCatBTS} and obtain that
	there is a unique $G_i \in \mathsf{BT}(O_K/\pi^iO_K)$
	lifting $G_{i-1}$ and such that $(\mathrm{Lie}G_i)^D \subset \mathbb{D}^*(G_{i-1})(O_K/\pi^i O_K)$
	is equal to the image of $\mathrm{Fil}^i M_i$ under the composite
	%tk: here one should also prove that this image is locally free.
	%I guess this is true, since we have the short exact sequence
	%of the second rem of this section, and it should be that Lie is free
	%(then splitting, etc).
	\begin{equation*}
	\begin{tikzcd}
		\mathrm{Fil}^1 M_i \subset M_i
		\arrow[r, "\sim", rightarrow] &
		\mathbb{D}^*(G_{i-1})(R_i)
		\arrow[r, "", rightarrow] &
		\mathbb{D}^*(G_{i-1})(O_K/\pi^iO_K)
	.\end{tikzcd}
	\end{equation*}
	Then, by construction, we obtain that the isomorphism
	$\mathbb{D}^*(G_i)(R_i) \simeq M_i$, where we recall the implicit use of 
	\cref{rem:DGValuationRestriction}, is compatible also with filtrations,
	i.e. is in $\mathsf{C}_{R_i}$.
	This concludes the first step of induction.

\item Finally we need to reiterate the above argument to the surjection
	$S \twoheadrightarrow R_e$.
	At first we need to show that the kernel of this map is equipped with divided powers.
	Since it is the $p$-adic completion of the kernel
	of the map $\mathcal{D}_{W[u]}(E(u)) \twoheadrightarrow R_e$,
	induced by $W[u] \twoheadrightarrow R_e$, we can study this and
	then apply \cref{PDExtendCompletion}.
	But this kernel is the P.D. ideal topologically generated 
	by $u^e$ and $p$ hence it is equipped with divided powers,
	since $(E(u)) \triangleleft\, \mathcal{D}_{W[u]}(E(u))$ has a P.D. structure
	compatible with that of $(p)$ (and it is an Eisenstein polynomial).
	%Notice that here I need to apply the remark on the valuations, so 
	%I need generators (topological) for which the Frobenius acts
	%like the $p$-th power map. So E(u) is not good.
	%tk: if you have time try to expand on this later.
	Then we can apply \cref{lem:A2Kisin} and obtain that the valuation
	$\mathbb{D}^*(G_e)(S)$ is in $\mathsf{C}_{S}$.
	As before, thanks to \cref{lem:A4Kisin} we lift the isomorphism 
	\begin{equation*}
	\begin{tikzcd}
		M_e \arrow[r, "\sim", rightarrow] &
		\mathbb{D}^*(G_e)(R_e)
	\end{tikzcd}
	\end{equation*}
	in $\mathsf{C}_{R_e}$ to an isomorphism $M \simeq \mathbb{D}^*(G_e)(S)$
	compatible with $\varphi$.
	Here notice that we need \cref{rem:j_iTheoremA6} to invoke
	\cref{lem:A4Kisin}, and we can argue as in \cref{rem:j_iTheoremA6}
	since the kernel of the projection $S \twoheadrightarrow R_e$ is
	the P.D. ideal topologically generated by $u^e$ and $p$,
	on which the Frobenius acts as the $p$-th power map.
	Now we need to lift the group $G_e$ to $O_K$ for which we assume that $p > 2$.
	In order to do so we notice that, for all $i$,
	$O_K/p^{i-1}O_K \twoheadrightarrow O_K/ (p^i)$ and
	$S \twoheadrightarrow O_K/ (p^i)$ have kernels equipped with
	divided powers, the first being nilpotent, the second existing by compatibility
	of divided powers on $u^e$ with those on $p$.
	Moreover, since $O_K$ is $p$-adically complete, thanks to \cite[Lemma 2.4.4]{deJong},
	we see that the datum of a Barsotti-Tate group on $O_K$ is equivalent to
	the datum of a compatible sequence of Barsotti-Tate groups $G^i$ on $O_K/ (p^i)$.
	Here compatible means that the restriction of $G^i$ to $O_K/ (p^{i-1})$
	is isomorphic to $G^{i-1}$.
	Then, invoking again \cref{thm:EquivCatBTS}, we can construct this sequence by induction,
	taking at each time the group $G^i$ determined by $(\mathrm{Lie} G^i)^D \subset
	\mathbb{D}^*(G_e)(O_K/p^iO_K)$ given by the image of $\mathrm{Fil}^1 M$ under
	\begin{equation*}
	\begin{tikzcd}
		M \arrow[r, "\sim", rightarrow] &
		\mathbb{D}^*(G_e)(S) \arrow[r, "", rightarrow] &
		\mathbb{D}^*(G_e)(O_K/p^iO_K)
	.\end{tikzcd}
	\end{equation*}
	By functoriality of $\mathbb{D}^*(G_e)$ this gives rise to a compatible family $G^i$,
	which in turn, by \cite[Lemma 2.4.4]{deJong}, defines $G(M) \in \mathsf{BT}(O_K)$
	such that $(\mathrm{Lie}G)^D \subset \mathbb{D}^*(G_e)(O_K)$ is
	equal to the image of $\mathrm{Fil}^iM$ under
	\begin{equation*}
	\begin{tikzcd}
		M \arrow[r, "\sim", rightarrow] &
		\mathbb{D}^*(G_e)(S) \arrow[r, "", rightarrow] &
		\mathbb{D}^*(G_e)(O_K)
	.\end{tikzcd}
	\end{equation*}
\end{enumerate}
At last we are only left to prove that the above morphisms are quasi inverses to each other.
It is clear, by construction, that $M \simeq M(G(M))$.
For the other direction we see, by induction on $i = 1, \ldots, e$, that uniqueness in
\cref{PDExtendCompletion} grants that, given any $G \in \mathsf{BT}(O_K)$, 
the group $G_i(M(G))$ is isomorphic to the base change of $G$ to $O_K/ (\pi^i)$.
Analogously, for $i \in \mathbb{N}$, uniqueness in \cref{PDExtendCompletion} grants
that $G^i(M(G))$ is isomorphic to the base change of $G$ to $O_K/ (p^i)$.
Then, invoking again \cite[Lemma 2.4.4]{deJong}, we can conclude that $G(M(G)) \simeq G$
and the two functors are quasi-inverses to one another.
\end{proof}
